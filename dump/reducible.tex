\documentclass{amsart}


%%%%%%%%%%%%%%%%%%%%%%%%%%%%%% packages %%%%%%%%%%%%%%%%%%%%%%%%%%%%%%

% Here are some common packages used in almost all documents.
\usepackage{amssymb}
\usepackage{graphicx}
\usepackage{color}
\usepackage[colorlinks]{hyperref}


%%%%%%%%%%%%%%%%%%%%%%% amsthm theorem styles %%%%%%%%%%%%%%%%%%%%%%%%

% Current styles are based on suggestions by the amsthm package
% authors.
\theoremstyle{plain}
\newtheorem{theorem}{Theorem}[section]
\newtheorem{lemma}[theorem]{Lemma}
\newtheorem{proposition}[theorem]{Proposition}
\newtheorem{corollary}[theorem]{Corollary}
\newtheorem{claim}[theorem]{Claim}

\theoremstyle{definition}
\newtheorem{definition}[theorem]{Definition}
\newtheorem{example}[theorem]{Example}
\newtheorem{conjecture}[theorem]{Conjecture}

\theoremstyle{remark}
\newtheorem{remark}[theorem]{Remark}
\newtheorem{note}[theorem]{Note}
\newtheorem{case}{Case}


%%%%%%%%%%%%%%%%%%%%%%%%% custom definitions %%%%%%%%%%%%%%%%%%%%%%%%%

% Since we are collaborating, we need to be careful with personal
% shortcuts to avoid clash.
\newcommand{\C}{\mathbb{C}}
\newcommand{\F}{\mathbb{F}}
\renewcommand{\P}{\mathbb{P}}
\newcommand{\Q}{\mathbb{Q}}

\newcommand{\preper}{\mathrm{PrePer}}
\newcommand{\gal}{\mathrm{Gal}}

\newcommand{\tup}[1]{\left<#1\right>}
\newcommand{\set}[1]{\left\{#1\right\}}


%%%%%%%%%%%%%%%%%%%%%%%%%%% document body %%%%%%%%%%%%%%%%%%%%%%%%%%%%

\title{Structure of $\Phi_{N,c}(z)$}
\author{Zhiming Wang}
\date{\today}

\begin{document}

\maketitle

According to Morton paper I, for $c \in \Q$, the structure of
$\Phi_{3,c}(z)$ over $\Q$ (degrees of irreducible factors) is always
one of the following: $\set{6}$, $\set{3,\, 3}$, $\set{1,\, 1,\, 1,\,
  3}$. And all three occur for infinitely many $c \in \Q$.

According to my computational results for $\Phi_{4, c}(z)$
\href{http://goo.gl/wgS3dn}{here}, it seems that the structure over
$\Q$ is always one of the following: $\set{12}$, $\set{4,\, 8}$,
$\set{2,\, 2\, 8}$. And all three seems to occur for infinitely many
$c \in \Q$.

My computational results for $\Phi_{5, c}(z)$ is
\href{http://goo.gl/dqjs8q}{here}. I enumerated all rationals $p/q$
with $\max{|p|, |q|} \le 1000$, but didn't get any nontrivial points
in the extensions. The sampling is too small to be faithful, but at
least the structures we have are $\set{5,\, 10,\, 15}$, $\set{10,\,
  20}$, $\set{5,\, 25}$.

These results seem to suggest that

\begin{conjecture}
  Let $N \ge 3$, $c \in \Q$. Let $\psi$ be an irreducible factor of
  $\Phi_{N, c}(z)$. Then either $\deg \psi = 1$ or $\gcd(\deg \psi, N)
  > 1$.
\end{conjecture}

\begin{remark}
  If for $N \ge 3$, $\deg \psi = 1$ never occurs, then the conjecture
  of no cycles of size $>3$ automatically follows.
\end{remark}

\end{document}
