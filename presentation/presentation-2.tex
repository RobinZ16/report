\documentclass{amsart}


%%%%%%%%%%%%%%%%%%%%%%%%%%%%%% packages %%%%%%%%%%%%%%%%%%%%%%%%%%%%%%

% Here are some common packages used in almost all documents.
\usepackage{amssymb}
\usepackage{graphicx}
\usepackage{color}
\usepackage{hyperref}

\usepackage{enumerate} % customizable styles for counter printing


%%%%%%%%%%%%%%%%%%%%%%% amsthm theorem styles %%%%%%%%%%%%%%%%%%%%%%%%

% Current styles are based on suggestions by the amsthm package
% authors.
\theoremstyle{plain}
\newtheorem{theorem}{Theorem}[section]
\newtheorem{lemma}[theorem]{Lemma}
\newtheorem{proposition}[theorem]{Proposition}
\newtheorem{corollary}[theorem]{Corollary}
\newtheorem{claim}[theorem]{Claim}

\theoremstyle{definition}
\newtheorem{definition}[theorem]{Definition}
\newtheorem{example}[theorem]{Example}
\newtheorem{conjecture}[theorem]{Conjecture}

\theoremstyle{remark}
\newtheorem{remark}[theorem]{Remark}
\newtheorem{note}[theorem]{Note}
\newtheorem{case}{Case}


%%%%%%%%%%%%%%%%%%%%%%%%% custom definitions %%%%%%%%%%%%%%%%%%%%%%%%%

% Since we are collaborating, we need to be careful with personal
% shortcuts to avoid clash.
\newcommand{\C}{\mathbb{C}}
\newcommand{\F}{\mathbb{F}}
\newcommand{\N}{\mathbb{N}}
\renewcommand{\O}{\mathcal{O}}
\renewcommand{\P}{\mathbb{P}}
\newcommand{\Q}{\mathbb{Q}}
\newcommand{\U}{\mathcal{U}}
\newcommand{\Z}{\mathbb{Z}}

\renewcommand{\l}{\lambda}

\newcommand{\es}{\emptyset}
\newcommand{\ol}{\overline}

\newcommand{\gal}{\mathrm{Gal}}
\newcommand{\preper}{\mathrm{PrePer}}
\newcommand{\res}{\mathrm{Res}}

\newcommand{\set}[1]{\left\{#1\right\}}
\newcommand{\tup}[1]{\left<#1\right>}


\title{SURIM Number Theory Group\\Notes for Presentation II}

\author{Zhiming Wang}
\author{Robin Zhang}

\begin{document}

\maketitle

\section{Recap}

I'll assume that everyone has forgotten about what we've been doing,
so I'll quickly recap the specific problem we are working with.

Let $f(z) = z^2 + c$, where $c$ is a rational number. We want to study
the finite orbits of size $N$ of the function $f$, i.e., exact
$N$-cycles when we iterate $f$. Points of period $N$ are characterized
by the polynomial equation $f^N(z) - z = 0$ (which is in fact a
polynomial equation in both $z$ and $c$, so it defines a plane curve
in the $(z, c)$-plane); and for exact $N$-cycles, we have a slightly
better polynomial equation. But the general case is not the concern of
this talk. In this talk we will mainly focus on the case of $N = 5$
(which is neither completely understood nor hopelessly difficult), and
in this case, the $(z, c)$ polynomial is $\frac{f^5(z) - z}{f(z) -
  z}$, which has degree 30 in $z$ and genus 14 (we will define the
genus later). Little is known about a genus-14 curve, but by taking a
quotient of this curve with respect to an automorphism we can reduce
it to a more manageable genus-2 curve, that is, a hyperelliptic
curve. The genus-2 curve has a smooth model given by
\[
y^2 = f(x) = x^6 + 8x^5 + 22x^4 + 22x^3 + 5x^2 + 6x + 1.
\]
where we basically performed a series of coordinate changes. If we
trace back through the coordinate changes, we have an expression of
the original $c$ in terms of the $x, y$ here:
\[
c = \frac{P_0(x) + P_1(x) y}{h(x)},
\]
where $P_0$, $P_1$ and $h(x)$ are single-variate polynomials in $x$
with degrees 6, 3, and 4. The detailed shapes of these polynomials are
not relevant, and you don't want to see them (trust me).

It was shown by FPS (Flynn, Poonen, and Schaffer) in 1996 that there
are only six rational points on the projective curve $y^2 = f(x)$, and
none of them corresponds to affine rational points on the original
$(z, c)$-curve. This way they established

\begin{theorem}[FPS, 1996]
  There is no rational $c$ such that $f(z) = z^2 + c$ has a rational
  periodic point of exact period 5.
\end{theorem}

We want to generalize this result by extending the field from $\Q$ to
quadratic number fields $\Q(\sqrt{d})$. That is, we want to fully
characterize periodic points of $f$ of exact period 5 in any quadratic
number field, with $c$ still rational.

\begin{remark}
  The problem becomes somewhat less interesting when we also let $c$
  to be quadratic. We have some observations in that direction but we
  won't talk about them today.
\end{remark}

Before we proceed, we need to introduce some concepts and machinery.

\section{Preliminaries}

\begin{definition}[Genus of a smooth algebraic curve]
  To each smooth algebraic curve, we can associate a nonnegative
  integer $g$, which is basically the genus of the complex manifold
  formed by the complex solutions. (The genus of a manifold is the
  number of handles, or holes.)
\end{definition}

\begin{remark}
  I'm cheating a little bit here, but the exact definition is not
  important.  \textbf{It is enough to know that given a curve defined
    by a polynomial $P(x, y)$ in two variables, we can calculate $g$
    computationally.}

  The reason that we are interested in the genus is because it gives
  us information about the number of rational points on a curve, due
  to the following theorem.
\end{remark}

\begin{theorem}[Faltings, 1983]
Let $C$ be a non-singular algebraic curve of genus $g \ge 2$ over
$\Q$. Then the number of rational points on $C$ is finite.
\end{theorem}

\begin{remark}
  Faltings' theorem only gives an ineffective bound on the number of
  rational points, as it doesn't offer an explicit bound, nor a way to
  find an explicit bound.

  There are some methods to bound the number of rational points, and
  they are behind most of the currently available results in our
  field. We will mention these methods later.
\end{remark}

\begin{definition}[Resultant of polynomials]
  Given two univariate polynomials $P$ and $Q$ over a field $k$ with
  leading coefficients $p$ and $q$, the resultant of $P$ and $Q$ is
  defined as
  \[
  \res(P, Q) = p^{\deg Q} q^{\deg P}
  \prod_{\substack{\text{$x$ root of $P$}\\
                   \text{$y$ root of $Q$}}} (x - y),
  \]
  with multiplicities counted.
\end{definition}

\begin{remark}
  Similar to the discriminant of a polynomial, \textbf{the resultant
    of two polynomials can be written as a polynomial in the
    coefficients of $P$ and $Q$, and computed efficiently.}

  From its definition it is obvious that $P$ and $Q$ have a common
  factor if and only if $\res(P, Q) = 0$.
\end{remark}

For a quick recap, we defined genus and resultant, both of which
are computable (actually by software). Genus gives information about
the number of rational points, and resultant gives information about
comman roots.

\section{$x$, $y$, and $c$}

After introducing the preliminaries, let us approach the original
problem. Let me write down what we have again:
\[
\begin{gathered}
  y^2 = f(x),\\
  c = \frac{P_0(x) + P_1(x) y}{h(x)},
\end{gathered}
\]
where $f, h, P_0, P_1$ are polynomials in $x$ with rational
coefficients. $c$ is rational. It turns out that quadratic periodic
points correspond to quadratic $(x, y)$ on the curve $y^2 = f(x)$, so
we would like $x$ to solve some quadratic polynomial equation
\[
x^2 + ax + b = 0,
\]
with coefficients $a, b \in \Q$. The reason we are trying to
characterize one variable $x$ with two rational variables $a$ and $b$,
which seems to be a dumb decision upon first sight, is due to the fact
that, as you see before, a lot of machinery has been developed to
charactize rational points on algebraic curves, and more generally,
algebraic varieties. So we really want rational points rather than
quadratic ones.

Now one obvious thing to do is to eliminate $y$, which is not very
hard:
\[
(h(x) c - P_0(x))^2 = P_1^2(x) y^2 = P_1^2(x) f(x),
\]
so we have a polynomial equation $P$ in two variables $c$ and
$x$. Recall that $x^2 + ax + b = 0$, so if we regard $P$ as a
polynomial in $x$ with $c$ in coefficients, and do long division with
respect to $x^2 + ax + b$, we end up with a remainder of degree one in
$x$, which must be zero:
\[
\l_1(a, b, c) x + \l_0(a, b, c) = 0.
\]
Now we observe that $a, b, c$ are rational, and the coefficients of
$\l_1$ and $\l_0$ are rational, so both $\l_1$ and $\l_0$ are
rational. And $x$ is quadratic. So unless $x$ is also rational, in
which case we already completely understand the six rational points of
$y^2 = f(x)$, we must have
\[
\l_1(a, b, c) = \l_0(a, b, c) = 0.
\]

Now the resultant trick kicks in. Recall that two single-variate
polynomials have a common root if and only if their resultant is
zero. Here we can regard $\l_1$ and $\l_0$ as single-variate
polynomials in any of $a$, $b$, and $c$, and calculate the
resultant. For instance, if we regard them as polynomials in $a$, then
the resultant is a polynomial in $b$ and $c$, and it must be zero in
order for $\l_1$ and $\l_0$ to have a common root $a$.

We calculated all three resultants (\textit{in fact, the calculation
  of resultant is somewhat more subtle, depending on the vanishing of
  leading coefficients; but it seems that we don't have enough time to
  explain this}), and found out that the resultant w.r.t.\ $c$, which
is a polynomial in $a$ and $b$, is nicer than the others. By nicer I
mean it has only degree 8 in $a$ and degree 9 in $b$, and its
coefficients have only around ten digits. Very nice.

Therefore, we finally reduced the task of finding quadratic points on
the curve $y^2 = f(x)$ (and simultaneously guarantee that $c$ is
rational), to finding rational points on a curve $Q(a, b) = 0$.

We calculated the genus of this curve. The genus turns out to be
11. Therefore, by Faltings' theorem, there are only finitely many
rational points on it. For each pair of rational $(a, b)$, $c$ is a
solution to the system $\l_0 = \l_1 = 0$, which of course has only
finitely many solutions. In fact there are some other edge cases, but
they are easy to take care of. So we have a theorem:

\begin{theorem}
  There are at most finitely many rational $c$ such that $f(z) = z^2 +
  c$ has a quadratic periodic point of exact period 5.
\end{theorem}

We still want to have an explicit upper bound on the number of such
$c$'s, or if we get lucky, prove that the bound is actually zero,
which is our conjecture. For this purpose we need to study the
rational points on the curve $Q(a, b)$. There are some methods to
figure out explicit upper bounds on the number of rational points,
among them Chabauty and Coleman's method, which might be computational
very hard, unfortunately seldom give sharp bounds. We are looking into
the methods right now.

\section{Other Results}

\begin{theorem} [Erkama 2006]
For $f(z) = z^2 + c$, $c$ and the points of period 4 can be parametrized
over $\C$ by the following:
\[
\begin{gathered}
	c = \frac{1-4t^3-t^6}{4t^2(t^2-1)}, \\
	x_1 = \frac{t^4 - t^2 + \sqrt{(t^4-1)(t^2+2t-1)}}{2t(t^2-1)}, x_2 = \frac{1 - t^2 + t\sqrt{(t^4-1)(t^2+2t-1)}}{2t(t^2-1)}, \\
	x_3 = \frac{t^4 - t^2 - \sqrt{(t^4-1)(t^2+2t-1)}}{2t(t^2-1)}, x_4 = \frac{1 - t^2 - t\sqrt{(t^4-1)(t^2+2t-1)}}{2t(t^2-1)}
\end{gathered}
\]
where $t = x_1 + x_3$.
\end{theorem}

Panraksa showed that all the points of period 4 over a quadratic field
for $c \in \Q$ are given by ranging $t \in \Q$ by showing that such
cycles must have $x_1$ and $x_3$ as Galois conjugates.
Thus, it directly follows that:
\begin{theorem}
There are infinitely many points in quadratic fields with exact period 4 for
$f(z) = z^2 + c$ with $c \in \Q$.
\end{theorem}

Plugging in the $c$ from Morton's parametrization and factoring $\Phi_4(z,c)$
with respect to $z$ yields:
\begin{lemma}
For $c$ such that $f$ has 4-cycles over quadratic fields, the degree 12
polynomial $\Phi_4(z,c)$ must factor as $\{2,2,8\}$ (list of degrees of
factors, with the 8th degree factor not necessarily irreducible).
\end{lemma}

Using Panraksa's theorem that $\Phi_4(z,c)$ cannot factor as $\{2,2,2,2,4\}$,
we obtain the following result:
\begin{theorem}
For each $c \in \Q$, there is at most one 4-cycle over quadratic fields.
\end{theorem}

We tried to work out similar results for the $N = 6$ case and have the
following general lemma which may be useful for showing Galois conjugacy of
elements in even-parity cycles.
\begin{lemma}
	Let $\{x_1, \ldots, x_{N}\}$ be a $N$-cycle with N even,
	defined over quadratic field K as before. If
	$x_{\frac{N}{2}+1} \neq \overline{x_1}$ and if there are no
	rational points of exact period $N$, then $\{x_1, \ldots, x_{N}\}
	\cap \{\overline{x_1}, \ldots, \overline{x_{N}}\} = \emptyset$.
\end{lemma}

We think that there should be a more general situation:
\begin{conjecture}
	Let $\{x_1, \ldots, x_{N}\}$ be a $N$-cycle defined over number field K,
	with N divisible by	$d = [K:\Q ]$. If $x_{\frac{N}{d}+1} \neq \sigma(x_1)$
	for all $\sigma \in \gal(K/\Q)$ and if there are no rational points of
	exact period $N$, then $\{x_1, \ldots, x_{N}\} \cap \{\tau(x_1),
	\ldots, \tau(x_{N})\} = \emptyset$, $\forall \tau \in \gal(K/\Q)$.
\end{conjecture}

\end{document}
