\documentclass{amsart}


%%%%%%%%%%%%%%%%%%%%%%%%%%%%%% packages %%%%%%%%%%%%%%%%%%%%%%%%%%%%%%

% Here are some common packages used in almost all documents.
\usepackage{amssymb}
\usepackage{graphicx}
\usepackage{color}
\usepackage{hyperref}

\usepackage{enumerate} % customizable styles for counter printing


%%%%%%%%%%%%%%%%%%%%%%% amsthm theorem styles %%%%%%%%%%%%%%%%%%%%%%%%

% Current styles are based on suggestions by the amsthm package
% authors.
\theoremstyle{plain}
\newtheorem{theorem}{Theorem}[section]
\newtheorem{lemma}[theorem]{Lemma}
\newtheorem{proposition}[theorem]{Proposition}
\newtheorem{corollary}[theorem]{Corollary}
\newtheorem{claim}[theorem]{Claim}

\theoremstyle{definition}
\newtheorem{definition}[theorem]{Definition}
\newtheorem{example}[theorem]{Example}
\newtheorem{conjecture}[theorem]{Conjecture}

\theoremstyle{remark}
\newtheorem{remark}[theorem]{Remark}
\newtheorem{note}[theorem]{Note}
\newtheorem{case}{Case}


%%%%%%%%%%%%%%%%%%%%%%%%% custom definitions %%%%%%%%%%%%%%%%%%%%%%%%%

% Since we are collaborating, we need to be careful with personal
% shortcuts to avoid clash.
\newcommand{\C}{\mathbb{C}}
\newcommand{\F}{\mathbb{F}}
\newcommand{\N}{\mathbb{N}}
\renewcommand{\O}{\mathcal{O}}
\renewcommand{\P}{\mathbb{P}}
\newcommand{\Q}{\mathbb{Q}}
\newcommand{\U}{\mathcal{U}}
\newcommand{\Z}{\mathbb{Z}}

\renewcommand{\l}{\lambda}

\newcommand{\es}{\emptyset}
\newcommand{\ol}{\overline}

\newcommand{\gal}{\mathrm{Gal}}
\newcommand{\preper}{\mathrm{PrePer}}
\newcommand{\res}{\mathrm{Res}}

\newcommand{\set}[1]{\left\{#1\right\}}
\newcommand{\tup}[1]{\left<#1\right>}


\renewcommand{\b}{\beta}
\newcommand{\N}{\mathbb{N}}

\begin{document}

What we are working on is \emph{arithmetic dynamics}, which is
basically the study of iteration of polynomial or rational functions
over various kinds of number rings or fields, say, the $p$-adic
integers (you don't need to understand that). If this isn't clear
enough, you can think of the famous Mandelbrot set, which is a
dynamical system over the complex plane --- iterating $z^2 + c$. In
arithmetic dynamics we are of course interested in arithmetic
properties, so we need to consider fields or rings like $\Q$, $\Z$,
$\Q_p$, $\Z_p$, etc.

We currently have two families of problems at hand. The statement of
one family is very simple, so I'll talk about it first. The second
family is a little bit tricky, so I'll see if I can finish that.

So the first family is dynamics over $\Z$.

Consider a polynomial $\phi(x) \in \Z[x]$ with integer
coefficients, and let $\b \in \Z$ be a non-periodic point, namely,
the orbit of $\b$ under $\phi$ is infinite: $|O_\phi(\b)| =
\infty$. (You all know what the orbit is, right? Basically all the
points you can get from $\b$ through applying the map.)

Here are the questions we can ask:

\begin{enumerate}
\item \textbf{Does this orbit contain infinitely many primes?}

  You know, the distribution of primes are rather random, and
  $\phi^j(\b)$ blows up very quickly, so this is not clear.

\item \textbf{What is the density of primes that divides some element of the
  orbit?}

  \[
  \set{p \in \mathrm{primes} : \exists j \in \N \text{ s.t. } p \mid
    \phi^j(\b)}.
  \]

  You know, even if there are infinitely many prime factors, we might
  skip a lot, so the density is not clear.
\end{enumerate}

Another family of questions are dynamics over finite fields, namely,
$\F_{p^n}$. What's good about finite fields is that, according to the
pigeon hole principle, every point is at least preperiodic under a
map, namely, every point eventually fall into some loop. But whether
a point is periodic or not, namely, if it's actually inside a loop, is
unclear.

Consider a rational function $\phi(x) \in F_p(x)$ whose numerator and
denominator are of the same degree. We may ask, what is the
proportion of periodic points in $\F_{p^n}$? Namely, what's this limit

\[
\lim_{n \to \infty} \frac{|\mathrm{Per}(\phi, \P^1(\F_{p^n}))|}{p^n}.
\]

This expression requires some explanation. First, the $\P^1$ means the
projective line. You can think of it as the original field plus a
point at the infinity. Why do we need $\P^1(\F_{p^n})$ instead of just
$\F_{p^n}$? That's because $\phi$ is a rational function, so when we
apply it we might get a zero on the denominator. Then we need
$\infty$. For instance, say $\phi$ is $\dfrac{x+1}{x-1}$, then when we
apply it to $1$ we get $\infty$, and when we apply it $\infty$, we get
\[
\frac{\infty + 1}{\infty - 1} = 1,
\]
just like when you take limits. And here's actually why I required $\phi$ to
have numerator and denominator of same degree in the first
place. Everything here can be made rigirous, but I'll just skip that.

Another thing that requires explanation is we defined $\phi$ over
$\F_p$ but we are considering periodic points in $\F_{p^n}$. So why is
this legit?  Notice that $\F_p$ is a subfield of $\F_{p^n}$, so we can
apply $\phi$ to an element of $\F_{p^n}$ (in fact, I mean
$\P^1(\F_{p^n})$), and get an image in the same field. So there's no
problem here.

\end{document}
