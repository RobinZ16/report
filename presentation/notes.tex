\documentclass{amsart}


%%%%%%%%%%%%%%%%%%%%%%%%%%%%%% packages %%%%%%%%%%%%%%%%%%%%%%%%%%%%%%

% Here are some common packages used in almost all documents.
\usepackage{amssymb}
\usepackage{graphicx}
\usepackage{color}
\usepackage{hyperref}

\usepackage{enumerate} % customizable styles for counter printing


%%%%%%%%%%%%%%%%%%%%%%% amsthm theorem styles %%%%%%%%%%%%%%%%%%%%%%%%

% Current styles are based on suggestions by the amsthm package
% authors.
\theoremstyle{plain}
\newtheorem{theorem}{Theorem}[section]
\newtheorem{lemma}[theorem]{Lemma}
\newtheorem{proposition}[theorem]{Proposition}
\newtheorem{corollary}[theorem]{Corollary}
\newtheorem{claim}[theorem]{Claim}

\theoremstyle{definition}
\newtheorem{definition}[theorem]{Definition}
\newtheorem{example}[theorem]{Example}
\newtheorem{conjecture}[theorem]{Conjecture}

\theoremstyle{remark}
\newtheorem{remark}[theorem]{Remark}
\newtheorem{note}[theorem]{Note}
\newtheorem{case}{Case}


%%%%%%%%%%%%%%%%%%%%%%%%% custom definitions %%%%%%%%%%%%%%%%%%%%%%%%%

% Since we are collaborating, we need to be careful with personal
% shortcuts to avoid clash.
\newcommand{\C}{\mathbb{C}}
\newcommand{\F}{\mathbb{F}}
\newcommand{\N}{\mathbb{N}}
\renewcommand{\O}{\mathcal{O}}
\renewcommand{\P}{\mathbb{P}}
\newcommand{\Q}{\mathbb{Q}}
\newcommand{\U}{\mathcal{U}}
\newcommand{\Z}{\mathbb{Z}}

\renewcommand{\l}{\lambda}

\newcommand{\es}{\emptyset}
\newcommand{\ol}{\overline}

\newcommand{\gal}{\mathrm{Gal}}
\newcommand{\preper}{\mathrm{PrePer}}
\newcommand{\res}{\mathrm{Res}}

\newcommand{\set}[1]{\left\{#1\right\}}
\newcommand{\tup}[1]{\left<#1\right>}

\newcommand{\C}{\mathbb{C}}
\newcommand{\F}{\mathbb{F}}
\newcommand{\N}{\mathbb{N}}
\renewcommand{\O}{\mathcal{O}}
\renewcommand{\P}{\mathbb{P}}
\newcommand{\Q}{\mathbb{Q}}
\newcommand{\U}{\mathcal{U}}
\newcommand{\Z}{\mathbb{Z}}

\renewcommand{\l}{\lambda}
\newcommand{\s}{\sigma}
\renewcommand{\t}{\tau}

\newcommand{\es}{\emptyset}
\newcommand{\ol}{\overline}

\newcommand{\gal}{\mathrm{Gal}}
\newcommand{\preper}{\mathrm{PrePer}}
\newcommand{\res}{\mathrm{Res}}

\newcommand{\set}[1]{\left\{#1\right\}}
\newcommand{\tup}[1]{\left<#1\right>}


\begin{document}

\section{Disclaimer}
\label{sec:disc}

We are the number theory group working on \emph{arithmetic dynamics}.
For today's presentation, we will not include a lot of the material or
background from our previous talks. We will focus on our main results
and the new directions that we have come upon in the last few weeks.

\section{Define the problem}
\label{sec:defprob}

Our goal this summer was to study the points in quadratic number
fields (fields of the form $\Q(\sqrt{d})$ with $d$ a squarefree
integer) with exact period $N$ on quadratic polynomials. We call each
orbit of periodic points an ``$N$-cycle'', and we want to see how many
$N$-cycles can arise over quadratic number fields for each $N$.

Our study is simplified by the fact that quadratic polynomials $a_0z
^2 + a_1z + a_2$ can be reduced to the form $z^2 + c$ by linear
conjugation. Linear conjugation preserves the length of orbits, so it
suffices to only consider quadratic polynomials of the form $\phi
_c(z) = z^2 + c$. For our project, we will only consider the case when
$c \in \Q$.

\section{General model}
\label{sec:model}

In this section, we set up models to study periodic points of
$\phi_c(z) = z^2 + c$ of exact period $N$. We first set up a general
model, and then specialize to $N = 4$, $5$, and $6$. These models have
already been set up in the literature. We also discuss the results
associated to these models in the literature.

\section{The general model}
\label{sec:model-general}

Let $\phi_c(z) = z^2 + c$. Let $\phi_c^N(z)$ denote the $N$-th
iteration of $f$. Then all periodic points of periodic $N$ (including
points of exact period $d$ such that $d | N$) satisfy the polynomial
equation
\[
\phi_c^N(z) - z = 0.
\]
By the M\"obius inversion formula, we have
\[
\phi_c^N(z) - z = \prod_{d|N} \Phi_d(z, c),
\]
where
\[
\Phi_d(z, c) = \prod_{m|d}(\phi_c^m(z) - z)^{\mu(d/m)}.
\]
With a little bit of effort we can show that $\Phi_d(z, c)$ resides in
$\Z[z, c]$, and that all periodic points of exact period $N$ satisfy
the polynomial equation $\Phi_N(z, c) = 0$.

$\Phi_N(z, c)$ defines an algebraic curve in the $(z,
c)$-plane. Denote the normalization of this curve by $C_1(N)$. We take
the quotient curve of $C_1(N)$ by associating points within an orbit,
and denote the normalization of the quotient curve by $C_0(N)$.

Next we provide models of $C_1(N)$ or $C_0(N)$ for $N = 4$, $5$, or
$6$, and briefly discuss what is already known about these models.

\section{Previous Results}
\label{sec:prevres}

\subsection{N = 4}
\begin{theorem}[Morton]
  There are no finite rational solutions $(z, c)$ of the equation
  $\Phi_4(z, c) = 0$. In other words, there are no quadratic
  polynomials defined over $\Q$ with a rational 4-cycle.
\end{theorem}

\subsection{N = 5}
\begin{theorem}[Flynn, Poonen, Schaefer]
  There are no quadratic polynomials defined over $\Q$ with a rational
  5-cycle.
\end{theorem}

\subsection{N = 6}
\begin{theorem}[Stoll]
  Let $J$ be the Jacobian of $C_0(6)$. If the $L$-series $L(J,s)$
  extends to an entire function and satisfies the standard functional
  equation, and if the weak Birch and Swinnerton-Dyer conjecture is
  valid for $J$, then there are no quadratic polynomials defined over
  $\Q$ with a rational 6-cycle.
\end{theorem}

\section{Our question}
\label{sec:ourq}

We want to extend the previous results by lifting the base field from
the rational field $\Q$ to quadratic extensions of $\Q$.

Let $\U$ be the collection of all rational or quadratic elements in
$\ol{\Q}$,
\[
\U = \set{\alpha \in \ol{\Q}: \text{$a \alpha^2 + b \alpha + c = 0$
    for some $a, b, c \in \Q$}},
\]
or equivalently,
\[
\U = \bigcup_{[K : \Q] = 2}K,
\]
where the union is the union of sets, taken over all quadratic
extensions of $\Q$.

We pose the following question:

\begin{question}
  \label{question}
  How many $c \in \Q$ are there such that $\phi_c(z) = z^2 + c$ has
  periodic points in $\U$ of exact period $N$?
\end{question}

\begin{remark}
  In the above question, we only asked for the number of $c$. However,
  once $c$ is known, we can immediately determine the number of
  periodic points of exact period $N$ from the equation $\Phi_N(z, c)
  = 0$; and the number of such points is bounded above by
  \[
  \deg_z \Phi_N(z, c) = \sum_{d|N}2^d \mu(N/d),
  \]
  which is a constant only depending on $N$.
\end{remark}

It turns out that we can fully answer Questions~\ref{question} for $N
= 4$, and partially answer it for $N = 5$.

For $N = 4$, the answer is infinite, since an explicit parametrization
of $c$ is available to us.

For $N = 5$, the answer is finite, but right now we do not have a
concrete bound. We conjecture that the answer is zero.

We also have some general results from extending previous results on
$N = 4$. More to follow later...

\section{General Results}
\label{sec:genres}

While studying the quadratic N = 4 case, we tried to mimic results to
the general case. In particular, the following result by Panraksa was
crucial to proving his main result.

\begin{lemma} [Panraksa]
  Let $c \in \Q$, and let $\set{z_1, z_2, z_3, z_4} \subset K$ be a
  4-cycle of $\phi_c(z) = z^2 + c$, where $K$ is a quadratic extension
  to $\Q$. Then exactly one of the following holds:
  \begin{enumerate}[(i)]
  \item $z_3 = \ol{z_1}$, where $\ol{z_1}$ denotes the Galois
    conjugate of $z_1$ in $K$;

  \item $\{z_1, z_2, z_3, z_4\} \cap \{\ol{z_1}, \ol{z_2}, \ol{z_3},
    \ol{z_4}\} =\es.$
  \end{enumerate}
\end{lemma}

Panraksa then proves that the second case $\{z_1, z_2, z_3, z_4\} \cap
\{\ol{z_1}, \ol{z_2}, \ol{z_3}, \ol{z_4}\} = \es$ is impossible, from
which his main result immediately follows. We managed to prove a
much more general version of his lemma.

\newcommand{\nd}{\frac{N}{(N, d)}}

\begin{theorem}
  Let $N \in \N^*$, $c \in \Q$, $K$ be a Galois extension of $\Q$ with
  degree $d = [K : \Q]$, and $\set{z_0, \dots, z_{N-1}} \subset K$ be
  an exact $N$-cycle of $\phi_c(z) = z^2 + c$. Then exactly one of the
  following holds:
  \begin{enumerate}[(i)]
  \item $z_{i\nd} = \t(z_0)$ for some $0 \le i \le (N, d)-1$ and some
    nontrivial $\t \in \gal(K/\Q)$;

  \item $\set{z_0, \dots, z_{N-1}} \cap \set{\t(z_0), \dots,
      \t(z_{N-1})} = \es$ for all nontrivial $\t \in \gal(K/\Q)$.
  \end{enumerate}
\end{theorem}

\begin{proof}
  First note that (i) and (ii) cannot be simultaneously true. In fact,
  if (i) is true, i.e., $z_{i\nd} = \t(z_0)$ for some $i$ and
  nontrivial $\t \in \gal(K/\Q)$, then $z_{i\nd} \in \set{z_0, \dots,
    z_{N-1}} \cap \set{\t(z_0), \dots, \t(z_{N-1})}$, so (ii) is
  false.

  If (ii) is true, then we are done. Otherwise, (ii) is false, so we
  have some nontrivial $\t \in \gal(K/\Q)$ such that $z_k = \t(z_j)$
  for some $j$ and $k$ satisfying $0 \le j, k \le N-1$. Note that $\t$
  commutes with $\phi_c$ (as $\phi_c$ is defined over $\Q$ and $\t$ is
  a field automorphism fixing the base field $\Q$), so we may assume
  $j = 0$. Assuming $j = 0$, we have $\t(z_0) = z_k$; this,
  together with the fact that $\t$ commutes with $\phi_c$, implies
  that $\t \equiv \phi_c^k$ on the whole cycle $\set{z_0, \dots,
    z_{N-1}}$.

  Now recall that $K$ is Galois, so the order of the Galois group
  $\gal(K/\Q)$ is exactly $[K : \Q] = d$. Therefore, by Lagrange's
  theorem, $\t^d = Id$, and hence
  \[
  z_0 = \t^d(z_0) = (\phi_c^k)^d(z_0) = z_{kd}.
  \]
  Let $r$ be the remainder of $kd$ modulo $N$. If $r$ is nonzero, then
  $\set{z_0, \dots, z_{r-1}}$ forms a cycle of $\phi_c$ with length $r
  < N$, violating the premise that $\set{z_0, \dots, z_{N-1}}$ is an
  exact $N$-cycle. Therefore, $r = 0$, i.e., $N \mid kd$, and hence
  $\nd \mid k$. Since $0 \le k \le N - 1$, we have $k = i \cdot \nd$
  for some $0 \le i \le (N, d) - 1$. Consequently, $\t(z_0) = z_k =
  z_{i\nd}$, (i) is true, and we are done.
\end{proof}

In fact, we believe that the second case never occurs in
general. Although the supply of examples is limited, all of the
examples that we currently know satisfy the first case (all 4-cycles,
the 5-cycles described by Flynn, Poonen, and Schaefer, and the 6-
cycles described by Stoll). We state our conjecture as follows.

\begin{conjecture}
  \label{cj:galois-conjugate-re}
  Let $N \in \N^*$, $c \in \Q$, $K$ be a Galois extension of $\Q$ with
  degree $d = [K : \Q]$, and $\set{z_0, \dots, z_{N-1}} \subset K$ be
  an exact $N$-cycle of $\phi_c(z) = z^2 + c$. Then $z_{i\nd} =
  \t(z_0)$ for some $0 \le i \le (N, d)-1$ and some nontrivial $\t \in
  \gal(K/\Q)$.
\end{conjecture}

If the conjecture is true, then in particular we have some interesting
consequences for $d = 2$, i.e., $K$ quadratic, which is the case we
are most interested in. In this case, $K$ is automatically Galois, so
the conjecture can be applied unconditionally.

If $N$ is odd, then $(N, d) = 1$, so we have $z_0 = \ol{z_0}$, i.e.,
$z_0$ is rational. Consequently the entire cycle is defined over $\Q$,
and we can reduce the problem of finding quadratic $N$-cycles to
purely finding rational $N$-cycles ($C_1(N)$). In particular, for $N = 5$,
we already know there are no rational 5-cycles, so
the conjecture that says no quadratic 5-cycles
exist for $c \in \Q$, follows automatically. We record this corollary
as follows.

\begin{corollary}
  If Conjecture~\ref{cj:galois-conjugate-re} holds, then there are no $c
  \in \Q$ such that $\phi_c(z) = z^2 + c$ has periodic points in $\U$
  of exact period 5.
\end{corollary}

If $N$ is even, then $(N, d) = 2$, and we have either $z_0 =
\ol{z_0}$, in which case the cycle is defined over $\Q$; or $z_0 =
\ol{z_{\frac{N}{2}}}$, in which case $z_0 + z_{\frac{N}{2}}$ is
rational, and consequently the trace $z_0 + \phi_c(z_0) + \cdots +
\phi_c^{N-1}(z_0)$ is rational. In either case, the point on $C_0(N)$
that corresponds to $c$ and the cycle is a rational point, so finding
quadratic periodic points of exact period $N$ is reduced to studying
rational points on $C_0(N)$.

In particular, for $N = 6$, since the rational points on $C_0(6)$ are
already understood due to Stoll's work (conditional
on the weak Birch and Swinnerton-Dyer conjecture on the Jacobian of
$C_0(6)$), then we can easily show by exhaustion that the only
quadratic 6-cycle is defined over $\Q(\sqrt{33})$, with $c =
-\frac{71}{48}$ and
\begin{equation}
  \label{eq:6-cycle}
  z_0 = -1 + \frac{\sqrt{33}}{12},\,
  z_1 = -\frac{1}{4} - \frac{\sqrt{33}}{6},\,
  z_2 = -\frac{1}{2} + \frac{\sqrt{33}}{12},\,
  z_3 = \ol{z_0},\,
  z_4 = \ol{z_1},\,
  z_5 = \ol{z_2}.
\end{equation}
This establishes yet another corollary to
Conjecture~\ref{cj:galois-conjugate-re}.

\begin{corollary}
  \label{cor:6-cycle}
  Let $J$ be the Jacobian of $C_0(6)$. If the $L$-series $L(J, s)$
  extends to an entire function and satisfies the standard functional
  equation; the weak Birch and Swinnerton-Dyer conjecture is valid for
  $J$; and Conjecture~\ref{cj:galois-conjugate-re} holds, then the only
  $c \in \Q$ such that $\phi_c(z) = z^2 + c$ has periodic points of
  exact period 6 in $\U$ is $-\frac{71}{48}$, and the corresponding
  periodic points are $z_0, \dots, z_5$ as defined in
  (\ref{eq:6-cycle}).
\end{corollary}

At the time of writing, though, little is known about how to approach
the conjecture. The known proof in the $N = 4$ case relies on a
special structure of $\Phi_4(z, c)$, and cannot be generalized even to
$N = 6$. It might be useful to first check the conjecture for finite
fields; if the conjecture is true for infinitely many different
characteristics, some Chebotarev density-type argument might yield the
full conjecture over number fields (which have characteristic zero).


\end{document}
