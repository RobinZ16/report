\documentclass{amsart}


%%%%%%%%%%%%%%%%%%%%%%%%%%%%%% packages %%%%%%%%%%%%%%%%%%%%%%%%%%%%%%

% Here are some common packages used in almost all documents.
\usepackage{amssymb}
\usepackage{graphicx}
\usepackage{color}
\usepackage{hyperref}


%%%%%%%%%%%%%%%%%%%%%%% amsthm theorem styles %%%%%%%%%%%%%%%%%%%%%%%%

% Current styles are based on suggestions by the amsthm package
% authors.
\theoremstyle{plain}
\newtheorem{theorem}{Theorem}[section]
\newtheorem{lemma}[theorem]{Lemma}
\newtheorem{proposition}[theorem]{Proposition}
\newtheorem{corollary}[theorem]{Corollary}
\newtheorem{claim}[theorem]{Claim}

\theoremstyle{definition}
\newtheorem{definition}[theorem]{Definition}
\newtheorem{example}[theorem]{Example}
\newtheorem{conjecture}[theorem]{Conjecture}

\theoremstyle{remark}
\newtheorem{remark}[theorem]{Remark}
\newtheorem{note}[theorem]{Note}
\newtheorem{case}{Case}


%%%%%%%%%%%%%%%%%%%%%%%%% custom definitions %%%%%%%%%%%%%%%%%%%%%%%%%

% Since we are collaborating, we need to be careful with personal
% shortcuts to avoid clash.
\newcommand{\C}{\mathbb{C}}
\newcommand{\F}{\mathbb{F}}
\renewcommand{\P}{\mathbb{P}}
\newcommand{\Q}{\mathbb{Q}}
\newcommand{\Z}{\mathbb{Z}}

\newcommand{\preper}{\mathrm{PrePer}}
\newcommand{\gal}{\mathrm{Gal}}

\newcommand{\tup}[1]{\left<#1\right>}
\newcommand{\set}[1]{\left\{#1\right\}}

\newcommand{\C}{\mathbb{C}}
\newcommand{\F}{\mathbb{F}}
\newcommand{\N}{\mathbb{N}}
\renewcommand{\O}{\mathcal{O}}
\renewcommand{\P}{\mathbb{P}}
\newcommand{\Q}{\mathbb{Q}}
\newcommand{\U}{\mathcal{U}}
\newcommand{\Z}{\mathbb{Z}}

\renewcommand{\l}{\lambda}
\newcommand{\s}{\sigma}
\renewcommand{\t}{\tau}

\newcommand{\es}{\emptyset}
\newcommand{\ol}{\overline}

\newcommand{\gal}{\mathrm{Gal}}
\newcommand{\preper}{\mathrm{PrePer}}
\newcommand{\res}{\mathrm{Res}}

\newcommand{\set}[1]{\left\{#1\right\}}
\newcommand{\tup}[1]{\left<#1\right>}


\begin{document}

\section{Disclaimer}
\label{sec:disc}

We are the number theory group working on \emph{arithmetic dynamics}.
For today's presentation, we will not include a lot of the 
material or background from our previous talks. We will focus on 
our main results and the new directions that we have come upon in 
the last few weeks. 

\section{Define the problem}
\label{sec:defprob}

Our goal this summer was to study the points in quadratic number 
fields (fields of the form $\Q(\sqrt{d})$ with $d$ a squarefree 
integer) with exact period $N$ on quadratic polynomials. We call 
each orbit of periodic points an ``$N$-cycle'', and we want to see 
how many $N$-cycles can arise over quadratic number fields for each 
$N$.

Our study is simplified by the fact that quadratic polynomials $a_0z
^2 + a_1z + a_2$ can be reduced to the form $z^2 + c$ by linear 
conjugation. Linear conjugation preserves the length of orbits, so 
it suffices to only consider quadratic polynomials of the form $\phi
_c(z) = z^2 + c$. For our project, we will only consider the case
when $c \in \Q$.

\section{General model}
\label{sec:model}

In this section, we set up models to study periodic points of
$\phi_c(z) = z^2 + c$ of exact period $N$. We first set up a general
model, and then specialize to $N = 4$, $5$, and $6$. These models have
already been set up in the literature. We also discuss the results
associated to these models in the literature.

\section{The general model}
\label{sec:model-general}

Let $\phi_c(z) = z^2 + c$. Let $\phi_c^N(z)$ denote the $N$-th
iteration of $f$. Then all periodic points of periodic $N$ (including
points of exact period $d$ such that $d | N$) satisfy the polynomial
equation
\[
\phi_c^N(z) - z = 0.
\]
By the M\"obius inversion formula, we have
\[
\phi_c^N(z) - z = \prod_{d|N} \Phi_d(z, c),
\]
where
\[
\Phi_d(z, c) = \prod_{m|d}(\phi_c^m(z) - z)^{\mu(d/m)}.
\]
With a little bit of effort we can show that $\Phi_d(z, c)$ resides in
$\Z[z, c]$, and that all periodic points of exact period $N$ satisfy
the polynomial equation $\Phi_N(z, c) = 0$.

$\Phi_N(z, c)$ defines an algebraic curve in the $(z,
c)$-plane. Denote the normalization of this curve by $C_1(N)$. We 
take the quotient curve of $C_1(N)$ by associating points within an 
orbit, and denote the normalization of the quotient curve by $C_0(N)$.

Next we provide models of $C_1(N)$ or $C_0(N)$ for $N = 4$, $5$, or
$6$, and briefly discuss what is already known about these models.

\section{Previous Results}
\label{sec:prevres}

\subsection{N = 4}
\begin{theorem}[Morton]
  There are no finite rational solutions $(z, c)$ of the equation
  $\Phi_4(z, c) = 0$. In other words, there are no quadratic
  polynomials defined over $\Q$ with a rational 4-cycle.
\end{theorem}

\subsection{N = 5}
\begin{theorem}[Flynn, Poonen, Schaefer]
  There are no quadratic polynomials defined over $\Q$ with a rational
  5-cycle.
\end{theorem}

\subsection{N = 6}
\begin{theorem}[Stoll]
  Let $J$ be the Jacobian of $C_0(6)$. If the $L$-series $L(J,s)$
  extends to an entire function and satisfies the standard functional
  equation, and if the weak Birch and Swinnerton-Dyer conjecture is
  valid for $J$, then there are no quadratic polynomials defined over
  $\Q$ with a rational 6-cycle.
\end{theorem}

\section{Our question}
\label{sec:ourq}

We want to extend the previous results by lifting the base field 
from the rational field $\Q$ to quadratic extensions of $\Q$.

Let $\U$ be the collection of all rational or quadratic elements in
$\ol{\Q}$,
\[
\U = \set{\alpha \in \ol{\Q}: \text{$a \alpha^2 + b \alpha + c = 0$
    for some $a, b, c \in \Q$}},
\]
or equivalently,
\[
\U = \bigcup_{[K : \Q] = 2}K,
\]
where the union is the union of sets, taken over all quadratic
extensions of $\Q$.

We pose the following question:

\begin{question}
  \label{question}
  How many $c \in \Q$ are there such that $\phi_c(z) = z^2 + c$ has
  periodic points in $\U$ of exact period $N$?
\end{question}

\begin{remark}
  In the above question, we only asked for the number of $c$. However,
  once $c$ is known, we can immediately determine the number of
	periodic points of exact period $N$ from the equation $\Phi_N(z, c)
	= 0$; and the number of such points is bounded above by
  \[
  \deg_z \Phi_N(z, c) = \sum_{d|N}2^d \mu(N/d),
  \]
  which is a constant only depending on $N$.
\end{remark}

It turns out that we can fully answer Questions~\ref{question} for
$N = 4$, and partially answer it for $N = 5$.

For $N = 4$, the answer is infinite, since an explicit parametrization
of $c$ is available to us.

For $N = 5$, the answer is finite, but right now we do not have a 
concrete bound. We conjecture that the answer is zero.

We also have some general results from extending previous results on
$N = 4$. More to follow later...
\end{document}
