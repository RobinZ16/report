\documentclass{amsart}


%%%%%%%%%%%%%%%%%%%%%%%%%%%%%% packages %%%%%%%%%%%%%%%%%%%%%%%%%%%%%%

% Here are some common packages used in almost all documents.
\usepackage{amssymb}
\usepackage{graphicx}
\usepackage{color}
\usepackage[colorlinks]{hyperref}


%%%%%%%%%%%%%%%%%%%%%%% amsthm theorem styles %%%%%%%%%%%%%%%%%%%%%%%%

% Current styles are based on suggestions by the amsthm package
% authors.
\theoremstyle{plain}
\newtheorem{theorem}{Theorem}[section]
\newtheorem{lemma}[theorem]{Lemma}
\newtheorem{proposition}[theorem]{Proposition}
\newtheorem{corollary}[theorem]{Corollary}
\newtheorem{claim}[theorem]{Claim}

\theoremstyle{definition}
\newtheorem{definition}[theorem]{Definition}
\newtheorem{example}[theorem]{Example}
\newtheorem{conjecture}[theorem]{Conjecture}

\theoremstyle{remark}
\newtheorem{remark}[theorem]{Remark}
\newtheorem{note}[theorem]{Note}
\newtheorem{case}{Case}


%%%%%%%%%%%%%%%%%%%%%%%%% custom definitions %%%%%%%%%%%%%%%%%%%%%%%%%

% Since we are collaborating, we need to be careful with personal
% shortcuts to avoid clash.
\newcommand{\C}{\mathbb{C}}
\newcommand{\F}{\mathbb{F}}
\renewcommand{\P}{\mathbb{P}}
\newcommand{\Q}{\mathbb{Q}}

\newcommand{\preper}{\mathrm{PrePer}}


%%%%%%%%%%%%%%%%%%%%%%%%%%% document body %%%%%%%%%%%%%%%%%%%%%%%%%%%%

\begin{document}

\section{The Uniform Boundedness Conjecture}

\begin{definition}[Projective space]
\end{definition}

\begin{definition}[Rational maps and morphisms between projective spaces]
  A \emph{rational map of degree $d$} between projective spaces is a
  map
  \[
  \begin{gathered}
    \phi: \P^n \to \P^m,\\
    \phi(P) = [f_0(P), \dots, f_m(P)],
  \end{gathered}
  \]
  where $f_0, \dots, f_m \in \bar{K}[X_0, \dots, X_n]$ are homogeneous
  polynomials of degree $d$ with no common factors. The rational map
  $\phi$ is \emph{defined at} $P$ if at least one of the values
  $f_0(P), \dots, f_m(P)$ is nonzero. The rational map $\phi$ is
  called a \emph{morphism} if it is defined at every point of
  $\P^n(\bar{K})$, or equivalently, if the only solution to the
  simultaneous equations
  \[
  f_0(X_0, \dots, X_n) = \cdots = f_m(X_0, \dots, X_n) = 0
  \]
  is the trivial solution $X_0 = \cdots = X_n = 0$. If the polynomials
  $f_0, \dots, f_n$ have coefficients in $K$, we say that $\phi$
  \emph{is defined over} $K$.
\end{definition}

\begin{example}[Morphism]
\end{example}

\begin{definition}[Preperiod points in $P^n(K)$]
\end{definition}

\begin{theorem}[Northcott, 1950]
  Let $\phi: \P^n \to \P^n$ be a morphism of degree $d \ge 2$ defined
  over a number field $K$. Then $\preper(\phi, \P^n(K))$ is finite.
\end{theorem}

\begin{remark}
  So for each combination of $\phi$, $n$ and $K/\Q$, the number of
  period points is finite. However, it is not clear if there is a
  uniform bound. Morton and Silverman conjectured in 1994 that there
  is a uniform bound depending only on the degree of the morphism, the
  dimension of the projective space, and the degree of the number
  field.
\end{remark}

\begin{conjecture}[Morton---Silverman, 1994]
  Fix integers $d \ge 2$, $n \ge 1$, and $D \ge 1$. There is a
  constant $C(d, n, D)$ such that for all number fields $K/\Q$ of
  degree at most $D$ and all morphisms $\phi: \P^n \to \P^n$ of degree
  $d$ defined over $K$,
  \[
  \#\preper(\phi, \P^n(K)) \le C(d, n, D).
  \]
\end{conjecture}

\begin{remark}
  The conjecture is not even proved in the simplest case $(d, n, D) =
  (2, 1, 1)$, i.e., rational preperiod points of degree-two morphisms
  over $\Q$. Specializing further, there are some partial results
  about degree-two polynomials over $\Q$.

  Before we proceed, note that the sizes of finite orbits are $PGL_2$
  invariant (invariant under conjugation by fractional linear
  transformations). In the case of any quadratic polynomial $g$, there
  exists a linear function $l$ such that $l \circ g \circ l^{-1}$ is
  of the form $z^2 + c$. So it suffices to consider $z^2 + c$.

  It has been proved in several papers that
\end{remark}

\begin{theorem}
  There are infinitely many quadratic polynomials with a rational
  point of exact point 3, while there are no quadratic polynomials
  with a rational point of exact period 4 or 5.
\end{theorem}

Over the past weeks we have been studying these papers, and we will
summarize the basic ideas and techniques here.



\section{Morton's Result on Rational 4-Cycles of Quadratic Maps}

\begin{theorem} [Morton, 1998]
  The quadratic map $f(z) = z^2 + c$ for $c \in \Q$ has no rational
  points of exact period 4.
\end{theorem}

% The following definition is somewhat problematic. The basic ideas
% are correct, but there are some crucial paradigm shifts here (e.g.,
% univariate polynomial for specific $f = z^2 + c$ to multivariate
% polynomial characterizing both $f$ and $z$; polynomial equation to
% algebraic curve), so I suppose we should better be precise.

% $C_1(N)$ is a curve, not a polynomial function. The polynomial
% function is denoted $\Phi_{n, \sigma}(x)$ (should be converted to
% $\Phi_{N, f}(z)$ or better yet $\Phi_{N, c}(z)$ using our notation).

% The curve $C_1(N)$ is defined as the curve consisting of pairs $(c,
% z)$ ($c$ stands for the polynomial $z^2 + c$, same thing) satisfying
% the polynomial $\Phi_N(c, z)$.

% Also, I think we should give a little bit more details about how the
% rational points are dealt with. Otherwise it will be quite confusing
% (or totally boring due to lack of math) to the audience.

% I'll add some details later.

\begin{definition}
  $C_1(N)$ is the algebraic curve consisting of pairs of quadratic
  polynomials and points of period $N$.
  \[
  C_1(N) = \prod_{d|n}(f^d(x) - x)^{\mu(n/d)}
  \]
\end{definition}

\begin{example}
  $C_1(4) = \frac{f^4(x) - x}{f^2(x) - x}$ when considering a
  particular function $f$.
\end{example}

\begin{remark}
  From the above example, we can see that the roots of $C_1(4)$ are
  precisely the points with exact period 4. Morton demonstrates that
  $C_1(4)$ is a model for the modular curve $X_1(16)$ for which its
  rational points were already known. $C_1(4)$ can then be shown to
  have no finite rational points.
	
  One of our current goals is to study the methods used in Morton's
  paper and see if it is reproducible over quadratic fields.
\end{remark}

\begin{remark}
  The strategy of comparing $C_1(N)$ to modular curves, however, does
  not work for case $N = 5$ as shown in the following paper and is
  unlikely to work for $N > 5$.
\end{remark}

\end{document}
