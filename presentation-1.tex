\documentclass{amsart}


%%%%%%%%%%%%%%%%%%%%%%%%%%%%%% packages %%%%%%%%%%%%%%%%%%%%%%%%%%%%%%

% Here are some common packages used in almost all documents.
\usepackage{amssymb}
\usepackage{graphicx}
\usepackage{color}
\usepackage[colorlinks]{hyperref}


%%%%%%%%%%%%%%%%%%%%%%% amsthm theorem styles %%%%%%%%%%%%%%%%%%%%%%%%

% Current styles are based on suggestions by the amsthm package
% authors.
\theoremstyle{plain}
\newtheorem{theorem}{Theorem}[section]
\newtheorem{lemma}[theorem]{Lemma}
\newtheorem{proposition}[theorem]{Proposition}
\newtheorem{corollary}[theorem]{Corollary}
\newtheorem{claim}[theorem]{Claim}

\theoremstyle{definition}
\newtheorem{definition}[theorem]{Definition}
\newtheorem{example}[theorem]{Example}
\newtheorem{conjecture}[theorem]{Conjecture}

\theoremstyle{remark}
\newtheorem{remark}[theorem]{Remark}
\newtheorem{note}[theorem]{Note}
\newtheorem{case}{Case}


%%%%%%%%%%%%%%%%%%%%%%%%% custom definitions %%%%%%%%%%%%%%%%%%%%%%%%%

% Since we are collaborating, we need to be careful with personal
% shortcuts to avoid clash.
\newcommand{\C}{\mathbb{C}}
\newcommand{\F}{\mathbb{F}}
\renewcommand{\P}{\mathbb{P}}
\newcommand{\Q}{\mathbb{Q}}

\newcommand{\preper}{\mathrm{PrePer}}


%%%%%%%%%%%%%%%%%%%%%%%%%%% document body %%%%%%%%%%%%%%%%%%%%%%%%%%%%

\begin{document}

\section{The Uniform Boundedness Conjecture}

\begin{definition}[Projective space]
\end{definition}

\begin{definition}[Rational maps and morphisms between projective spaces]
  A \emph{rational map of degree $d$} between projective spaces is a
  map
  \[
  \begin{gathered}
    \phi: \P^N \to \P^M,\\
    \phi(P) = [f_0(P), \dots, f_M(P)],
  \end{gathered}
  \]
  where $f_0, \dots, f_M \in \bar{K}[X_0, \dots, X_N]$ are homogeneous
  polynomials of degree $d$ with no common factors. The rational map
  $\phi$ is \emph{defined at} $P$ if at least one of the values
  $f_0(P), \dots, f_M(P)$ is nonzero. The rational map $\phi$ is
  called a \emph{morphism} if it is defined at every point of
  $\P^N(\bar{K})$, or equivalently, if the only solution to the
  simultaneous equations
  \[
  f_0(X_0, \dots, X_N) = \cdots = f_M(X_0, \dots, X_N) = 0
  \]
  is the trivial solution $X_0 = \cdots = X_N = 0$. If the polynomials
  $f_0, \dots, f_N$ have coefficients in $K$, we say that $\phi$
  \emph{is defined over} $K$.
\end{definition}

\begin{example}[Morphism]
\end{example}

\begin{definition}[Preperiod points in $P^N(K)$]
\end{definition}

\begin{theorem}[Northcott, 1950]
  Let $\phi: \P^N \to \P^N$ be a morphism of degree $d \ge 2$ defined
  over a number field $K$. Then $\preper(\phi, \P^N(K))$ is finite.
\end{theorem}

\begin{remark}
  So for each combination of $\phi$, $N$ and $K/\Q$, the number of
  period points is finite. However, it is not clear if there is a
  uniform bound. Morton and Silverman conjectured in 1994 that there
  is a uniform bound depending only on the degree of the morphism, the
  dimension of the projective space, and the degree of the number
  field.
\end{remark}

\begin{conjecture}[Morton---Silverman, 1994]
  Fix integers $d \ge 2$, $N \ge 1$, and $D \ge 1$. There is a
  constant $C(d, N, D)$ such that for all number fields $K/\Q$ of
  degree at most $D$ and all morphisms $\phi: \P^N \to \P^N$ of degree
  $d$ defined over $K$,
  \[
  \#\preper(\phi, \P^N(K)) \le C(d, N, D).
  \]
\end{conjecture}

\begin{remark}
  The conjecture is not even proved in the simplest case $(d, N, D) =
  (2, 1, 1)$, i.e., rational preperiod points of degree-two morphisms
  over $\Q$. Specializing further, there are some partial results
  about degree-two polynomials over $\Q$.

  Before we proceed, note that the sizes of finite orbits are $PGL_2$
  invariant (invariant under conjugation by fractional linear
  transformations). In the case of any quadratic polynomial $g$, there
  exists a linear function $l$ such that $l \circ g \circ l^{-1}$ is
  of the form $z^2 + c$. So it suffices to consider $z^2 + c$.

  It has been proved in several papers that
\end{remark}

\begin{theorem}
  There are infinitely many quadratic polynomials with a rational
  point of exact point 3, while there are no quadratic polynomials
  with a rational point of exact period 4 or 5.
\end{theorem}

Over the past weeks we have been studying these papers, and we will
summarize the basic ideas and techniques here.

\end{document}
