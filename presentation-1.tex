\documentclass{amsart}


%%%%%%%%%%%%%%%%%%%%%%%%%%%%%% packages %%%%%%%%%%%%%%%%%%%%%%%%%%%%%%

% Here are some common packages used in almost all documents.
\usepackage{amssymb}
\usepackage{graphicx}
\usepackage{color}
\usepackage[colorlinks]{hyperref}


%%%%%%%%%%%%%%%%%%%%%%% amsthm theorem styles %%%%%%%%%%%%%%%%%%%%%%%%

% Current styles are based on suggestions by the amsthm package
% authors.
\theoremstyle{plain}
\newtheorem{theorem}{Theorem}[section]
\newtheorem{lemma}[theorem]{Lemma}
\newtheorem{proposition}[theorem]{Proposition}
\newtheorem{corollary}[theorem]{Corollary}
\newtheorem{claim}[theorem]{Claim}

\theoremstyle{definition}
\newtheorem{definition}[theorem]{Definition}
\newtheorem{example}[theorem]{Example}
\newtheorem{conjecture}[theorem]{Conjecture}

\theoremstyle{remark}
\newtheorem{remark}[theorem]{Remark}
\newtheorem{note}[theorem]{Note}
\newtheorem{case}{Case}


%%%%%%%%%%%%%%%%%%%%%%%%% custom definitions %%%%%%%%%%%%%%%%%%%%%%%%%

% Since we are collaborating, we need to be careful with personal
% shortcuts to avoid clash.
\newcommand{\C}{\mathbb{C}}
\newcommand{\F}{\mathbb{F}}
\renewcommand{\P}{\mathbb{P}}
\newcommand{\Q}{\mathbb{Q}}

\newcommand{\preper}{\mathrm{PrePer}}
\newcommand{\gal}{\mathrm{Gal}}

\newcommand{\tup}[1]{\left<#1\right>}


%%%%%%%%%%%%%%%%%%%%%%%%%%% document body %%%%%%%%%%%%%%%%%%%%%%%%%%%%

\begin{document}

\section{The Uniform Boundedness Conjecture}

\begin{definition}[Projective space]
\end{definition}

\begin{definition}[Rational maps and morphisms between projective spaces]
  A \emph{rational map of degree $d$} between projective spaces is a
  map
  \[
  \begin{gathered}
    \phi: \P^n \to \P^m,\\
    \phi(P) = [f_0(P), \dots, f_m(P)],
  \end{gathered}
  \]
  where $f_0, \dots, f_m \in \bar{K}[X_0, \dots, X_n]$ are homogeneous
  polynomials of degree $d$ with no common factors. The rational map
  $\phi$ is \emph{defined at} $P$ if at least one of the values
  $f_0(P), \dots, f_m(P)$ is nonzero. The rational map $\phi$ is
  called a \emph{morphism} if it is defined at every point of
  $\P^n(\bar{K})$, or equivalently, if the only solution to the
  simultaneous equations
  \[
  f_0(X_0, \dots, X_n) = \cdots = f_m(X_0, \dots, X_n) = 0
  \]
  is the trivial solution $X_0 = \cdots = X_n = 0$. If the polynomials
  $f_0, \dots, f_n$ have coefficients in $K$, we say that $\phi$
  \emph{is defined over} $K$.
\end{definition}

\begin{example}[Morphism]
\end{example}

\begin{definition}[Preperiod points in $P^n(K)$]
\end{definition}

\begin{theorem}[Northcott, 1950]
  Let $\phi: \P^n \to \P^n$ be a morphism of degree $d \ge 2$ defined
  over a number field $K$. Then $\preper(\phi, \P^n(K))$ is finite.
\end{theorem}

\begin{remark}
  So for each combination of $\phi$, $n$ and $K/\Q$, the number of
  period points is finite. However, it is not clear if there is a
  uniform bound. Morton and Silverman conjectured in 1994 that there
  is a uniform bound depending only on the degree of the morphism, the
  dimension of the projective space, and the degree of the number
  field.
\end{remark}

\begin{conjecture}[Morton---Silverman, 1994]
  Fix integers $d \ge 2$, $n \ge 1$, and $D \ge 1$. There is a
  constant $C(d, n, D)$ such that for all number fields $K/\Q$ of
  degree at most $D$ and all morphisms $\phi: \P^n \to \P^n$ of degree
  $d$ defined over $K$,
  \[
  \#\preper(\phi, \P^n(K)) \le C(d, n, D).
  \]
\end{conjecture}

\begin{remark}
  The conjecture is not even proved in the simplest case $(d, n, D) =
  (2, 1, 1)$, i.e., rational preperiod points of degree-two morphisms
  over $\Q$. Specializing further, there are some partial results
  about degree-two polynomials over $\Q$.

  Before we proceed, note that the sizes of finite orbits are $PGL_2$
  invariant (invariant under conjugation by fractional linear
  transformations). In the case of any quadratic polynomial $g$, there
  exists a linear function $l$ such that $l \circ g \circ l^{-1}$ is
  of the form $z^2 + c$. So it suffices to consider $z^2 + c$.

  It has been proved in several papers that
\end{remark}

\begin{theorem}
  There are infinitely many quadratic polynomials with a rational
  point of exact point 3, while there are no quadratic polynomials
  with a rational point of exact period 4 or 5.
\end{theorem}

Over the past weeks we have been studying these papers, and we will
summarize the basic ideas and techniques here.



\section{Morton's Result on Rational 4-Cycles of Quadratic Maps}

\begin{theorem} [Morton, 1998]
  The quadratic map $f(z) = z^2 + c$ for $c \in \Q$ has no rational
  points of exact period 4.
\end{theorem}

To demonstrate the ideas involved in the proof, first we give some
definitions for a general exact period $N$.

\begin{definition}
  Let $f(z) = z^2 + c$ be a quadratic polynomial. Let $\Phi_{N, c}(z)$
  be
  \[
  \Phi_{N,c}(z) = \prod_{d|n}(f^{\circ d}(x) - x)^{\mu(n/d)}.
  \]
\end{definition}

\begin{example}
  \[
  \Phi_{4, c}(z) = \frac{f^4(x) - x}{f^2(x) - x}.
  \]
\end{example}

You might have noticed that this is an analog to classical cyclotomic
polynomials. In fact, using the M\"obius inversion formula, it is not
hard to check that the roots of $\Phi_{N, c}(z)$ are the points of
exact period $N$ for the map $f$.

Now for a fixed $N$, we consider $\Phi_{N, c}(z)$ as a polynomial of
both $c$ and $z$, i.e., we consider it as $\Phi_N(z, c)$. Then this
polynomial defines an algebraic curve.

\begin{definition}
  Let $C_1(N)$ be the algebraic curve defined by $\Phi_N(z, c)$, i.e.,
  \[
  C_1(N) = \{(z, c) : \Phi_N(z, c) = 0\}.
  \]
\end{definition}

Since we are looking for rational points of exact period $N$ for $f(z)
= z^2 + c$ where $c \in \Q$, it boils down to finding rational points
$(z, c)$ on the curve $C_1(N)$. In fact, to be more precise, we are
looking for affine rational points on $C_1(N)$, since there could be
rational points at infinity.

\begin{example}[Rational points at infinity]
  Consider the simple example of $\Phi_2(x, y)$, which defines the
  curve $x^2 + x + y + 1 = 0$. (Here I've chosen $x$ and $y$ for
  notational convenience.) Since we are working in $P^1$, we should
  really homogenize the equation to
  \[
  X^2 + XZ + YZ + Z^2 = 0.
  \]
  Then $[X, Y, Z] = [0, 1, 0]$ is a rational solution, but since $Z$
  (which can be thought of as the denominator) is 0, this is a
  rational point at infinity.
\end{example}

Back to the special case $N = 4$. In this case $C_1(N)$ is given by
$\Phi_4(z, c)$, which is too complicated to write down here. However,
with some very clever change of coordinates, we can reduce it to a
birationally equivalent curve given by a simple equation
\[
v^2 = u(u^2 + 1)(1 + 2u - u^2).
\]
This equation was already studied in the literature and was shown to
be an equation for the modular curve $X_1(16)$. (Fancy jargon here,
doesn't really matter --- the point is it was already extensively
studied and all of its rational points are known.)

It turns out that the new curve have six rational points in total:
$(0,\, 0)$, $(\pm 1,\, \pm 2)$, and the point at infinity. However,
when we trace back to our original equation $\Phi_4(z, c)$ by
inverting the change of coordinates, it could happen that these
rational points are at infinity. In fact, by studying the prime
divisors of the function field of the curve above, it can be shown
that all six rational points map back to points at infinity on
$C_1(4)$. Therefore, there are no affine rational points on $C_1(4)$,
so we are done.

\begin{remark}
  One of our current goals is to study the methods used in Morton's
  paper and see if it can be applied to quadratic fields, that is, $D
  = 2$ in the uniform boundedness conjecture.
\end{remark}

\begin{remark}
  The strategy of comparing $C_1(N)$ to modular curves, however, does
  not work for case $N = 5$ as shown in the following paper and is
  unlikely to work for $N > 5$.
\end{remark}



\section{The case of 5-cycles}

The case of 5-cycles is more involved, since $C_1(5)$ is a much more
complicated than $C_1(4)$. In fact, $C_1(4)$ has genus 2, while
$C_1(5)$ has genus 14.

However, we notice that there is an obvious automorphism on
$C_1(N)$. In fact, if $z$ is in an exact $N$-cycle of $f(z) = z^2 + c$,
then $f(z)$ is also in the same $N$-cycle; so $\sigma: (z, c) \mapsto
(z^2 + c, c)$ defines an automorphism on the curve $C_1(N)$. Moreover,
$\sigma$ has order $N$, since $f^{\circ N}(z) = z$ for any $z$ in any
$N$-cycle. So we can consider the quotient curve $C_0(N) =
C_1(N)/\tup{\sigma}$. It turns out that $C_0(N)$ is also a nonsingular
algebraic curve over $\Q$, whose rational points correspond to pairs
of a quadratic polynomial $z^2 + c$ with $c \in \Q$, and a
$\gal(\bar{\Q}/\Q)$-stable $N$-cycle.

$C_0(N)$ is much more manageable than $C_1(N)$. For our case $N = 5$,
$C_1(5)$ has genus 14, while $C_0(5)$ only has genus 2, so it is
easier to study. However, here is a subtle point. A
$\gal(\bar{\Q}/\Q)$-stable $N$-cycle might not be a rational cycle,
that is, might not correspond to rational points on $C_1(N)$.

\begin{example}[$\gal(\bar{\Q}/\Q)$-stable $N$-cycle not in $\Q$]
  The paper gives an example on p.~3, but we might as well give a
  conceptual example in the talk to demonstrate the point, e.g., a
  2-cycle $\{a + b \sqrt{2},\ a - b \sqrt{2}\}$.
\end{example}

So our goal is basically to find all rational points on $C_0(5)$ (as
in the $N=4$ case), and to prove that these rational points are either
at infinity or do not correspond to rational points on $C_1(5)$.

With some change of coordinates we can show that $C_0(5)$ is
birationally equivalent to the hyperelliptic curve
\[
y^2 = x^6 + 8x^5 + 22x^4 + 22x^3 + 5x^2 + 6x + 1.
\]

It is easy to find six rational points on this curve: $(0,\, \pm 1)$,
$(-3,\, \pm 1)$, and two points at infinity. After we recovery $c$
from these points (by tracing back through the change of coordinates
we did before), we can show that three such points correspond to $c =
\infty$, and the remaining three correspond to $5$-cycles defined over
degree-5 extensions of $\Q$.

The difficult part of the proof is to show that we have found all the
rational points on $C_0(5)$, i.e., there are only six rational
points. This part requires machinery from algebraic geometry, and the
techinques involved are not applicable to larger $N$.

\end{document}