\documentclass{amsart}


%%%%%%%%%%%%%%%%%%%%%%%%%%%%%% packages %%%%%%%%%%%%%%%%%%%%%%%%%%%%%%

% Here are some common packages used in almost all documents.
\usepackage{amssymb}
\usepackage{graphicx}
\usepackage{color}
\usepackage{hyperref}

\usepackage{enumerate} % customizable styles for counter printing


%%%%%%%%%%%%%%%%%%%%%%% amsthm theorem styles %%%%%%%%%%%%%%%%%%%%%%%%

% Current styles are based on suggestions by the amsthm package
% authors.
\theoremstyle{plain}
\newtheorem{theorem}{Theorem}[section]
\newtheorem{lemma}[theorem]{Lemma}
\newtheorem{proposition}[theorem]{Proposition}
\newtheorem{corollary}[theorem]{Corollary}
\newtheorem{claim}[theorem]{Claim}

\theoremstyle{definition}
\newtheorem{definition}[theorem]{Definition}
\newtheorem{example}[theorem]{Example}
\newtheorem{conjecture}[theorem]{Conjecture}

\theoremstyle{remark}
\newtheorem{remark}[theorem]{Remark}
\newtheorem{note}[theorem]{Note}
\newtheorem{case}{Case}


%%%%%%%%%%%%%%%%%%%%%%%%% custom definitions %%%%%%%%%%%%%%%%%%%%%%%%%

% Since we are collaborating, we need to be careful with personal
% shortcuts to avoid clash.
\newcommand{\C}{\mathbb{C}}
\newcommand{\F}{\mathbb{F}}
\newcommand{\N}{\mathbb{N}}
\renewcommand{\O}{\mathcal{O}}
\renewcommand{\P}{\mathbb{P}}
\newcommand{\Q}{\mathbb{Q}}
\newcommand{\U}{\mathcal{U}}
\newcommand{\Z}{\mathbb{Z}}

\renewcommand{\l}{\lambda}

\newcommand{\es}{\emptyset}
\newcommand{\ol}{\overline}

\newcommand{\gal}{\mathrm{Gal}}
\newcommand{\preper}{\mathrm{PrePer}}
\newcommand{\res}{\mathrm{Res}}

\newcommand{\set}[1]{\left\{#1\right\}}
\newcommand{\tup}[1]{\left<#1\right>}
