\section{Introduction}
\label{sec:bg}

This work has its roots in the study of the dynamics of morphisms over
projective spaces, in particular the periodic points (or more
generally, preperiodic points) of such morphisms. One profound
conjecture in this field is the Uniform Boundedness Conjecture, put
forth by Morton and Silverman in 1994~\cite{MR1264933}, which bounds
the number of preperiodic points of a morphism in terms of the the
degree of the base field (over the rational field $\Q$), the dimension
of the projective space, and the degree of the morphism. The
conjecture is strong in the sense that the dependence on the morphism
is the bare minimum --- the coefficients of the morphism are not
referenced at all.

We define the necessary objects leading up to the Uniform Boundedness
Conjecture, briefly describe the status of the conjecture, and discuss
the specific cases that we concentrate on. At the end of this section,
our main results are presented, along with some open conjectures.


\begin{definition}
  A rational map $\phi$ is called a \emph{morphism} if it is defined
  everywhere in $\P^N(\ol{K})$, or equivalently, if the only solution
  in $\ol{K}$ to the simultaneous equations
  \[
  f_0(X_0, \dots, X_N) = \cdots = f_M(X_0, \dots, X_N) = 0
  \]
  is the trivial solution $X_0 = \dots = X_N = 0$.
\end{definition}

\begin{definition}
  Given any morphism $\phi: \P^N \to \P^N$, the \emph{preperiodic
    points of $\phi$} over some field $K$ is the collection of points
  in $\P^N(K)$ with finite forward orbits, or equivalently,
  \[
  \preper(\phi, \P^N(K)) = \set{P \in \P^N(K) : \text{exist $n_1, n_2
      \in \N$ such that $\phi^{n_1}(P) = \phi^{n_2}(P)$}}.
  \]
\end{definition}

Now we are ready to state the Uniform Boundedness Conjecture.

\begin{conjecture}[Morton-Silverman, 1994 ~\cite{MR1264933}]
  Fix integers $d \ge 2$, $N \ge 1$, and $D \ge 1$. There is a
  constant $C(d, N, D)$ such that for all number fields $K/\Q$ of
  degree at most $D$ and all morphisms $\P^N \to \P^N$ of degree $d$
  defined over $K$,
  \[
  \#\preper(\phi, \P^N(K)) \le C(d, N, D).
  \]
\end{conjecture}

Very little is known about this conjecture. In fact, even the simplest
case $(d, N, D) = (2, 1, 1)$ is not known, that is, the problem of
bounding the number of rational preperiodic points of quadratic
rational morphisms. If we specialize to quadratic polynomials (i.e.,
maps of the type $[X_0, X_1] \mapsto [a X_0^2 + b X_0 X_1 + c X_1^2,
X_1^2]$, conveniently written as $a z^2 + b z + c$, where $a \ne 0$),
note that $a z^2 + b z + c$ can be reduced to the form $z^2 + c'$ by
linear conjugation, which preserves the sizes of orbits, so it
suffices to consider quadratic polynomials of the type $\phi_c(z) =
z^2 + c$. In this case of $\phi_c(z) = z^2 + c$, the uniform bound is
again not known, but there are partial results for specific small
periods.

One can easily show that there are one-parameter families of
$c$-values for which $\phi_c(z)$ has rational periodic point(s) of
exact period 1, 2, or 3 \cite{MR1199627}. It was shown by
Morton~\cite{MR1665198} that $\phi_c$ cannot have rational periodic
points of exact period 4. The non-existence result was extended to
exact period 5 by Flynn, Poonen, and Schaefer~\cite{MR1480542}, and to
exact period 6 (conditional on the Birch and Swinnerton-Dyer
conjecture for a specific abelian variety) by Stoll~\cite{MR2465796}.

We are interested in generalizing the aforementioned results for
quadratic polynomials and specific small periods to quadratic
extensions of $\Q$, i.e., the $D = 2$ case. For a more detailed
discussion of other takes on the Uniform Boundedness Conjecture,
together with a list of references, see Section 3.3 of
\cite{MR2316407}.

From now on, unless otherwise stated, the dimension of the projective
space is always 1, and we reserve the letter $N$ for the exact period
of periodic points.

\subsection{Main results}
\label{subsec:results}

Below are our main results.

For $N = 4$, by putting together and re-interpreting others' results,
we have the following theorem.

\begin{theorem}
  \label{th:n=4-infinite}
  There are infinitely many rational values $c \in \Q$ such that
  $\phi_c(z) = z^2 + c$ has periodic points of exact period 4 defined
  in some quadratic extension to $\Q$. All such $c$ are given by
  \[
  c = \frac{1 - 4t^3 - t^6}{4t^2(t^2 - 1)}
  \]
  with $t \in \Q$.

  Moreover, for each such $c \in \Q$, $\phi_c(z)$ has exactly four
  such periodic points.
\end{theorem}

For $N = 5$, we were able to prove the following theorem.

\begin{theorem}
  \label{th:n=5-finite}
  There are finitely many rational values $c \in \Q$ such that
  $\phi_c(z) = z^2 + c$ has periodic points of exact period 5 defined
  in some quadratic extension to $\Q$.
\end{theorem}

In fact, all empirical evidence (see
Subsection~\ref{subsec:seek-quadratic} and \ref{subsec:pab-ratpoint})
points to the following stronger conjecture that no such $c$ exists.

\begin{conjecture}
  \label{cj:n=5-zero}
  There are no rational values $c$ such that $\phi_c(z) = z^2 + c$ has
  periodic points of exact period 5 defined in some quadratic
  extension to $\Q$.
\end{conjecture}

One general result we produced is the following.

\newcommand{\nd}{\frac{N}{(N, d)}}

\begin{theorem}
  Let $N \in \N^*$, $c \in \Q$, $K$ be a Galois extension of $\Q$ with
  degree $d = [K : \Q]$, and $\set{z_0, \dots, z_{N-1}} \subset K$ be
  an exact $N$-cycle of $\phi_c(z) = z^2 + c$. Then exactly one of the
  following holds:
  \begin{enumerate}[(i)]
  \item $z_{i\nd} = \t(z_0)$ for some $0 \le i \le (N, d)-1$ and some
    nontrivial $\t \in \gal(K/\Q)$;

  \item $\set{z_0, \dots, z_{N-1}} \cap \set{\t(z_0), \dots,
      \t(z_{N-1})} = \es$ for all nontrivial $\t \in \gal(K/\Q)$.
  \end{enumerate}
\end{theorem}

In fact, we have good reasons to believe that the second case never
occurs, hence we propose the following conjecture.

\begin{conjecture}
  \label{cj:galois-conjugate}
  Let $N \in \N^*$, $c \in \Q$, $K$ be a Galois extension of $\Q$ with
  degree $d = [K : \Q]$, and $\set{z_0, \dots, z_{N-1}} \subset K$ be
  an exact $N$-cycle of $\phi_c(z) = z^2 + c$. Then $z_{i\nd} =
  \t(z_0)$ for some $0 \le i \le (N, d)-1$ and some nontrivial $\t \in
  \gal(K/\Q)$.
\end{conjecture}

Assuming this conjecture, some problems of finding quadratic points
(or even points in higher degree extensions) may be reduced to finding
rational points. In particular, it turns out that this conjecture
implies Conjecture~\ref{cj:n=5-zero}.

Before we close the introduction section, note that we have a handful
of C++, Mathematica, or Sage programs to verify various results and
provide empirical evidence for certain conjectures. These programs are
available in a public repository on GitHub; see \cite{src} in the
reference list. Whenever we need to refer to such a program, we print
its path in the repository in monospaced font; for instance,
\texttt{mma/abc.m} refers to the Mathematica package \texttt{abc.m} in
the \texttt{mma/} directory of the repository.

%%% Local Variables:
%%% TeX-master: "report"
%%% End:
