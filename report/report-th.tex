\section{Theoretical results}
\label{sec:th}

We want to extend the results from the previous section by lifting the
base field from the rational field $\Q$ to quadratic number fields,
i.e., raising the degree $D$ in the Uniform Boundedness Conjecture
statement from 1 to 2.

%%% TODO
\begin{remark}
  \textbf{We are not strictly increasing $D$ to 2, but rather, we are
    still restricting ourselves to $c \in \Q$. Why?}
  % e.g., allowing c to be quadratic makes the N=4 case trivial, by
  % appealing to the result about quadratic points on X_1(16).
\end{remark}

% I use \mathcal{U} to denote our "universe".
Let $\U$ be the collection of all rational or quadratic elements in
$\ol{\Q}$,\footnote{%
  This is a bad set with very little structure. We only refer
  to it to simplify our statements.}
i.e.,
\[
\U = \set{\alpha \in \ol{\Q}: \text{$a \alpha^2 + b \alpha + c = 0$
    for some $a, b, c \in \Q$}},
\]
or equivalently,
\[
\U = \bigcup_{[K : \Q] = 2}K,
\]
where the union is the union of sets, taken over all quadratic number
fields.

We pose the following question:

% The question below is dead, since when I checked the proof, it turns
% out that we cannot answer this question for N = 5, whereas it's
% trivial for N = 4. (Didn't we find that it is finite for N = 5?)
% --------------------------------------------------------------------
% For a given $N$, how many $c \in \U$ are there such that $\phi_c(z)
% = z^2 + c$ has a periodic point of exact period $N$ in the same
% quadratic extension of $\Q$ as $c$?%
% \footnote{If $c$ is rational, then $c$ lies in any quadratic number
% field, so we allow the periodic point to be any element of $\U$. See
% Question~(ii).}

\begin{question}
  \label{question}
  How many $c \in \Q$ are there such that $\phi_c(z) = z^2 + c$ has a
  periodic point in $\U$ of exact period $N$?
\end{question}

\begin{remark}
  In the question above, we only asked for the number of $c$. However,
  once $c$ is known, we can immediately determine all the periodic
  points of exact period $N$ from the equation $\Phi_N(z, c) = 0$; and
  the number of such points is bound above by
  \[
  \deg_z \Phi_N(z, c) = \sum_{d|N}2^d \mu(N/d),
  \]
  which is a constant only depending on $N$. Moreover,
  in some cases we can obtain a much sharper bound.
\end{remark}

It turns out that we can fully answer Questions~(\ref{question}) for
$N = 4$, and partially answer it for $N = 5$. For $N = 4$, the answer
is infinite, since an explicit parametrization of $c$. For $N =
5$, the answer is finite, but at the time of writing we do not have a
concrete bound. We conjecture that the answer is in fact zero.

The results highlighted above are presented below in detail with
proof.

\subsection{The quadratic $N = 4$ case}

Points with exact period 4 for $\phi_c(z) = z^2 + c$ are quite 
well-known thanks to a parametrization of all 4-cycles given by
Morton:

\begin{theorem} [Morton, Proposition~2 in \cite{MR1665198}]
  For $\phi_c(z) = z^2 + c$,
  \begin{enumerate}[(i)]
  \item $c$ and the points of exact period 4 can be parametrized over
    $\Q$ by the following:
    \[
    \begin{gathered}
      c = \frac{1 - 4t^3 - t^6}{4t^2(t^2 - 1)}, \\
      x_1 = \frac{t^4 - t^2 + \sqrt{(t^4 - 1)(t^2 + 2t - 1)}}{2t(t^2 -
        1)},\,
      x_2 = \frac{1 - t^2 + t \sqrt{(t^4 - 1)(t^2 + 2t - 1)}}{4t^2(t^2
        - 1)}, \\
      x_3 = \frac{t^4 - t^2 - \sqrt{(t^4 - 1)(t^2 + 2t - 1)}}{2t(t^2 -
        1)},\,
      x_4 = \frac{1 - t^2 - t \sqrt{(t^4 - 1)(t^2 + 2t - 1)}}{4t^2(t^2
        - 1)},
  \end{gathered}
  \]
  where $t = x_1 + x_3$.

  \end{enumerate}
\end{theorem}

Furthermore, Panraksa demonstrated that all 4-cycles with points
defined in quadratic number fields are given with $t$ ranging over
$\Q$.

\begin{theorem} [Panraksa, Theorem~1.5.2 in \cite{MR2982105}]
	Let $c \in \Q$ and $x_1, x_2, x_3, x_4 \in K$, $K$ a quadratic 
	number field. Then $t \in \Q$, so all 4-cycles over quadratic
	fields are obtained by ranging $t$ over $\Q$.
\end{theorem}

So we have an explicit expression for all points of exact period 4 over
$\C$ for any $c$ and over quadratic fields for $c \in \Q$. Additionally,
if we allow $t$ to vary over $\Z$ or $\Q$, then it follows that:

\begin{corollary}
	There are infinitely many $c$ in $\Q$ with points in quadratic
	fields of exact period 4 for $\phi_c(z)$.
\end{corollary}

Furthermore, we can specify how many such points exist for each $c
\in \Q$ by considering the possible degrees of factors of $\Phi_4(
z,c)$. The substitution of $c = \frac{1 - 4t^3 - t^6}{4t^2(t^2 - 1)}$
into $\Phi_4(x,c)$ and polynomial factorization yields:

\begin{lemma}
	For $c$ such that $\phi$ has 4-cycles over quadratic fields,
	the degree 12 polynomial $\Phi_4(z,c)$ must factor as $\{2,2,8\}$
	(list of degrees of factors, with the 8th degree factor not
	necessarily irreducible).
\end{lemma}

We already know that $\Phi_4(z_c)$ cannot factor as $\{2,2,2,2,4\}$
by Panraksa (Theorem~2.3.5 in \cite{MR2982105}). Therefore, we
obtain the following result:

\begin{theorem}
	For each $c \in \Q$, there is at most one 4-cycle with points
	defined over quadratic fields.
\end{theorem}

Subsequently, we sought to understand precisely which quadratic
fields could contain points of period 4. Due to the parametrization
given by Morton, this problem reduces to finding integers of the form
$d = (t^4 - 1)(t^2 + 2t - 1)$. Further discussion of this problem is
made in Section 5. %Zhiming to add actual \ref 

\subsection{The quadratic $N = 5$ case}

Let us recall from (\ref{eq:c0(5)}) and (\ref{eq:c-in-xy}) that $y^2 =
f(x)$ and $c = (P_0(x) + P_1(x) y)/h(x)$, so eliminating $y$ we have
\[
(c \cdot h(x) - P_0(x))^2 = P_1(x)^2 y^2 = P_1(x)^2 f(x),
\]
i.e., $x$ satisfies the polynomial equation
\[
[(c \cdot h - P_0)^2 - P_1^2 f](x) = 0.
\]
Since we are only looking for $x \in \U$,\footnote{%
  Recall that we are looking for $z \in \U$ and $c \in \Q$. On $C_0(5)
  = C_1(5) / \tup{\sigma}$, whose points correspond to pairs $(\O,
  \phi_c)$, each orbit $\O$ is represented by its trace $\tau = z +
  \phi_c(z) + \cdots + \phi_c^4(z)$, which is in the same quadratic
  extension as $z$. The series of change of coordinates from $(\tau,
  c)$ to $(x, y)$ involves only arithmetic operations, so the
  resulting $x$ and $y$ are still in the same quadratic extension as
  $\tau$. In particular, we have $x \in \U$. See \cite{MR1480542} for
  details.}
there exist some rational coefficients $a$ and $b$ such that $x$
solves the quadratic equation
\[
x^2 + a x + b = 0.
\]
We have obtained two polynomial equations in $x$, so we may divide
$[(c \cdot h - P_0)^2 - P_1^2 f]$ by $x^2 + ax + b$ using long
division, and the remainder must be zero. Since $x^2 + ax + b$ is
quadratic in $x$, the remainder is linear, in the form
\[
\l_1(a, b, c) x + \l_0(a, b, c),
\]
where $\l_1$ and $\l_0$ are polynomials in $a$, $b$, and $c$ given by
%%% TODO
%%% Rewrite lambda_1 and lambda_2 by collecting in terms of c
%%% This will be much more reader friendly since we are dealing with
%%% the vanishing of leading terms later.
\[
\begin{aligned}
  \l_1&(a, b, c) = - 4a^5c - 8a^5 + 40a^4c + 74a^4 + 16a^3bc + 32a^3b
  - 16a^3c^2 - \\
  & 184a^3c - 271a^3 - 120a^2bc - 222a^2b + 96a^2c^2 + 416a^2c + 452a^2 -
  12ab^2c - \\
  & 24ab^2 + 32abc^2 + 368abc + 542ab - 144ac^2 - 380ac - 325a + 40b^2c + \\
  & 74b^2 - 96bc^2 - 416bc - 452b + 96c + 110,
\end{aligned}
\]
and
\[
\begin{aligned}
  \l_0&(a, b, c) = - 4a^4bc - 8a^4b + 40a^3bc + 74a^3b + 12a^2b^2c +
  24a^2b^2 - 16a^2bc^2 - \\
  & 184a^2bc - 271a^2b - 80ab^2c - 148ab^2 + 96abc^2 +
  416abc + 452ab - 4b^3c - \\
  & 8b^3 + 16b^2c^2 + 184b^2c + 271b^2 - 144bc^2 - 380bc -
  325b + 36c + 64.
\end{aligned}
\]
Note that $a$, $b$, and $c$ are all rational, so $\l_1(a, b, c)$ and
$\l_0(a, b, c)$ are rational, and hence from
\[
\l_1(a, b, c) x + \l_0(a, b, c) = 0
\]
we conclude that either $x$ is rational, or both $\l_1$ and $\l_0$ are
zero. Thanks to \cite{MR1480542}, we already fully understand the case
where $x$ is rational (in which case there are no corresponding
periodic points even in $\U$ --- they are either at infinity or
defined only in a quintic extension of $\Q$). Therefore, here we
consider $x \in \U \setminus \Q$, and hence
\[
\l_1(a, b, c) = \l_0(a, b, c) = 0.
\]

Notice that $c$ is a common root to $\l_1$ and $\l_0$, so by
Proposition~\ref{res}, we have
\[
\res_c(\l_1(a, b, c), \l_2(a, b, c)) = 0.
\]

Now we have three cases.

\begin{case}
  The leading coefficient of $\l_1(a, b, c)$, considered as a
  polynomial in $c$, vanishes. This happens when
  \[
  -16(a - 3)(a(a-3) - 2b) = 0,
  \]
  i.e., either $a = 3$, or $b = a(a-3)/2$, or both.

  If $a = 3$, plugging into $\l_1 = \l_2 = 0$ we get two polynomial
  equations in $b$ and $c$. Taking the resultant of the two with
  respect to $c$, we get $- 96b^7 - 1264b^5 - 4256b^5 + 32b^4 +
  13248b^3 + 37632b^2 + 92160b = 0$, which has only one rational root
  $b = 0$. However, $\l_1(3, 0, c) \equiv -64 \ne 0$, so there is no
  rational solution in this case.

  If $b = a(a-3)/2$, plugging into $\l_1 = \l_2 = 0$ we get two
  polynomials equations in $a$ and $c$. Again using the resultant we
  can show that there is no rational solution in this case. See the
  attached Mathematica notebook \texttt{mma/abc.nb} for details.
\end{case}

\begin{case}
  The leading coefficient of $\l_0(a, b, c)$, considered as a
  polynomial in $c$, vanishes. This happens when
  \[
  -16b(b - (a-3)^2) = 0,
  \]
  i.e., either $b = 0$ or $b = (a - 3)^2$. Similar to Case~1, we can
  easily show that there is no rational solution in this case. See
  \texttt{mma/abc.nb} for details.
\end{case}

\begin{case}
  Both of the leading coefficients of $\l_1(a, b, c)$ and $\l_0(a, b,
  c)$, considered as a polynomial in $c$, are non-vanishing, i.e.,
  both $\l_1(a, b, c)$ and $\l_0(a, b, c)$ are quadratic in
  $c$. Therefore, by Proposition~\ref{res}, $\res_c(\l_1(a, b, c),
  \l_0(a, b, c))$ is given by a polynomial in the six coefficients
  (three in $\l_1$ and three in $\l_0$). Mathematica is capable of
  computing this polynomial, and when we plug in the six coefficients,
  we have
  \[
  \res_c(\l_1(a, b, c), \l_0(a, b, c)) = P(a, b)
  \]
  for some polynomial $P(a, b) \in \Z[a, b]$.
\end{case}

\subsection{General results}

While studying the quadratic N = 4 case, we tried to mimic results
to the quadratic N = 6 case. In particular, the following result by
Panraksa is quite important in demonstrating that, for points of exact
period 4 in quadratic number fields $x_i$, the trace $t = \sum\limits_
{i=1}^4 x_i$ is rational.

\begin{theorem} [Panraksa, Lemma~2.3.1 in \cite{MR2982105}]
	Let $\{x_1, x_2, x_3, x_4\}$ be an exact 4-cycle defined in some
	quadratic number field $K$ for $\phi_c(z) = z^2 + c$ with $c \in \Q$.
	If $x_3 \neq \overline{x_1}$, then $\{x_1, x_2, x_3, x_4\} \cap 
	\{\overline{x_1}, \overline{x_2}, \overline{x_3}, \overline{x_4}\}
	= \emptyset$.
\end{theorem}

Panraksa then proves that $\{x_1, x_2, x_3, x_4\} \cap 
\{\overline{x_1}, \overline{x_2}, \overline{x_3}, \overline{x_4}\}
= \emptyset$ is impossible to get his final result. We managed to prove
a much more general version of his lemma:

\begin{theorem}
	Let $\phi_c(z) = z^2 + c$ with $c \in \Q$. Let $\{x_1, \ldots, x_{N}\}$ be
	an exact $N$-cycle defined over a Galois number field K, with N 
	divisible by	$d = [K:\Q ]$. Then either: \\
	\begin{itemize}
	\item $x_{\frac{iN}{d}+1} = \sigma(x_1)$ for some nontrivial $\sigma \in
	\gal(K/\Q)$, $i \in \Z$
	\item $\{x_1, \ldots, x_{N}\} \cap \{\tau(x_1), \ldots, \tau(x_{N})\} =
	\emptyset$, $\forall \tau \in \gal(K/\Q)$ nontrivial.
	\end{itemize}
\end{theorem}

\begin{proof}
	Note that $\{\tau(x_1), \ldots, \tau(x_{N})\}$ for any $\tau \in 
	\gal(K/\Q)$ is a cycle of $\phi_c$ because $c \in \Q$. Assume 
	that neither case is true. Then $x_{\frac{N}{d}+1} \neq \sigma(
	x_1)$ $\forall \sigma \in \gal(K/\Q)$ and $\{x_1, \ldots, x_{N}\} 
	\cap \{\tau(x_1), \ldots, \tau(x_{N})\} \neq \emptyset$ $\forall 
	\tau \in \gal(K/\Q)$. Then $\exists x_j \in \{x_1, \ldots, x_{N}\}
	\cap \{\tau(x_1), \ldots, \tau(x_{N})\}$ for some $\tau \in \gal(K/
	\Q)$. Applying $\phi$ an appropriate number of times allows us to 
	renumerate the cycle	such that $x_1$ is in the intersection. 
	Thus, $x_k = \tau(x_1)$ for some $1 \leq k \leq N$ and therefore $
	\tau \equiv \phi^{k-1}$ on the N-cycle (because Galois conjugation 
	commutes with $\phi$). Since $K$ is Galois, $\tau^d = Id$ and so:
	\[
		x_d = \tau^d(x_d) = \phi^{dk-d}(x_d) = x_{dk}
	\]
	Notice that the application of $\phi^{N-(d-1)}$ yields $x_1 =
	x_{dk-(d-1)} = x_{d(k-1) + 1}$. Notice that this produces a
	contradiction if $d(k-1) + 1 \not\equiv 1$ (modulo $N$): if $l$
	is the representative of $d(k-1) + 1$ modulo N, then $\{x_1, \ldots
	, x_l\}$ gives an $l$-cycle with $l < N$. Thus, it must be that
	$d(k-1) + 1 \equiv 1$ (modulo $N$). The only such $k$ are given
	by $k = \frac{lN}{d} + 1$ for some $l \in \Z$, so $x_{\frac{lN}{d}
	+ 1} = x_k = \tau(x_1)$.	However, we already have that
	$x_{\frac{iN}{d}+1} \neq \sigma(x_1)$ $\forall \sigma \in \gal(K/\Q)$,
	$i \in \Z$. This gives a contradiction.
\end{proof}

In fact, we strongly believe that the second case never occurs in general.
Although we do not have much data, all of the examples that we have found
satisfy the first case (e.g. all 4-cycles, the 5-cycles described by
Flynn-Poonen-Schaefer \cite{MR1480542}, and the 6-cycles described by
Stoll \cite{MR2465796}). If this is true in general, then it follows that
all periodic points for $\phi_c$ (with $c$ rational) in Galois number fields
with degree divisible by their period produce rational points in $C_0(N)$
since the first case produces rational trace $t = \sum\limits_{i=1}^N x_i$.
So in general, we conjecture that:

\begin{conjecture}
	Let $\phi_c(z) = z^2 + c$ with $c \in \Q$. Let $\{x_1, \ldots, x_{N}\}$ be
	an exact $N$-cycle defined over a Galois number field K, with N 
	divisible by	$d = [K:\Q ]$. Then $x_{\frac{iN}{d}+1} = \sigma(x_1)$
	for some nontrivial $\sigma \in \gal(K/\Q)$, $i \in \Z$ and therefore the
	$x_i$'s produce rational points on $C_0(N)$.
\end{conjecture}

\begin{remark}
	If this conjecture were true, one tangible consequence is that the
	6-cycle over $\Q(\sqrt{33})$ with $c = -\frac{71}{48}$ and $x_1 = -1 +
	\frac{\sqrt{33}}{12}$ found by Stoll \cite{MR2465796} is the only 6-cycle
	defined over a quadratic number field (conditionally on the weak form of
	BSD). Since the only results on rational points of $C_0(N)$ currently are
	those of Morton \cite{MR1665198} for $N = 4$, Flynn-Poonen-Schaefer
	\cite{MR1480542} for $N = 5$, and Stoll \cite{MR2465796} for $N = 6$
	(conditionally on the weak form of BSD), we have to wait for other results
	on $C_0(N)$ for other $N$ to see if this conjecture might imply results for
	other numerical cases.
\end{remark}

%%% Local Variables:
%%% TeX-master: "report"
%%% End:
