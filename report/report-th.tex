\section{Theoretical results}
\label{sec:th}

We want to extend the results from the previous section by lifting the
base field from the rational field $\Q$ to quadratic number fields,
i.e., raising the degree $D$ in the Uniform Boundedness Conjecture
statement from 1 to 2.

% I use \mathcal{U} to denote our "universe".
Let $\U$ be the collection of all rational or quadratic elements in
$\ol{\Q}$,%
\footnote{This is a bad set with very little structure. We only refer
  to it to simplify our statements.}  i.e.,
\[
\U = \set{\alpha \in \ol{\Q}: \text{$a \alpha^2 + b \alpha + c = 0$
    for some $a, b, c \in \Q$}},
\]
or equivalently,
\[
\U = \bigcup_{[K : \Q] = 2}K,
\]
where the union is the union of sets, taken over all quadratic number
fields.

We pose the following questions:
\begin{enumerate}[(i)]
\item For a given $N$, how many $c \in \U$ are there such that
  $\phi_c(z) = z^2 + c$ has a periodic point in $\U$ of exact period
  $N$?

\item For a given $N$, if the answer to (i) is infinite, then what if
  we restrict $c$ to being rational? That is, how many $c \in \Q$ are
  there such that $\phi_c(z) = z^2 + c$ has a periodic point in $\U$
  of exact period $N$?
\end{enumerate}

\begin{remark}
  In the questions above, we only asked for the number of
  $c$. However, once $c$ is known, we can immediately determine all
  the periodic points of exact period $N$ from the equation $\Phi_N(z,
  c) = 0$; and the number of such points is bound above by
  \[
  \deg_z \Phi_N(z, c) = \sum_{d|N}2^d \mu(N/d),
  \]
  which is a constant only depending on $N$. Moreover,
  in some cases we can obtain a much sharper bound.
\end{remark}

It turns out that we can fully answer Questions~(i) and (ii) for $N =
4$, and partially answer them for $N = 5$. For $N = 4$, the answer to
both questions are infinite, and we have an explicit parametrization
of $c$ for Question~(ii). For $N = 5$, the answer to Question~(i) is
finite, but at the time of writing we do not have a concrete bound. We
conjecture that the answer is in fact zero.

The results highlighted above are presented below in detail with
proof.

\subsection{The $N = 4$ case}

In 1998, Morton proved a useful parametrization of all 4-cycles for
$\phi_c(z) = z^2 + c$.

\begin{theorem} [Morton, Proposition~2 in \cite{MR1665198}]
	For $\phi_c(z) = z^2 + c$, $c$ and the points of exact period 4
	can be parametrized over $\Q$ by the following:
	\[
	\begin{gathered}
		c = \frac{1 - 4t^3 - t^6}{4t^2(t^2 - 1)}, \\
		x_1 = \frac{t^4 - t^2 + \sqrt{(t^4 - 1)(t^2 + 2t - 1)}}{2t(t^2 - 1)},
		x_2	= \frac{1 - t^2 + t \sqrt{(t^4 - 1)(t^2 + 2t - 1)}}{4t^2(t^2 - 1)}, \\
		x_3 = \frac{t^4 - t^2 - \sqrt{(t^4 - 1)(t^2 + 2t - 1)}}{2t(t^2 - 1)},
		x_4 = \frac{1 - t^2 - t \sqrt{(t^4 - 1)(t^2 + 2t - 1)}}{4t^2(t^2 - 1)}, \\
	\end{gathered}
	\]
	where $t = x_1 + x_3$.
\end{theorem}

If we allow $t$ to vary over $\Z$ or $\Q$, then it follows that:

\begin{corollary}
	There are infinitely many $c$ in $\Q$ with points in quadratic
	fields of exact period 4 for $\phi_c(z)$.
\end{corollary}

Furthermore, we can specify how many such points exist for each $c
\in \Q$ by considering the possible degrees of factors of $\Phi_4(
x,c)$. The substitution of $c = \frac{1 - 4t^3 - t^6}{4t^2(t^2 - 1)}$
into $\Phi_4(x,c)$ and polynomial factorization yields:

\begin{lemma}
	For $c$ such that $\phi$ has 4-cycles over quadratic fields,
	the degree 12 polynomial $\Phi_4(x,c)$ must factor as $\{2,2,8\}$
	(list of degrees of factors, with the 8th degree factor not
	necessarily irreducible).
\end{lemma}

We already know that $\Phi_4(x_c)$ cannot factor as $\{2,2,2,2,4\}$
by Panraksa (Theorem~2.3.5 in \cite{MR2982105}). Therefore, we
obtain the following result:

\begin{theorem}
  For each $c \in \Q$, there is at most one 4-cycle with points
	defined over quadratic fields.
\end{theorem}

%%%TO DO
Also, it is worth pointing to the distribution of $d$ (as in
$\Q(\sqrt{d})$), briefly presented in the comp section.

\subsection{The $N = 5$ case}

\subsection{General results}

While studying the N = 4 case, we found a generalization of Lemma~2.3.1
in Panraksa's paper \cite{MR2982105}.

\begin{theorem}
	Let $\phi_c(z) = z^2 + c$ with $c \in \Q$. Let $\{x_1, \ldots, x_{N}\}$ be
	an exact $N$-cycle defined over a Galois number field K, with N 
	divisible by	$d = [K:\Q ]$. Then either: \\
	\begin{itemize}
	\item $x_{\frac{iN}{d}+1} = \sigma(x_1)$ for some nontrivial $\sigma \in
	\gal(K/\Q)$, $i \in \Z$
	\item $\{x_1, \ldots, x_{N}\} \cap \{\tau(x_1), \ldots, \tau(x_{N})\} =
	\emptyset$, $\forall \tau \in \gal(K/\Q)$ nontrivial.
	\end{itemize}
\end{theorem}

\begin{proof}
	Note that $\{\tau(x_1), \ldots, \tau(x_{N})\}$ for any $\tau \in 
	\gal(K/\Q)$ is a cycle of $\phi_c$ because $c \in \Q$. Assume 
	that neither case is true. Then $x_{\frac{N}{d}+1} \neq \sigma(
	x_1)$ $\forall \sigma \in \gal(K/\Q)$ and $\{x_1, \ldots, x_{N}\} 
	\cap \{\tau(x_1), \ldots, \tau(x_{N})\} \neq \emptyset$ $\forall 
	\tau \in \gal(K/\Q)$. Then $\exists x_j \in \{x_1, \ldots, x_{N}\}
	\cap \{\tau(x_1), \ldots, \tau(x_{N})\}$ for some $\tau \in \gal(K/
	\Q)$. Applying $\phi$ an appropriate number of times allows us to 
	renumerate the cycle	such that $x_1$ is in the intersection. 
	Thus, $x_k = \tau(x_1)$ for some $1 \leq k \leq N$ and therefore $
	\tau \equiv \phi^{k-1}$ on the N-cycle (because Galois conjugation 
	commutes with $\phi$). Since $K$ is Galois, $\tau^d = Id$ and so:
	\[
		x_d = \tau^d(x_d) = \phi^{dk-d}(x_d) = x_{dk}
	\]
	Notice that the application of $\phi^{N-(d-1)}$ yields $x_1 =
	x_{dk-(d-1)} = x_{d(k-1) + 1}$. Notice that this produces a
	contradiction if $d(k-1) + 1 \not\equiv 1$ (modulo $N$): if $l$
	is the representative of $d(k-1) + 1$ modulo N, then $\{x_1, \ldots
	, x_l\}$ gives an $l$-cycle with $l < N$. Thus, it must be that
	$d(k-1) + 1 \equiv 1$ (modulo $N$). The only such $k$ are given
	by $k = \frac{lN}{d} + 1$ for some $l \in \Z$, so $x_{\frac{lN}{d}
	+ 1} = x_k = \tau(x_1)$.	However, we already have that
	$x_{\frac{iN}{d}+1} \neq \sigma(x_1)$ $\forall \sigma \in \gal(K/\Q)$,
	$i \in \Z$. This gives a contradiction.
\end{proof}

In fact, we strongly believe that the second case never occurs (e.g.
all 4-cycles, the 5-cycles described by Flynn-Poonen-Schaefer
\cite{MR1480542}, and the 6-cycles described by Stoll \cite{MR2465796}).
If this is the case, then it follows that all periodic points
for $\phi_c$ (with $c$ rational) in Galois number fields with degree
divisible by their period produce rational points in $C_0(N)$ since the
first case produces rational trace $t = \sum\limits_{i=1}^N x_i$. So in
general, we conjecture that:

\begin{conjecture}
	Let $\phi_c(z) = z^2 + c$ with $c \in \Q$. Let $\{x_1, \ldots, x_{N}\}$ be
	an exact $N$-cycle defined over a Galois number field K, with N 
	divisible by	$d = [K:\Q ]$. Then $x_{\frac{iN}{d}+1} = \sigma(x_1)$
	for some nontrivial $\sigma \in \gal(K/\Q)$, $i \in \Z$ and therefore the
	$x_i$'s produce rational points on $C_0(N)$.
\end{conjecture}

\begin{remark}
	If this conjecture were true, one tangible consequence is that the
	6-cycle over $\Q(\sqrt{33})$ with $c = -\frac{71}{48}$ and $x_1 = -1 +
	\frac{\sqrt{33}}{12}$ found by Stoll \cite{MR2465796} is the only 6-cycle
	defined over a quadratic number field (conditionally on the weak form of
	BSD). Since the only results on rational points of $C_0(N)$ currently are
	those of Morton \cite{MR1665198} for $N = 4$, Flynn-Poonen-Schaefer
	\cite{MR1480542} for $N = 5$, and Stoll \cite{MR2465796} for $N = 6$
	(conditionally on the weak form of BSD), we have to wait for other results
	on $C_0(N)$ for other $N$ to see if this conjecture might imply results for
	other numerical cases.
\end{remark}

%%% Local Variables:
%%% TeX-master: "report"
%%% End:
