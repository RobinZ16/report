\section{Theoretical results}
\label{sec:th}

We want to extend the results from the previous section by lifting the
base field from the rational field $\Q$ to quadratic number fields,
i.e., raising the degree $D$ in the Uniform Boundedness Conjecture
statement from 1 to 2.

Let $\U$ be the collection of all rational or quadratic elements in
$\ol{\Q}$,\footnote{%
  $\U$ is a bad set with very little structure. We only refer to it to
  simplify our statements.}  i.e.,
\[
\U = \set{\alpha \in \ol{\Q}: \text{$a \alpha^2 + b \alpha + c = 0$
    for some $a, b, c \in \Q$}},
\]
or equivalently,
\[
\U = \bigcup_{[K : \Q] = 2}K,
\]
where the union is the union of sets, taken over all quadratic number
fields.

We pose the following question:

\begin{question}
  \label{question}
  How many $c \in \Q$ are there such that $\phi_c(z) = z^2 + c$ has
  periodic points in $\U$ of exact period $N$?
\end{question}

\begin{remark}
  In the above question, we only asked for the number of $c$. However,
  once $c$ is known, we can immediately determine all the periodic
  points of exact period $N$ from the equation $\Phi_N(z, c) = 0$; and
  the number of such points is bound above by
  \[
  \deg_z \Phi_N(z, c) = \sum_{d|N}2^d \mu(N/d),
  \]
  which is a constant only depending on $N$. Moreover, in some cases
  we can obtain a much sharper bound; see, for instance,
  Subsection~\ref{subsec:quadratic-4}.
\end{remark}

\begin{remark}
  We are not strictly increasing $D$ to 2, since we are still
  requiring $c \in \Q$.

  The rationale is that, for $N = 4$, the problem becomes easy once we
  also allow $c$ to be quadratic; in fact, Corollary~5.11 in
  \cite{2013arXiv1308.3267B} tells us that the number of quadratic
  points on $X_1(16)$ is infinite (recall that $C_1(4)$ is
  birationally equivalent to $X_1(16)$; see
  Subsection~\ref{subsec:model-4}).

  For $N = 5$, the problem might be harder if we allow $c$ to be
  quadratic, since the curve at hand given by (\ref{eq:c0(5)}) is not
  so special as the $N = 4$ case, and it is generally much harder to
  find quadratic points on curves than to find rational ones (which is
  already highly nontrivial). We did not put much thought into this
  case, so we have little to report on this front.
\end{remark}

It turns out that we can fully answer Questions~\ref{question} for
$N = 4$, and partially answer it for $N = 5$.

For $N = 4$, the answer is infinite, since an explicit parametrization
of $c$ is available to us. The details can be found in
Subsection~\ref{subsec:quadratic-4}.

For $N = 5$, the answer is finite, but at the time of writing we do
not have a concrete bound. We conjecture that the answer is in fact
zero. The details can be found in Subsection~\ref{subsec:quadratic-5}.

Furthermore, when we were studying the proofs that led up to the
conclusion of the $N = 4$ case, we found an interesting lemma that can
be made to work, under certain limitations, for any $N$. This result
is potentially useful for investigating Question~\ref{question} in the
general case, so we document it with proof in
Subsection~\ref{subsec:general}.

\subsection{The quadratic $N = 4$ case}
\label{subsec:quadratic-4}

The study of points with exact period 4 for $\phi_c(z) = z^2 + c$ is
made easy by a parametrization of all 4-cycles that immediately
follows from Morton's work in \cite{MR1665198}.

\begin{proposition}
  For $\phi_c(z) = z^2 + c$, all combinations of $c$ and a
  corresponding 4-cycle $\set{z_1, z_2, z_3, z_4}$ of $\phi_c(z)$ can
  be parametrized over $\C$ by
  \begin{equation}
    \label{eq:4-param}
    \begin{gathered}
      c = \frac{1 - 4t^3 - t^6}{4t^2(t^2 - 1)}, \\
      z_1 = \frac{t^4 - t^2 + \sqrt{(t^4 - 1)(t^2 + 2t - 1)}}{2t(t^2 -
        1)},\,
      z_2 = \frac{1 - t^2 + t \sqrt{(t^4 - 1)(t^2 + 2t - 1)}}{4t^2(t^2
        - 1)}, \\
      z_3 = \frac{t^4 - t^2 - \sqrt{(t^4 - 1)(t^2 + 2t - 1)}}{2t(t^2 -
        1)},\,
      z_4 = \frac{1 - t^2 - t \sqrt{(t^4 - 1)(t^2 + 2t - 1)}}{4t^2(t^2
        - 1)},
    \end{gathered}
  \end{equation}
  where $t = z_1 + z_3$.
\end{proposition}

\begin{remark}
  Notice the ``parametrized over $\C$'' in the statement. This is just
  an algebraic result that works universally.
\end{remark}

Furthermore, Panraksa proved the following lemma:

\begin{lemma}[Panraksa, Theorem~1.5.1 in \cite{MR2982105}]
  \label{lem:z1+z3}
  Let $c \in \Q$, and let $\set{z_1, z_2, z_3, z_4} \subset K$ be a
  4-cycle of $\phi_c(z) = z^2 + c$, where $K$ is a quadratic number
  field. Then $t = z_1 + z_3$ is rational.
\end{lemma}

So in fact,

\begin{theorem} [Panraksa, Theorem~1.5.2 in \cite{MR2982105}]
  \label{th:all}
  All combinations of $c \in \Q$ and a corresponding 4-cycle
  $\set{z_1, z_2, z_3, z_4} \subset K$ for some quadratic number field
  $K$ are given by ranging $t$ over $\Q$ in the parametrization
  (\ref{eq:4-param}).
\end{theorem}

Notice that for any $c \in \Q$ and $z \in \U$ a periodic point of
$\phi_c$ with exact period 4, the orbit to which $z$ belongs, $\set{z,
  \phi_c(z), \phi_c^2(z), \phi_c^3(z)}$, lies in the same quadratic
number field as $z$. Therefore, the theorem above gives us complete
information for what we asked about in Question~\ref{question}. In
particular, allowing $t$ to range over $\Q$, Question~\ref{question}
is immediately answered by

\begin{corollary}
  There are infinitely many $c$ in $\Q$ such that $\phi_c(z) = z^2 +
  c$ has periodic points in $\U$ of exact period 4. Moreover, all such
  $c$ are given by
  \[
  \label{eq:c-param}
  c = \frac{1 - 4t^3 - t^6}{4t^2(t^2 - 1)},
  \]
  where $t \in \Q$.
\end{corollary}

Furthermore, we can specify how many quadratic periodic points of
exact period 4 exist for each such $c \in \Q$. Recall that Panraksa
showed that for a 4-cycle $\set{z_1, z_2, z_3, z_4} \subset \U$, $z_1
+ z_3$ must be rational (i.e., $z_1$ and $z_3$ must be of the form $a
\pm b \sqrt{d}$ for some $a, b, d \in \Q$), and similarly for $z_2$
and $z_4$. Therefore, $(z - z_1)(z - z_3) \in \Q[z]$ and $(z - z_2)(z
- z_4) \in \Q[z]$, and hence each 4-cycle in $\U$ shows up as two
quadratic factors (not necessarily irreducible) in the factorization
of $\Phi_4(z, c)$ in $\Q[z]$. We can get a hold on the number of
4-cycles (and consequently the number of 4-periodic points) by
studying the number of quadratic factors in the factorization of
$\Phi_4(z, c)$ in $\Q[z]$. This was also partially done by Panraksa,
who showed that (Theorem~2.3.5 in \cite{MR2982105}) for any $c \in
\Q$, $\Phi_4(z, c)$ cannot factor as $\set{2, 2, 2, 2, 4}$ over $\Q$
(list of degrees of factors, not necessarily irreducible). Therefore,
we obtain the following result:

\begin{theorem}
  For each $c \in \Q$, $\phi_c(z) = z^2 + c$ has at most one 4-cycle
  with points in $\U$, i.e., with points defined over a quadratic
  number field. Consequently, for each $c \in \Q$, $\phi_c(z) = z^2 +
  c$ has at most four periodic points in $\U$ of exact period 4.
\end{theorem}

Subsequently, we sought to understand precisely which quadratic fields
could contain the 4-cycles. Due to the precise parametrization
(\ref{eq:4-param}), this problem reduces to finding rational numbers
of the form $d = (t^4 - 1)(t^2 + 2t - 1)$, with $t \in \Q$, up to
rational squares. Further discussion of this problem can be found in
Subsection.
%% TODO add subsection reference

\subsection{The quadratic $N = 5$ case}
\label{subsec:quadratic-5}

We aim to prove Theorem~\ref{th:n=5-finite}, i.e., there are finitely
many $c \in \Q$ such that $\phi_c(z) = z^2 + c$ has periodic points in
$\U$ of exact period 5.

Let us recall from (\ref{eq:c0(5)}) and (\ref{eq:c-in-xy}) that $y^2 =
f(x)$ and $c = (P_0(x) + P_1(x) y)/h(x)$, so eliminating $y$ we have
\[
(c \cdot h(x) - P_0(x))^2 = P_1(x)^2 y^2 = P_1(x)^2 f(x),
\]
i.e., $x$ satisfies the polynomial equation
\[
[(c \cdot h - P_0)^2 - P_1^2 f](x) = 0.
\]
Since we are only looking for $x \in \U$,\footnote{%
  Recall that we are looking for $z \in \U$ and $c \in \Q$. On $C_0(5)
  = C_1(5) / \tup{\sigma}$, whose points correspond to pairs $(\O,
  \phi_c)$, each orbit $\O$ is represented by its trace $\tau = z +
  \phi_c(z) + \cdots + \phi_c^4(z)$, which is in the same quadratic
  extension as $z$. The series of change of coordinates from $(\tau,
  c)$ to $(x, y)$ involves only arithmetic operations, so the
  resulting $x$ and $y$ are still in the same quadratic extension as
  $\tau$. In particular, we have $x \in \U$. See \cite{MR1480542} for
  details.}
there exist some rational coefficients $a$ and $b$ such that $x$
solves the quadratic equation
\[
x^2 + a x + b = 0.
\]
We have obtained two polynomial equations in $x$, so we may divide
$[(c \cdot h - P_0)^2 - P_1^2 f]$ by $x^2 + ax + b$ using long
division, and the remainder must be zero. Since $x^2 + ax + b$ is
quadratic in $x$, the remainder is linear, in the form
\[
\l_1(a, b, c) x + \l_0(a, b, c),
\]
where $\l_1$ and $\l_0$ are polynomials in $a$, $b$, and $c$ given by
\[
\begin{aligned}
  \l_1&(a, b, c) = 16 (a - 3) (2b - a(a-3)) c^2 -
  4(a^{5} - 10 a^{4} - 4 a^{3} b + 46 a^{3} + 30 a^{2} b + \\
  & 3 a b^{2} - 104 a^{2} - 92 a b - 10 b^{2} + 95 a + 104 b - 24) c -
  (8 a^{5} - 74 a^{4} - 32 a^{3} b + \\
  & 271 a^{3} + 222 a^{2} b + 24 a b^{2} - 452 a^{2} - 542 a b - 74
  b^{2} + 325 a + 452 b - 110),
\end{aligned}
\]
and
\[
\begin{aligned}
  \l_0&(a, b, c) = 16 b (b - (a-3)^2) c^2 -
  4(a^{4} b - 10 a^{3} b - 3 a^{2} b^{2} + 46 a^{2} b + \\
  & 20 a b^{2} + b^{3} - 104 a b - 46 b^{2} + 95 b - 9) c -
  (8 a^{4} b - 74 a^{3} b - 24 a^{2} b^{2} + \\
  & 271 a^{2} b + 148 a b^{2} + 8 b^{3} - 452 a b - 271 b^{2} + 325 b
  - 64)
\end{aligned}
\]
Note that $a$, $b$, and $c$ are all rational, so $\l_1(a, b, c)$ and
$\l_0(a, b, c)$ are rational, and hence from
\[
\l_1(a, b, c) x + \l_0(a, b, c) = 0
\]
we conclude that either $x$ is rational, or both $\l_1$ and $\l_0$ are
zero.\footnote{%
  The same argument follows through if $c$ is allowed to take value
  in $\U$ but we further stipulate that $c$ and $x$ are not in the
  same quadratic extension, i.e., $\Q(c, x)$ has degree 4 over
  $\Q$. This is because $\l_1$ and $\l_0$ are in $\Q(c)$, so if $\l_1$
  and $\l_0$ are nonzero, then from $x = -\l_0/\l_1$ we conclude that
  $x \in \Q(c) \cap \Q(x) = \Q$, which contradicts the assumption that
  $x \in \U \setminus \Q$.

  However, this method fails when $c$ is allowed to be in the same
  quardratic extension as $x$, in which case any combination of $a, b
  \in Q$ and $c \in \U$ trivially produces an $x \in \U$ unless $\l_1
  = 0$ and $\l_0 \ne 0$. This is one complication of allowing $c$ to
  be quadratic.
}
Thanks to \cite{MR1480542}, we already fully understand the case
where $x$ is rational (in which case there are no corresponding
periodic points even in $\U$ --- they are either at infinity or
defined only in a quintic extension of $\Q$). Therefore, we only
consider $x \in \U \setminus \Q$, and hence
\[
\l_1(a, b, c) = \l_0(a, b, c) = 0.
\]

Notice that $c$ is a common root to $\l_1$ and $\l_0$, so by
Proposition~\ref{prop:res}, we have
\[
\res_c(\l_1(a, b, c), \l_2(a, b, c)) = 0.
\]

Now, depending on the degrees of $\l_1$ and $\l_2$ in $c$, we have
three cases.

\begin{case}
  The leading coefficient of $\l_1(a, b, c)$, considered as a
  polynomial in $c$, vanishes. This happens when
  \[
  16(a - 3)(2b - a(a-3)) = 0,
  \]
  i.e., either $a = 3$, or $b = a(a-3)/2$, or both.

  If $a = 3$, plugging $a = 3$ into $\l_1 = \l_2 = 0$ we get two
  polynomial equations in $b$ and $c$. Taking the resultant of the two
  with respect to $c$, we get $- 96b^7 - 1264b^5 - 4256b^5 + 32b^4 +
  13248b^3 + 37632b^2 + 92160b = 0$, which has only one rational root
  $b = 0$. However, $\l_1(3, 0, c) \equiv -64 \ne 0$, so there is no
  rational solution in this case.

  If $b = a(a-3)/2$, plugging this into $\l_1 = \l_2 = 0$ we get two
  polynomials equations in $a$ and $c$. Again by taking the resultant
  we can show that there is no rational solution in this case. See the
  attached Mathematica notebook \texttt{mma/abc.nb} for details.
\end{case}

\begin{case}
  The leading coefficient of $\l_0(a, b, c)$, considered as a
  polynomial in $c$, vanishes. This happens when
  \[
  16b(b - (a-3)^2) = 0,
  \]
  i.e., either $b = 0$ or $b = (a - 3)^2$. Similar to Case~1, we can
  easily show that there is no rational solution in this case. See
  \texttt{mma/abc.nb} for details.
\end{case}

\begin{case}
  Both of the leading coefficients of $\l_1(a, b, c)$ and $\l_0(a, b,
  c)$ (considered as polynomials in $c$), are non-vanishing, i.e.,
  both $\l_1(a, b, c)$ and $\l_0(a, b, c)$ are quadratic in $c$.

  In this case, by Proposition~\ref{prop:res}, $\res_c(\l_1(a, b, c),
  \l_0(a, b, c))$ is given by a polynomial in the six coefficients
  (three of $\l_1$ and three of $\l_0$). Mathematica is capable of
  computing this polynomial (see \texttt{mma/abc.nb}), and when we
  plug in the six coefficients, we get
  \[
  \label{eq:pab}
  \res_c(\l_1(a, b, c), \l_0(a, b, c)) = P(a, b)
  \]
  for some polynomial $P(a, b) \in \Z[a, b]$. This polynomial is far
  too complicated to write down.\footnote{%
    Expanding $P(a, b)$ here would likely take half a page. Anyway,
    as always the full details can be found in \texttt{mma/abc.nb}.}
  It has degree $8$ in $a$ and degree $9$ in $b$. So the problem now
  reduces to finding rational points on the curve $C_P$ defined by
  $P(a, b) = 0$ --- for each rational point $(a, b)$ on $C_P$, there
  are at most two rational $c$ solving $\l_1(a, b, c) = 0$ and
  $\l_0(a, b, c) = 0$.

  Using Sage (see \texttt{sage/pab-genus.sage}), we were able to
  calculate the genus of $C_P$ (more precisely, the genus of its
  normalization). It turns out that the genus
  \[
  g(C_P) = 11 > 2,
  \]
  so $C_P$ is a high genus curve; hence by Faltings' theorem
  (Theorem~\ref{th:faltings}), the number of rational points on $C_P$
  is finite. Consequently, the number of $c$ that satisfy our
  conditions is finite. This completes the proof of
  Theorem~\ref{th:n=5-finite}.
\end{case}

\subsection{General results}
\label{subsec:general}

While studying the quadratic N = 4 case, we tried to mimic results
to the quadratic N = 6 case. In particular, the following result by
Panraksa was crucial in proving Lemma~\ref{lem:z1+z3}, which was later
used to establish a complete understanding of the $N = 4$ case:

\begin{theorem} [Panraksa, Lemma~2.3.1 in \cite{MR2982105}]
  Let $c \in \Q$, and let $\set{z_1, z_2, z_3, z_4} \subset K$ be a
  4-cycle of $\phi_c(z) = z^2 + c$, where $K$ is a quadratic number
  field. Then exactly one of the following holds:
  \begin{enumerate}[(i)]
  \item $z_3 = \ol{z_1}$;\footnote{%
      $\ol{z_1}$ denotes the Galois conjugate of $z_1$ in $K$, which
      is unambiguous since $K$ is quadratic.}

  \item $\{z_1, z_2, z_3, z_4\} \cap \{\ol{z_1}, \ol{z_2}, \ol{z_3},
    \ol{z_4}\} =\es.$
  \end{enumerate}
\end{theorem}

Panraksa then proves that the second case $\{z_1, z_2, z_3, z_4\} \cap
\{\ol{z_1}, \ol{z_2}, \ol{z_3}, \ol{z_4}\} = \es$ is impossible, from
which Lemma~\ref{lem:z1+z3} immediately follows. We managed to prove a
much more general version of his lemma:

\newcommand{\nd}{\frac{N}{d}}

\begin{theorem}
  Let $c \in \Q$. Let $K$ be a Galois extension of $\Q$ with degree $d
  = [K : \Q]$. Let $N$ be a multiple of $d$, and $\set{z_0, \dots,
    z_{N-1}} \subset K$ be an exact $N$-cycle of $\phi_c(z) = z^2 +
  c$. Then exactly one of the following holds:
  \begin{enumerate}[(i)]
  \item $z_{i\nd} = \t(z_0)$ for some $0 \le i \le d-1$ and some
    nontrivial $\t \in \gal(K/\Q)$;

  \item $\set{z_0, \dots, z_{N-1}} \cap \set{\t(z_0), \dots, \t(z_{N-1})} =
    \es$ for all nontrivial $\t \in \gal(K/\Q)$.
  \end{enumerate}
\end{theorem}

%%% TODO
%
% Looking at the proof, K being Galois and d dividing N can be relaxed
% to give something more general. For instance, if we do not assume $d
% | N$, then we only need to replace $N/d$ by $N/gcd(N,d)$.
%
% (Note that those two conditions are arbitrary to begin with. They
% were introduced as makeshift fixes for holes in the original
% "proof", and were never carefully considered later.)
%
% I will address this when I have time.
\begin{proof}
  For notational convenience, we let $z_{j+N} = z_j$ for all $j \in
  \Z$, i.e., if $j = q \cdot N + r$ with $0 \le r \le N-1$, then $z_j
  = z_r = \phi_c^r(z_0)$.

  First note that (i) and (ii) cannot be simultaneously true. In fact,
  if (i) is true, i.e., $z_{i\nd} = \t(z_0)$ for some $i$ and
  nontrivial $\t \in \gal(K/\Q)$, then $z_{i\nd} \in \set{z_0, \dots,
    z_{N-1}} \cap \set{\t(z_0), \dots, \t(z_{N-1})}$, so (ii) is
  false.

  If (ii) is true, then we are done. Otherwise, (ii) is false, so we
  have some nontrivial $\t \in \gal(K/\Q)$ such that $z_k = \t(z_j)$
  for some $j$ and $k$ satisfying $0 \le j, k \le N-1$. Note that $\t$
  commutes with $\phi_c$ (as $\phi_c$ is defined over $\Q$ and $\t$ is
  a field automorphism fixing the base field $\Q$), so we may assume
  $j = 0$; otherwise, note that $\t(z_0) = \t(z_N) =
  \t(\phi_c^{N-j}(z_j)) = \phi_c^{N-j}(\t(z_j)) = \phi_c^{N-j}(z_k) =
  z_{N+k-j}$, so we may set $j$ to 0 and $k$ to the remainder of
  $N+k-j$ modulo $N$. Assuming $j = 0$, we have $\t(z_0) = z_k$; this,
  together with the fact that $\t$ commutes with $\phi_c$, implies
  that $\t = \phi_c^k$ on the whole cycle $\set{z_0, \dots, z_{N-1}}$.

  Now recall that $K$ is Galois, so the order of the Galois group
  $\gal(K/\Q)$ is exactly $[K : \Q] = d$. Therefore, by Lagrange's
  theorem, $\t^d = Id$, and hence
  \[
  z_0 = \t^d(z_0) = (\phi_c^k)^d(z_0) = z_{kd}.
  \]
  Let $r$ be the remainder of $kd$ modulo $N$. If $r$ is nonzero, then
  $\set{z_0, \dots, z_{r-1}}$ forms a cycle of $\phi_c$ with length $r
  < N$, violating the requirement that $\set{z_0, \dots, z_{N-1}}$ is
  an exact $N$-cycle. Therefore, $r = 0$, i.e., $N$ divides
  $kd$. Since $d$ divides $N$, we have $k = i \cdot \nd$ for some $i$,
  so $\t(z_0) = z_k = z_{i\nd}$. Then (i) is true, and we are done.
\end{proof}

In fact, we strongly believe that the second case never occurs in
general. Although the supply of examples is limited, all of the
examples that we currently know satisfy the first case (all 4-cycles,
the 5-cycles described by Flynn, Poonen, and Schaefer
\cite{MR1480542}, and the 6-cycles described by Stoll
\cite{MR2465796}). If this is true in general, then it follows that
all periodic points for $\phi_c$ (with $c$ rational) in Galois
extensions to $\Q$ fields with degree dividing their period
correspond to rational points in $C_0(N)$ since the first case
produces rational trace $t = \sum_{i=1}^N z_i$.  So in general,
we conjecture that:

\begin{conjecture}
  Let $\phi_c(z) = z^2 + c$ with $c \in \Q$. Let $\{z_1, \dots, z_N\}$
  be an exact $N$-cycle defined over a Galois number field K, with N
  divisible by $d = [K:\Q ]$. Then $z_{\frac{iN}{d}+1} = \s(z_1)$ for
  some nontrivial $\s \in \gal(K/\Q)$, $i \in \Z$ and therefore the
  $z_i$'s produce rational points on $C_0(N)$.
\end{conjecture}

\begin{remark}
  If this conjecture were true, one tangible consequence is that the
  6-cycle over $\Q(\sqrt{33})$ with $c = -\frac{71}{48}$ and $z_1 = -1
  + \frac{\sqrt{33}}{12}$ found by Stoll \cite{MR2465796} is the only
  6-cycle defined over a quadratic number field (conditionally on the
  weak form of BSD). Since the only results on rational points of
  $C_0(N)$ currently are those of Morton \cite{MR1665198} for $N = 4$,
  Flynn-Poonen-Schaefer \cite{MR1480542} for $N = 5$, and Stoll
  \cite{MR2465796} for $N = 6$ (conditionally on the weak form of
  BSD), we have to wait for other results on $C_0(N)$ for other $N$ to
  see if this conjecture might imply results for other numerical
  cases.
\end{remark}

%%% Local Variables:
%%% TeX-master: "report"
%%% End:
