\section{Theoretical results}

We want to extend the results from the previous section by lifting the
base field from the rational field $\Q$ to quadratic number fields,
i.e., raising the degree $D$ in the Uniform Boundedness Conjecture
statement from 1 to 2.

% I use \mathcal{U} to denote our "universe".
Let $\U$ be the collection of all rational or quadratic elements in
$\ol{\Q}$,%
\footnote{This is a bad set with very little structure. We only refer
  to it to simplify our statements.}  i.e.,
\[
\U = \set{\alpha \in \ol{\Q}: \text{$a \alpha^2 + b \alpha + c = 0$
    for some $a, b, c \in \Q$}},
\]
or equivalently,
\[
\U = \bigcup_{[K : \Q] = 2}K,
\]
where the union is the union of sets, taken over all quadratic number
fields.

We pose the following questions:
\begin{enumerate}[(i)]
\item For a given $N$, how many $c \in \U$ are there such that
  $\phi_c(z) = z^2 + c$ has a periodic point in $\U$ of exact period
  $N$?

\item For a given $N$, if the answer to (i) is infinite, then what if
  we restrict $c$ to being rational? That is, how many $c \in \Q$ are
  there such that $\phi_c(z) = z^2 + c$ has a periodic point in $\U$
  of exact period $N$?
\end{enumerate}

\begin{remark}
  In the questions above, we only asked for the number of
  $c$. However, once $c$ is known, we can immediately determine all
  the periodic points of exact period $N$ from the equation $\Phi_N(z,
  c) = 0$; and the number of such points is bound above by
  \[
  \deg_z \Phi_N(z, c) = \sum_{d|N}2^d \mu(N/d),
  \]
  which is a constant only depending on $N$. Moreover,
  in some cases we can obtain a much sharper bound.
\end{remark}

It turns out that we can fully answer Questions~(i) and (ii) for $N =
4$, and partially answer them for $N = 5$. For $N = 4$, the answer to
both questions are infinite, and we have an explicit parametrization
of $c$ for Question~(ii). For $N = 5$, the answer to Question~(i) is
finite, but at the time of writing we do not have a concrete bound. We
conjecture that the answer is in fact zero.

The results highlighted above are presented below in detail with
proof.

\subsection{The $N = 4$ case}

%%% TODO

\textbf{This is Robin's land!}

Do remember to mention quadratic points on $X_1(16)$, which
immediately answers Question~(i) without any further computation.

Also, it is worth pointing to the distribution of $d$ (as in
$\Q(\sqrt{d})$), briefly presented in the comp section.

\subsection{The $N = 5$ case}



%%% Local Variables:
%%% TeX-master: "report"
%%% End:
