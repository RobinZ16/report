\section{Theoretical results}
\label{sec:th}

We want to extend the results from the previous section by lifting the
base field from the rational field $\Q$ to quadratic extensions of
$\Q$, i.e., raising the degree $D$ in the Uniform Boundedness
Conjecture statement from 1 to 2.

Let $\U$ be the collection of all rational or quadratic elements in
$\ol{\Q}$,\footnote{%
  $\U$ is a bad set with very little structure. We only refer to it to
  simplify our statements.}  i.e.,
\[
\U = \set{\alpha \in \ol{\Q}: \text{$a \alpha^2 + b \alpha + c = 0$
    for some $a, b, c \in \Q$}},
\]
or equivalently,
\[
\U = \bigcup_{[K : \Q] = 2}K,
\]
where the union is the union of sets, taken over all quadratic
extensions of $\Q$.

We pose the following question:

\begin{question}
  \label{question}
  How many $c \in \Q$ are there such that $\phi_c(z) = z^2 + c$ has
  periodic points in $\U$ of exact period $N$?
\end{question}

\begin{remark}
  In the above question, we only asked for the number of $c$. However,
  once $c$ is known, we can immediately determine all the periodic
  points of exact period $N$ from the equation $\Phi_N(z, c) = 0$; and
  the number of such points is bounded above by
  \[
  \deg_z \Phi_N(z, c) = \sum_{d|N}2^d \mu(N/d),
  \]
  which is a constant only depending on $N$. Moreover, in some cases
  we can obtain a much sharper bound; see, for instance,
  Subsection~\ref{subsec:quadratic-4}.
\end{remark}

\begin{remark}
  We are not strictly increasing $D$ to 2, as we are still requiring
  $c \in \Q$.

  The rationale is that, for $N = 4$, the problem becomes easy once we
  also allow $c$ to be quadratic; in fact, Corollary~5.11 in
  \cite{2013arXiv1308.3267B} tells us that the number of quadratic
  points on $X_1(16)$ is infinite (recall that $C_1(4)$ is
  birationally equivalent to $X_1(16)$; see
  Subsection~\ref{subsec:model-4}).

  For $N = 5$, the problem might be harder if we allow $c$ to be
  quadratic, since the curve at hand given by (\ref{eq:c0(5)}) is not
  so special as the $N = 4$ case, and it is generally much harder to
  find quadratic points on curves than to find rational ones (which is
  already highly nontrivial). We did not put much thought into this
  case, so we have little to report on this front.
\end{remark}

It turns out that we can fully answer Questions~\ref{question} for
$N = 4$, and partially answer it for $N = 5$.

For $N = 4$, the answer is infinite, since an explicit parametrization
of $c$ is available to us. The details can be found in
Subsection~\ref{subsec:quadratic-4}.

For $N = 5$, the answer is finite, but at the time of writing we do
not have a concrete bound. We conjecture that the answer is in fact
zero. The details can be found in Subsection~\ref{subsec:quadratic-5}.

Furthermore, when we were studying the proofs that led up to the
conclusion of the $N = 4$ case, we found an interesting lemma that can
be adapted to, under certain limitations, any $N$. This result is
potentially useful for investigating Question~\ref{question} in the
general case, so we document it with proof in
Subsection~\ref{subsec:general}, and propose a more powerful version
of this result as a conjecture.

\subsection{The quadratic $N = 4$ case}
\label{subsec:quadratic-4}

The study of points with exact period 4 for $\phi_c(z) = z^2 + c$ is
made easy by a parametrization of all 4-cycles that immediately
follows from Morton's work in \cite{MR1665198}.

\begin{proposition}
  For $\phi_c(z) = z^2 + c$, all combinations of $c$ and a
  corresponding 4-cycle $\set{z_1, z_2, z_3, z_4}$ of $\phi_c(z)$ can
  be parametrized over $\C$ by
  \begin{equation}
    \label{eq:4-param}
    \begin{gathered}
      c = \frac{1 - 4t^3 - t^6}{4t^2(t^2 - 1)}, \\
      z_1 = \frac{t^4 - t^2 + \sqrt{(t^4 - 1)(t^2 + 2t - 1)}}{2t(t^2 -
        1)},\,
      z_2 = \frac{1 - t^2 + t \sqrt{(t^4 - 1)(t^2 + 2t - 1)}}{4t^2(t^2
        - 1)}, \\
      z_3 = \frac{t^4 - t^2 - \sqrt{(t^4 - 1)(t^2 + 2t - 1)}}{2t(t^2 -
        1)},\,
      z_4 = \frac{1 - t^2 - t \sqrt{(t^4 - 1)(t^2 + 2t - 1)}}{4t^2(t^2
        - 1)},
    \end{gathered}
  \end{equation}
  where $t = z_1 + z_3$.
\end{proposition}

\begin{remark}
  Notice the ``parametrized over $\C$'' in the statement. This is just
  an algebraic result that works universally.
\end{remark}

Furthermore, Panraksa \cite{MR2982105} proved the following lemma:

\begin{lemma}[Panraksa, Theorem~1.5.1 in \cite{MR2982105}]
  \label{lem:z1+z3}
  Let $c \in \Q$, and let $\set{z_1, z_2, z_3, z_4} \subset K$ be a
  4-cycle of $\phi_c(z) = z^2 + c$, where $K$ is a quadratic extension
  to $\Q$. Then $t = z_1 + z_3$ is rational.
\end{lemma}

So in fact,

\begin{theorem} [Panraksa, Theorem~1.5.2 in \cite{MR2982105}]
  \label{th:all}
  All combinations $(c, \set{z_1, z_2, z_3, z_4})$, where $c \in \Q$
  and $\set{z_1, z_2, z_3, z_4}$ is a corresponding 4-cycle of
  $\phi_c(z) = z^2 + c$, are given by ranging $t$ over $\Q$ in the
  parametrization (\ref{eq:4-param}).
\end{theorem}

Notice that for any $c \in \Q$ and $z \in \U$ a periodic point of
$\phi_c$ with exact period 4, the orbit to which $z$ belongs, $\set{z,
  \phi_c(z), \phi_c^2(z), \phi_c^3(z)}$, lies in the same quadratic
extension to $\Q$ as $z$. Therefore, the theorem above gives us
complete information for what we asked about in
Question~\ref{question}. In particular, allowing $t$ to range over
$\Q$, Question~\ref{question} is immediately answered by

\begin{corollary}
  There are infinitely many $c \in \Q$ such that $\phi_c(z) = z^2 + c$
  has periodic points in $\U$ of exact period 4. Moreover, all such
  $c$ are given by
  \[
  \label{eq:c-param}
  c = \frac{1 - 4t^3 - t^6}{4t^2(t^2 - 1)},
  \]
  where $t \in \Q$.
\end{corollary}

Furthermore, we can specify how many quadratic periodic points of
exact period 4 exist for each such $c \in \Q$. Recall that Panraksa
showed that for a 4-cycle $\set{z_1, z_2, z_3, z_4} \subset \U$, $z_1
+ z_3$ must be rational (i.e., $z_1$ and $z_3$ must be of the form $a
\pm b \sqrt{d}$ for some $a, b, d \in \Q$), and similarly for $z_2$
and $z_4$. Therefore, $(z - z_1)(z - z_3) \in \Q[z]$ and $(z - z_2)(z
- z_4) \in \Q[z]$, and hence each 4-cycle in $\U$ shows up as two
quadratic factors (not necessarily irreducible) in the factorization
of $\Phi_4(z, c)$ in $\Q[z]$. We can get a hold on the number of
4-cycles (and consequently the number of 4-periodic points) by
studying the number of quadratic factors in the factorization of
$\Phi_4(z, c)$ in $\Q[z]$. This was also partially done by Panraksa,
who showed that (Theorem~2.3.5 in \cite{MR2982105}) for any $c \in
\Q$, $\Phi_4(z, c)$ cannot factor as $\set{2, 2, 2, 2, 4}$ over $\Q$
(list of degrees of factors, not necessarily irreducible). Therefore,
we obtain the following result:

\begin{theorem}
  For each $c \in \Q$, $\phi_c(z) = z^2 + c$ has at most one 4-cycle
  with points in $\U$, i.e., with points defined over a quadratic
  extension to $\Q$. Consequently, for each $c \in \Q$, $\phi_c(z) =
  z^2 + c$ has at most four periodic points in $\U$ of exact period 4.
\end{theorem}

Subsequently, we sought to understand precisely which quadratic fields
could contain the 4-cycles. Due to the precise parametrization
(\ref{eq:4-param}), this problem reduces to finding rational numbers
of the form $d = (t^4 - 1)(t^2 + 2t - 1)$, with $t \in \Q$, up to
rational squares. Further discussion of this problem can be found in
Subsection~\ref{subsec:d-distribution}.

\subsection{The quadratic $N = 5$ case}
\label{subsec:quadratic-5}

We aim to prove Theorem~\ref{th:n=5-finite}, i.e., there are finitely
many $c \in \Q$ such that $\phi_c(z) = z^2 + c$ has periodic points in
$\U$ of exact period 5.

Let us recall from (\ref{eq:c0(5)}) and (\ref{eq:c-in-xy}) that $y^2 =
f(x)$ and $c = (P_0(x) + P_1(x) y)/h(x)$, so eliminating $y$ we have
\[
(c \cdot h(x) - P_0(x))^2 = P_1(x)^2 y^2 = P_1(x)^2 f(x),
\]
i.e., $x$ satisfies the polynomial equation
\[
[(c \cdot h - P_0)^2 - P_1^2 f](x) = 0.
\]
Since we are only looking for $x \in \U$,\footnote{%
  Recall that we are looking for $z \in \U$ and $c \in \Q$. On $C_0(5)
  = C_1(5) / \tup{\sigma}$, whose points correspond to pairs $(\O,
  \phi_c)$, each orbit $\O$ is represented by its trace $\tau = z +
  \phi_c(z) + \cdots + \phi_c^4(z)$, which is in the same quadratic
  extension as $z$. The series of change of coordinates from $(\tau,
  c)$ to $(x, y)$ involves only arithmetic operations, so the
  resulting $x$ and $y$ are still in the same quadratic extension as
  $\tau$. In particular, we have $x \in \U$. See \cite{MR1480542} for
  details.}
there exist some rational coefficients $a$ and $b$ such that $x$
solves the quadratic equation
\[
x^2 + a x + b = 0.
\]
We have obtained two polynomial equations in $x$, so we may divide $(c
\cdot h - P_0)^2 - P_1^2 f$ by $x^2 + ax + b$ using long division, and
the remainder must be zero. Since $x^2 + ax + b$ is quadratic in $x$,
the remainder is linear, in the form
\[
\l_1(a, b, c) x + \l_0(a, b, c),
\]
where $\l_1$ and $\l_0$ are polynomials in $a$, $b$, and $c$ given by
\[
\begin{aligned}
  \l_1&(a, b, c) = 16 (a - 3) (2b - a(a-3)) c^2 -
  4(a^{5} - 10 a^{4} - 4 a^{3} b + 46 a^{3} + 30 a^{2} b + \\
  & 3 a b^{2} - 104 a^{2} - 92 a b - 10 b^{2} + 95 a + 104 b - 24) c -
  (8 a^{5} - 74 a^{4} - 32 a^{3} b + \\
  & 271 a^{3} + 222 a^{2} b + 24 a b^{2} - 452 a^{2} - 542 a b - 74
  b^{2} + 325 a + 452 b - 110),
\end{aligned}
\]
and
\[
\begin{aligned}
  \l_0&(a, b, c) = 16 b (b - (a-3)^2) c^2 -
  4(a^{4} b - 10 a^{3} b - 3 a^{2} b^{2} + 46 a^{2} b + \\
  & 20 a b^{2} + b^{3} - 104 a b - 46 b^{2} + 95 b - 9) c -
  (8 a^{4} b - 74 a^{3} b - 24 a^{2} b^{2} + \\
  & 271 a^{2} b + 148 a b^{2} + 8 b^{3} - 452 a b - 271 b^{2} + 325 b
  - 64)
\end{aligned}
\]
Note that $a$, $b$, and $c$ are all rational, so $\l_1(a, b, c)$ and
$\l_0(a, b, c)$ are rational, and hence from
\[
\l_1(a, b, c) x + \l_0(a, b, c) = 0
\]
we conclude that either $x$ is rational, or both $\l_1$ and $\l_0$ are
zero.\footnote{%
  In fact, this step of deducing either $x \in \Q$ or $\l_1 = \l_0 =
  0$ works as long as $\Q(c) \cap \Q(x) = \Q$. This is because $\l_1$
  and $\l_0$ are in $\Q(c)$ (as $a$ and $b$ are rational), so if
  $\l_1$ and $\l_0$ are nonzero, then from $x = -\l_0/\l_1$ we have $x
  \in \Q(c)$, and hence $x \in \Q(c) \cap \Q(x) = \Q$.}
Thanks to \cite{MR1480542}, we already fully understand the case
where $x$ is rational (in which case there are no corresponding
periodic points even in $\U$ --- they are either at infinity or
defined only in a quintic extension of $\Q$). Therefore, we only
consider $x \in \U \setminus \Q$, where we have
\[
\l_1(a, b, c) = \l_0(a, b, c) = 0.
\]

Notice that $c$ is a common root to $\l_1$ and $\l_0$, so by
Proposition~\ref{prop:res}, we have
\[
\res_c(\l_1(a, b, c), \l_2(a, b, c)) = 0.
\]

Now, depending on the degrees of $\l_1$ and $\l_2$ in $c$, we have
three cases.

\begin{case}
  The leading coefficient of $\l_1(a, b, c)$ (considered as a
  polynomial in $c$) vanishes. This happens when
  \[
  16(a - 3)(2b - a(a-3)) = 0,
  \]
  i.e., either $a = 3$, or $b = a(a-3)/2$, or both.

  If $a = 3$, substituting $a = 3$ into $\l_1 = \l_2 = 0$ we get two
  polynomial equations in $b$ and $c$. Taking the resultant of the two
  with respect to $c$, we get $- 96b^7 - 1264b^5 - 4256b^5 + 32b^4 +
  13248b^3 + 37632b^2 + 92160b = 0$, which has only one rational root
  $b = 0$. However, $\l_1(3, 0, c) \equiv -64 \ne 0$, so there are no
  rational solutions in this case.

  If $b = a(a-3)/2$, substituting this into $\l_1 = \l_2 = 0$ we get
  two polynomials equations in $a$ and $c$. Again by taking the
  resultant we can show that there are no rational solutions in this
  case. See the attached Mathematica notebook \texttt{mma/abc.nb} for
  details.
\end{case}

\begin{case}
  The leading coefficient of $\l_0(a, b, c)$ (considered as a
  polynomial in $c$) vanishes. This happens when
  \[
  16b(b - (a-3)^2) = 0,
  \]
  i.e., either $b = 0$ or $b = (a - 3)^2$. Similar to Case~1, we can
  easily show that there are no rational solutions in this case. See
  \texttt{mma/abc.nb} for details.
\end{case}

\begin{case}
  Both of the leading coefficients of $\l_1(a, b, c)$ and $\l_0(a, b,
  c)$ (considered as polynomials in $c$) are non-vanishing, i.e., both
  $\l_1(a, b, c)$ and $\l_0(a, b, c)$ are quadratic in $c$.

  In this case, by Proposition~\ref{prop:res}, $\res_c(\l_1(a, b, c),
  \l_0(a, b, c))$ is given by a polynomial in the six coefficients
  (three of $\l_1$ and three of $\l_0$). Mathematica is capable of
  computing this polynomial (see \texttt{mma/abc.nb}), and when we
  plug in the six coefficients, we get
  \[
  \label{eq:pab}
  \res_c(\l_1(a, b, c), \l_0(a, b, c)) = P(a, b)
  \]
  for some polynomial $P(a, b) \in \Z[a, b]$. This polynomial is far
  too complicated to write down.\footnote{%
    Expanding $P(a, b)$ here would likely take half a
    page. Nevertheless, the full details can be found in
    \texttt{mma/abc.nb} as always.}
  It has degree $8$ in $a$ and degree $9$ in $b$. So the problem now
  reduces to finding rational points on the curve $C_P$ defined by
  $P(a, b) = 0$ --- for each rational point $(a, b)$ on $C_P$, there
  are at most two rational $c$ solving $\l_1(a, b, c) = 0$ and
  $\l_0(a, b, c) = 0$.

  Using Sage (see \texttt{sage/pab-genus.sage}), we were able to
  calculate the genus of $C_P$ (more precisely, the genus of its
  normalization). It turns out that the genus
  \[
  g(C_P) = 11 \ge 2,
  \]
  so $C_P$ is a high genus curve; hence by Faltings' theorem
  (Theorem~\ref{th:faltings}), the number of rational points on $C_P$
  is finite. Consequently, the number of $c$ that satisfy our
  conditions is finite. This completes the proof of
  Theorem~\ref{th:n=5-finite}.
\end{case}

Having established Theorem~\ref{th:n=5-finite}, which is a finiteness
theorem, the next natural thing to do is to find the actual finite
number. We first tried to find such $c$ computationally; the efforts
are documented in Subsection~\ref{subsec:seek-quadratic},
``Desperately seeking quadratic periodic points''. The computational
results strengthened our belief that $c \in \Q$ such that $\phi_c$ has
periodic points in $\U$ of exact period 5 simply do not exist, hence
Conjecture~\ref{cj:n=5-zero}.

To actually prove the conjecture, one promising direction is to study
the rational points on $C_P$, which, once fully understood, will give
us complete information on all the values of $c$ that we want.

Computationally (see Subsection~\ref{subsec:pab-ratpoint}), we found
five rational points on $C_P$: $(a, b)$ = $(3, 0)$, $(0, 0)$, $(4,
1/3)$, $(1, 8/3)$, and $(6, 9)$. All five points have very small
heights; then there is a vacuum --- at least up to height 10000 ---
without any rational points, so it is conjectured that the five points
are all the affine rational points on $C_P$. None of these points
produce $c$ that we want.

However, proving that there are only five affine rational points on
$C_P$ is very hard. See \cite{MR1956273} (2002) and \cite{MR2780629}
(2011) for surveys of contemporary methods of bounding the number of
rational points on algebraic curves. These methods usually work badly
for curves of genus greater than 2, if at all.

For our particular curve at hand, at the time of writing we are trying
to apply Chabauty and Coleman's method to get a rough bound on the
number of rational points. The computational complexity seems out of
reach at this point even if we assume the Birch and Swinnerton-Dyer
conjecture. Even if the method does work, the bound is estimated to be
at least about 50. This is too far from 11, which is what we believe
to be the maximal possible number of rational points on the
normalization of $C_P$, after accounting for points at infinity and
singularities.\footnote{%
  We know five regular rational points on $C_P$: $(0, 0)$, $(4,
  \frac{1}{3})$, $(1, \frac{8}{3})$, $(6, 9)$, and a point at infinity
  $[0, 1, 0]$; and three singular rational points on $C_P$ that turn
  out to be nodes: $(3, 0)$, and two points at infinity --- $[1, 3,
  0]$ and $[1, 0, 0]$. For each node there are at most two rational
  points ``upstairs'' in the normalization, so we expect the
  normalization of $C_P$ to have at most 11 rational points. See
  \texttt{mma/pab-ratpoint.nb} for details.}
It seems that radically new ideas are needed to solve this problem.

Another completely different approach to the non-existence conjecture
is to prove Conjecture~\ref{cj:galois-conjugate} (of a completely
different flavor) in Subsection~\ref{subsec:general}, which implies
the non-existence of quadratic 5-cycles. The implication will also be
demonstrated in Subsection~\ref{subsec:general}.

At any rate, the conjecture itself looks very solid.

\subsection{General results}
\label{subsec:general}

While studying the quadratic N = 4 case, we tried to mimic results to
the general case. In particular, the following result by Panraksa was
crucial to proving Lemma~\ref{lem:z1+z3}, which was later used to
establish a complete understanding of the $N = 4$ case:

\begin{lemma} [Panraksa, Lemma~2.3.1 in \cite{MR2982105}]
  Let $c \in \Q$, and let $\set{z_1, z_2, z_3, z_4} \subset K$ be a
  4-cycle of $\phi_c(z) = z^2 + c$, where $K$ is a quadratic extension
  to $\Q$. Then exactly one of the following holds:
  \begin{enumerate}[(i)]
  \item $z_3 = \ol{z_1}$;\footnote{%
      $\ol{z_1}$ denotes the Galois conjugate of $z_1$ in $K$, which
      is unambiguous since $K$ is quadratic.}

  \item $\{z_1, z_2, z_3, z_4\} \cap \{\ol{z_1}, \ol{z_2}, \ol{z_3},
    \ol{z_4}\} =\es.$
  \end{enumerate}
\end{lemma}

Panraksa then proves that the second case $\{z_1, z_2, z_3, z_4\} \cap
\{\ol{z_1}, \ol{z_2}, \ol{z_3}, \ol{z_4}\} = \es$ is impossible, from
which Lemma~\ref{lem:z1+z3} immediately follows. We managed to prove a
much more general version of his lemma:

%\newcommand{\nd}{\frac{N}{(N, d)}}

\begin{theorem}
  Let $N \in \N^*$, $c \in \Q$, $K$ be a Galois extension of $\Q$ with
  degree $d = [K : \Q]$, and $\set{z_0, \dots, z_{N-1}} \subset K$ be
  an exact $N$-cycle of $\phi_c(z) = z^2 + c$. Then exactly one of the
  following holds:
  \begin{enumerate}[(i)]
  \item $z_{i\nd} = \t(z_0)$ for some $0 \le i \le (N, d)-1$ and some
    nontrivial $\t \in \gal(K/\Q)$;

  \item $\set{z_0, \dots, z_{N-1}} \cap \set{\t(z_0), \dots,
      \t(z_{N-1})} = \es$ for all nontrivial $\t \in \gal(K/\Q)$.
  \end{enumerate}
\end{theorem}

\begin{proof}
  For notational convenience, we let $z_{j+N} = z_j$ for all $j \in
  \Z$, i.e., if $j = q \cdot N + r$ with $0 \le r \le N-1$, then $z_j
  = z_r = \phi_c^r(z_0)$.

  First note that (i) and (ii) cannot be simultaneously true. In fact,
  if (i) is true, i.e., $z_{i\nd} = \t(z_0)$ for some $i$ and
  nontrivial $\t \in \gal(K/\Q)$, then $z_{i\nd} \in \set{z_0, \dots,
    z_{N-1}} \cap \set{\t(z_0), \dots, \t(z_{N-1})}$, so (ii) is
  false.

  If (ii) is true, then we are done. Otherwise, (ii) is false, so we
  have some nontrivial $\t \in \gal(K/\Q)$ such that $z_k = \t(z_j)$
  for some $j$ and $k$ satisfying $0 \le j, k \le N-1$. Note that $\t$
  commutes with $\phi_c$ (as $\phi_c$ is defined over $\Q$ and $\t$ is
  a field automorphism fixing the base field $\Q$), so we may assume
  $j = 0$; otherwise, note that $\t(z_0) = \t(z_N) =
  \t(\phi_c^{N-j}(z_j)) = \phi_c^{N-j}(\t(z_j)) = \phi_c^{N-j}(z_k) =
  z_{N+k-j}$, so we may set $j$ to 0 and $k$ to the remainder of
  $N+k-j$ modulo $N$. Assuming $j = 0$, we have $\t(z_0) = z_k$; this,
  together with the fact that $\t$ commutes with $\phi_c$, implies
  that $\t = \phi_c^k$ on the whole cycle $\set{z_0, \dots, z_{N-1}}$.

  Now recall that $K$ is Galois, so the order of the Galois group
  $\gal(K/\Q)$ is exactly $[K : \Q] = d$. Therefore, by Lagrange's
  theorem, $\t^d = Id$, and hence
  \[
  z_0 = \t^d(z_0) = (\phi_c^k)^d(z_0) = z_{kd}.
  \]
  Let $r$ be the remainder of $kd$ modulo $N$. If $r$ is nonzero, then
  $\set{z_0, \dots, z_{r-1}}$ forms a cycle of $\phi_c$ with length $r
  < N$, violating the premise that $\set{z_0, \dots, z_{N-1}}$ is an
  exact $N$-cycle. Therefore, $r = 0$, i.e., $N \mid kd$, and hence
  $\nd \mid k$. Since $0 \le k \le N - 1$, we have $k = i \cdot \nd$
  for some $0 \le i \le (N, d) - 1$. Consequently, $\t(z_0) = z_k =
  z_{i\nd}$, (i) is true, and we are done.
\end{proof}

In fact, we believe that the second case never occurs in
general. Although the supply of examples is limited, all of the
examples that we currently know satisfy the first case (all 4-cycles,
the 5-cycles described by Flynn, Poonen, and Schaefer
\cite{MR1480542}, and the 6-cycles described by Stoll
\cite{MR2465796}). We state our conjecture as follows.

\begin{conjecture}
  \label{cj:galois-conjugate}
  Let $N \in \N^*$, $c \in \Q$, $K$ be a Galois extension of $\Q$ with
  degree $d = [K : \Q]$, and $\set{z_0, \dots, z_{N-1}} \subset K$ be
  an exact $N$-cycle of $\phi_c(z) = z^2 + c$. Then $z_{i\nd} =
  \t(z_0)$ for some $0 \le i \le (N, d)-1$ and some nontrivial $\t \in
  \gal(K/\Q)$.
\end{conjecture}

If the conjecture is true, then in particular we have some interesting
consequences for $d = 2$, i.e., $K$ quadratic, which is the case we
are most interested in. In this case, $K$ is automatically Galois, so
the conjecture can be applied unconditionally.

If $N$ is odd, then $(N, d) = 1$, so we have $z_0 = \ol{z_0}$, i.e.,
$z_0$ is rational. Consequently the entire cycle defined over $\Q$,
and we can reduce the problem of finding quadratic $N$-cycles to
purely finding rational $N$-cycles. In particular, for $N = 5$, we
already know there are no rational 5-cycles, so
Conjecture~\ref{cj:n=5-zero}, the main conjecture of
Subsection~\ref{subsec:quadratic-5} that says no quadratic 5-cycles
exist for $c \in \Q$, follows automatically. We record this corollary
as follows:

\begin{corollary}
  If Conjecture~\ref{cj:galois-conjugate} holds, then there are no $c
  \in \Q$ such that $\phi_c(z) = z^2 + c$ has periodic points in $\U$
  of exact period 5.
\end{corollary}

If $N$ is even, then $(N, d) = 2$, and we have either $z_0 =
\ol{z_0}$, in which case the cycle is defined over $\Q$; or $z_0 =
\ol{z_{\frac{N}{2}}}$, in which case $z_0 + z_{\frac{N}{2}}$ is
rational, and consequently the trace $z_0 + \phi_c(z_0) + \cdots +
\phi_c^{N-1}(z_0)$ is rational. In either case, the point on $C_0(N)$
that corresponds to $c$ and the cycle is a rational point, so finding
quadratic periodic points of exact period $N$ is reduced to studying
rational points on $C_0(N)$.

In particular, for $N = 6$, since the rational points on $C_0(6)$ are
already understood due to Stoll's work \cite{MR2465796} (conditional
on the weak Birch and Swinnerton-Dyer conjecture on the Jocabian of
$C_0(6)$), then we can easily show by exhaustion that the only
quadratic 6-cycle is defined over $\Q(\sqrt{33})$, with $c =
-\frac{71}{48}$ and
\begin{equation}
  \label{eq:6-cycle}
  z_0 = -1 + \frac{\sqrt{33}}{12},\,
  z_1 = -\frac{1}{4} - \frac{\sqrt{33}}{6},\,
  z_2 = -\frac{1}{2} + \frac{\sqrt{33}}{12},\,
  z_3 = \ol{z_0},\,
  z_4 = \ol{z_1},\,
  z_5 = \ol{z_2}.
\end{equation}

This establishes the following corollary to
Conjecture~\ref{cj:galois-conjugate}:

\begin{corollary}
  \label{cor:6-cycle}
  Let $J$ be the Jacobian of $C_0(6)$. If the $L$-series $L(J, s)$
  extends to an entire function and satisfies the standard functional
  equation; the weak Birch and Swinnerton-Dyer conjecture is valid for
  $J$; and Conjecture~\ref{cj:galois-conjugate} holds, then the only
  $c \in \Q$ such that $\phi_c(z) = z^2 + c$ has periodic points of
  exact period 6 in $\U$ is $-\frac{71}{48}$, and the corresponding
  periodic points are $z_0, \dots, z_5$ as defined in
  (\ref{eq:6-cycle}).
\end{corollary}

At the time of writing, though, little is known about how to approach
the conjecture. The known proof in the $N = 4$ case relies on a
special structure of $\Phi_4(z, c)$, and cannot be generalized even to
$N = 6$. It might be useful to first check the conjecture for finite
fields; if the conjecture is true for infinitely many different
characteristics, some Chebotarev density-type argument might yield the
full conjecture over number fields (which have characteristic zero).

%%% Local Variables:
%%% TeX-master: "report"
%%% End:
