\section{Background}

Our entire research project has its roots in the study of the dynamics
of morphisms over projective spaces, specifically the periodic points
(or more generally, preperiodic points) of such morphisms. One
profound conjecture in this field is the Uniform Boundedness
Conjecture, put forth by Morton and Silverman in
1994~\cite{MR1264933}, which bounds the number of preperiodic points
of a morphism in terms of the the degree of the base field (over the
rational field $\Q$), the dimension of the projective space, and the
degree of the morphism. We will define the necessary objects leading
up to the Uniform Boundedness Conjecture, briefly describe the status
of the conjecture, and discuss the specific cases that we will
concentrate on.

\begin{definition}
  The \emph{projective space} $\P^N(K)$ of dimension $N$ over a number
  field $K$ is the space of equivalence classes $K^{N+1}/\sim$ in
  $K^{N+1}$, where the equivalence relation $\sim$ is given by
  \[
  (x_0, \dots, x_N) \sim (y_0, \dots, y_N)
  \]
  if and only if $x_i y_j = x_j y_i$ for all $0 \le i, j \le N$. We
  denote such an equivalence class by $[x_0, \dots, x_N]$, where
  $(x_0, \dots, x_N)$ is any element of the equivalence class.
\end{definition}

\begin{definition}
  Given a number field $K$, a \emph{rational map} of degree $d$
  between projective spaces $\P^N(\bar{K})$ and $\P^M(\bar{K})$ is a
  map
  \[
  \begin{gathered}
    \phi: \P^N(\bar{K}) \to \P^M(\bar{K})\\
    P \mapsto [f_0(p), \dots, f_M(p)],
  \end{gathered}
  \]
  where $p \in P$ is any element of the equivalence class $P$, and
  $f_0, \dots, f_M \in \bar{K}[X_0, \dots, X_N]$ are homogeneous
  polynomials of degree $d$ with no common factors. (Well-definedness
  is guaranteed by homogeneity.)

  If all the coefficients of $f_0, \dots, f_M$ lie in $K$, we say that $\phi$ is \emph{defined over} $K$.

  The rational map $\phi$ is \emph{defined at} $P$ if at least one of
  the values $f_0(p), \dots, f_M(p)$ is nonzero.

  The rational map $\phi$ is called a \emph{morphism} if it is defined
  everywhere in $\P^N(\bar{K})$, or equivalently, if the only solution
  in $\bar{K}$ to the simultaneous equations
  \[
  f_0(X_0, \dots, X_N) = \cdots = f_M(X_0, \dots, X_N) = 0
  \]
  is the trivial solution $X_0 = \dots, X_N = 0$.
\end{definition}

\begin{definition}
  Given any morphism $\phi: \P^N \to \P^N$, the \emph{preperiodic
    points of $\phi$} in some space $\P^N(K)$ is the collection of
  points in $\P^N(K)$ with finite forward orbits, or equivalently,
  \[
  \preper(\phi, \P^N(K)) = \set{P \in \P^N(K) : \text{exist $n_1, n_2
      \in \N$ such that $\phi^{n_1}(P) = \phi^{n_2}(P)$}}.
  \]
\end{definition}

Now we are ready to state the Uniform Boundedness Conjecture.

\begin{conjecture}[Morton-Silverman, 1994 ~\cite{MR1264933}]
  Fix integers $d \ge 2$, $N \ge 1$, and $D \ge 1$. There is a
  constant $C(d, N, D)$ such that for all number fields $K/\Q$ of
  degree at most $D$ and all morphisms $\P^N \to \P^N$ of degree $d$
  defined over $K$,
  \[
  \#\preper(\phi, \P^N(K)) \ge C(d, N, D).
  \]
\end{conjecture}

Very little is known about this conjecture. In fact, even the simplest
case $(d, N, D) = (2, 1, 1)$ is not known, that is, the problem of
bounding the number of rational preperiodic points of degree-2
rational morphisms. If we specialize to degree-2 polynomials (i.e.,
maps of the type $[X_0, X_1] \mapsto [a X_0^2 + b X_0 X_1 + c X_1^2,
X_1^2]$, conveniently written as $a z^2 + b z + c$), note that $a z^2
+ b z + c$ can be reduced to the form $z^2 + c'$ by linear
conjugation, which preserves the sizes of orbits, so it suffices to
consider degree-2 polynomials of the type $\phi_c(z) = z^2 + c$. In
this case of $\phi_c(z) = z^2 + c$, the uniform bound is again not
known, but there are partial results when we further stipulate the
period.

One can easily show that there are one-parameter families of
$c$-values for which $\phi_c(z)$ has rational periodic point(s) of
exact period 1, 2, or 3. It was shown by Morton~\cite{MR1665198} that
$\phi_c$ cannot have rational periodic points of exact period 4. The
non-existence result was extended to exact period 5 by Flynn, Poonen,
and Schaefer~\cite{MR1480542}, and to exact period 6 (conditional on
the Birch and Swinnerton-Dyer conjecture for a specific abelian
variety) by Stoll~\cite{MR2465796}.

We are interested in generalizing the aforementioned results for
degree-2 polynomials and specific small periods to quadratic number
fields, i.e., the $D = 2$ case. For a more detailed discussion of
other takes on the Uniform Boundedness Conjecture, along with a list
of references, see Section 3.3 of \cite{MR2316407}.

From now on, unless otherwise states, the dimension of the projective
space will always be 1, and we reserve the letter $N$ for the exact
period of periodic points.

%%% Local Variables:
%%% TeX-master: "report"
%%% End:
