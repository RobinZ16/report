\section{The models and results from literature}
\label{sec:model}

In this section, we set up models to study periodic points of
$\phi_c(z) = z^2 + c$ of exact period $N$. We first set up a general
model, and then specialize to $N = 4$, $5$, and $6$. These models have
already been set up in the literature. We also discuss the results
associated to these models in the literature.

\subsection{The general model}
\label{subsec:model-general}

Let $\phi_c(z) = z^2 + c$. Let $\phi_c^N(z)$ denote the $N$-th
iteration of $f$. Then all periodic points of periodic $N$ (including
points of exact period $d$ such that $d | N$) satisfy the polynomial
equation
\[
\phi_c^N(z) - z = 0.
\]
By the M\"obius inversion formula, we have
\[
\phi_c^N(z) - z = \prod_{d|N} \Phi_d(z, c),
\]
where
\[
\Phi_d(z, c) = \prod_{m|d}(\phi_c^m(z) - z)^{\mu(d/m)}.
\]
With a little bit of effort we can show that $\Phi_d(z, c)$ resides in
$\Z[z, c]$, and that all periodic points of exact period $N$ satisfy
the polynomial equation $\Phi_N(z, c) = 0$.

$\Phi_N(z, c)$ defines an algebraic curve in the $(z,
c)$-plane. Denote the normalization of this curve by $C_1(N)$. Observe
that the map $\phi_c$ permutes cycles of length $N$, so the map
$\sigma: (z, c) \mapsto (\phi_c(z), c)$ is an automorphism of the
curve $C_1(N)$, and it generates a group $\tup{\sigma}$ of order
$N$. Take the quotient curve $C_1(N)/\tup{\sigma}$, and denote the
normalization of the quotient curve by $C_0(N)$. Note that for a given
number field $K$, the $K$-points on $C_0(N)$ do not necessarily
correspond to $K$-points on $C_1(N)$; rather, they correspond to
$\gal(\ol{K}/K)$-stable orbits on $C_1(N)$.

Now that we defined $C_1(N)$ and $C_0(N)$, the study of periodic
points of exact period $N$ in a number field $K$ (where we also
require that $c \in K$) boils down to the study of $K$-points on these
two curves. $K$-points on $C_1(N)$ correspond directly, with finitely
many exceptions due to removal of singularities, to pairs $(z,
\phi_c)$ of a periodic point $z \in K$ of period $N$ and a map
$\phi_c(z) = z^2 + c$ with $c \in K$. $K$-points on $C_0(N)$
correspond, with finitely many exceptions, to pairs $(\O, \phi_c)$ of
a $\gal(\ol{K}/K)$-stable orbit $\O$ of size $N$ and a map $\phi_c(z)
= z^2 + c$ with $c \in K$; obviously these include all $(\O, \phi_c)$
pairs where elements of $\O$ are strictly contained in $K$, and hence
contain full information about periodic points in $K$.

Next we provide models of $C_1(N)$ or $C_0(N)$ for $N = 4$, $5$, or
$6$, and briefly discuss what is already known about these models.

\subsection{The $N = 4$ case}
\label{subsec:model-4}

Morton showed in \cite{MR1665198} that $C_1(4)$, which has genus 2, is
birationally equivalent over $\Q$ to the curve
\begin{equation}
  \label{eq:c1(4)}
  v^2 = u(u^2 + 1)(1 + 2u - u^2),
\end{equation}
which is also an equation for the modular curve $X_1(16)$. $X_1(16)$
is known to have only six rational points $(0, 0)$, $(\pm 1, \pm 2)$,
and the point at infinity. Tracing back to $(z, c)$, Morton was able
to show that none of these six rational points correspond to affine
rational points on the original $(z, c)$-curve, hence

\begin{theorem}[Morton, Theorem~4 in \cite{MR1665198}]
  There are no finite rational solutions $(z, c)$ of the equation
  $\Phi_4(z, c) = 0$. In other words, there are no quadratic
  polynomials defined over $\Q$ with a rational 4-cycle.
\end{theorem}

\begin{remark}
  Here we used the linear conjugation mentioned in the background
  section to generalize the result from $\phi_c(z) = z^2 + c$, $c \in
  \Q$ to all quadratic polynomials defined over the rationals.
\end{remark}

\subsection{The $N = 5$ case}
\label{subsec:model-5}

Flynn, Poonen, and Schaefer showed in \cite{MR1480542} that $C_0(5)$,
which has genus 2, is birationally equivalent to the hyperelliptic
curve
\begin{equation}
  \label{eq:c0(5)}
  y^2 = f(x) = x^6 + 8x^5 + 22x^4 + 22x^3 + 5x^2 + 6x + 1,
\end{equation}
where the original $c$ is given in terms of $x$ and $y$ by
\begin{equation}
  \label{eq:c-in-xy}
  c = \frac{g(x)}{2(P_0(x) - P_1(x) y)}
  = \frac{P_0(x) + P_1(x) y}{h(x)},
\end{equation}
where $g, h, P_0, P_1 \in \Z[x]$ are given by
\begin{subequations}
  \label{eq:poly-defs}
  \begin{align}
    g(x) & = 8x^6 + 74x^5 271x^4 + 452x^3 + 325x^2 + 110x + 64,\\
    h(x) & = 8x^2(x+3)^2,\\
    P_0(x) & = - x^6 - 10x^5 - 46x^4 - 104x^3 - 95x^2 - 24x - 9,\\
    P_1(x) & = x^3 + 6x^2 + 3x - 9.
  \end{align}
\end{subequations}

Using a refined version of Chabauty and Coleman's method (refinement
appropriately cooked for this specific curve), the authors were able
to prove that there are only six rational points on $y^2 = f(x)$: $(0,
\pm 1)$, $(-3, \pm 1)$, and two points at infinity. From there they
were able to prove that none of these corresponds to affine rational
points on the original $(z, c)$-curve (the corresponding points are
either at infinity, or only defined in a degree-5 extension of $\Q$).
Therefore,

\begin{theorem}[FPS, Theorem~1 in \cite{MR1480542}]
  There are no quadratic polynomials defined over $\Q$ with a rational
  5-cycle.
\end{theorem}

The authors also conjectured from there that

\begin{conjecture}[FPS, Conjecture~2 in \cite{MR1480542}]
  If $N \ge 4$, then there are no quadratic polynomials defined over
  $\Q$ with a rational $N$-cycle.
\end{conjecture}

\subsection{The $N = 6$ case}
\label{subsec:model-6}

Stoll showed in \cite{MR2465796} that $C_0(6)$, which has genus 4, is
birationally equivalent to the curve given by
\begin{equation}
  \label{eq:c0(6)}
  w^2(w+1)u^3 - (5w^2+w+1)u^2 - w(w^2-2w-7)u + (w+1)(w-3) = 0,
\end{equation}
where
\begin{equation}
  \label{eq:c-in-uw}
  c = \frac{(- u^3 - 2u^2 + 5u - 10)uw - u^4 + 3u^3 + 8u^2 - 10u +
    12}{4u^2(uw+u-3)}.
\end{equation}
Again using Chabauty and Coleman's method, and assuming the weak Birch
and Swinnerton-Dyer conjecture\footnote{%
  For a description of the Birch and Swinnerton-Dyer conjecture, see
  \cite{MR2238272}.}
for the Jacobian of $C_0(6)$,\footnote{%
  See p.~299 of \cite{MR1917232} for the definition of Jacobian.
}
Stoll was able to show that there are only ten rational points on the
curve given by (\ref{eq:c0(6)}): $(0, \infty)$, $(0, -1)$, $(0, 3)$,
$(\infty, 0)$, $(1, 2)$, $(2, 1)$, $(1, \infty)$, $(\infty, -1)$,
$(-1, \infty)$, and $(-4/5, -1)$. None of them corresponds to affine
rational points on the original $(z, c)$-curve, so

\begin{theorem}[Stoll, Theorem~7 in \cite{MR2465796}]
  Let $J$ be the Jacobian of $C_0(6)$. If the $L$-series $L(J,s)$
  extends to an entire function and satisfies the standard functional
  equation, and if the weak Birch and Swinnerton-Dyer conjecture is
  valid for $J$, then there are no quadratic polynomials defined over
  $\Q$ with a rational 6-cycle.
\end{theorem}

%%% Local Variables:
%%% TeX-master: "report"
%%% End:
