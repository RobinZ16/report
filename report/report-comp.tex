\section{Computational efforts, confirmations, and observations}

\subsection{Quadratic periodic points with rational $c$}

We want to search, for rational $c$, quadratic periodic points with
exact period $N$, where by quadratic points we mean elements of
some quadratic number field $K/\Q$. This boils down to finding
quadratic points on $C_1(N)$, which is the smooth projective model of
$\Phi_N(z, c)$. There are several approaches.

\subsubsection{Enumerating $c \in \Q$}

The first possible approach is to enumerate $c \in \Q$. A natural
order of enumeration is the increasing order of height, where the
height of a rational number $p/q$, with $p,\, q \in \Z$ and $\gcd(p,
q) = 1$, is defined as $\max\set{|p|,\, |q|}$.

For each fixed $c$, we factorize the polynomial $\Phi_{N,c}(z)$ over
$\Q$ and look at its irreducible factors. Let $D_{N,c}$ be the
multiset of the degrees of $\Phi_{N,c}(z)$ irreducible
factors. Clearly the elements of $D_{N,c}$ sum to $\deg \Phi_{N,c}(z)
= \sum_{d|N}2^{d \mu(N/d)}$. Then $z^2 + c$ has quadratic periodic
points of exact period $N$ if and only if $2 \in D_{N,c}$. Factorizing
$\Phi_{N,c}(z)$ over $\Q$ to obtain $D_{N,c}$ is doable (polynomial
time algorithm exists) for all $N$ with software, e.g., Mathematica,
although the complexity grows exponentially with $N$ (note that $\deg
\Phi_{N,c}$ grows roughly as $2^N$).

We performed the search with $N = 4$ and 5, enumerating $c$ up to
height 1000. (See the Mathematica notebook \texttt{misc/mma/factor}.)

For $N = 4$, we found various $c$ with quadratic rational points:
$-4/5$, $-31/48$, $-95/48$, $-155/72$, $-209/72$. From this result it
is conjectured that there are infinitely many such $c$'s, and this is
later confirmed by theory. It is also worth noting that for $c$'s that
we observed, $D_{4,c}$ is always one of the following:
\[
\set{12},\, \set{4, 8},\, \set{2, 2, 8}.
\]
Note that the degrees of irreducible factors of $\Phi_{4,c}(z)$
(elements of $D_{4,c}$) always have common factors with $N = 4$. It is
further observed that $c$'s such that $\Phi_{4,c}(z)$ is reducible
always has ``fairly divisible'' denominators, i.e., their prime
factors are ``small'' compared to themselves. The implication of this
is not clear.

For $N=5$, no quadratic factors were observed, and it is conjectured
that there are no quadratic periodic points of exact period 5 for any
$c \in \Q$. Even the collection of $c$'s such that $\Phi_{5,c}(z)$ is
reducible is very limited:
\[
\begin{gathered}
  c = -2,\, D_{5,c} = \set{5,\, 10,\, 15},\\
  c = -4/3,\, D_{5,c} = \set{10,\, 20},\\
  c = -16/9,\, D_{5,c} = \set{5,\, 25},\\
  c = -64/9,\, D_{5,c} = \set{5,\, 25}.
\end{gathered}
\]
Note again that the degrees of irreducible factors of $\Phi_{5,c}(z)$ always have common
factors with $N = 5$.

We also performed the search for $N = 6$ up to height 500. Quadratic
factor was found for one $c$:
\[
c = -71/48,\, D_{6,c} = \set{2,\, 2,\, 2,\, 48},
\]
and two more $c$ were observed such that $\Phi_{6,c}(z)$ is reducible:
\[
\begin{gathered}
  c = -2,\, D_{6,c} = \set{6,\, 6\, 18,\, 24},\\
  c = -4,\, D_{4,c} = \set{6,\, 48}.
\end{gathered}
\]
Note once again that (from the limited data we have) the degrees of
irreducible factors of $\Phi_{6,c}(z)$ always have common factors with
$N = 6$.

% I will add bibliography and citations later
Also recall from Morton1 that
\begin{theorem}
  For $N = 3$ and any $c \in \Q$, $D_{3, c}$ must be one of the
  following:
  \[
  \set{6},\, \set{3,\, 3}, \set{3,\, 1,\, 1,\, 1}.
  \]
\end{theorem}

Combining our knowledge of $D_{3,c}$, $D_{4,c}$, $D_{5,c}$, and
$D_{6,c}$, we have the following conjecture.

\begin{conjecture}[Weak factorization]
  For $N \ge 3$ and any $c \in \Q$, one of the following holds:
  \begin{enumerate}
  \item $1 \in D_{N,c}$;
  \item For all $d \in D_{N,c}$, $\gcd(d,\, N) > 1$.
  \end{enumerate}
\end{conjecture}

$N = 3$ is still rather special, since there are inifinitely many $c$
with rational periodic points of exact period 3, and we know the
explicit parametrization of $c$ such that $\Phi_{3,c}(z)$ is
reducible (Morton1). If we exclude $N = 3$, we have the following
stronger conjecture.

\begin{conjecture}[Strong factorization]
  For $N \ge 4$ and any $c \in \Q$, we have $\gcd(d, N) > 1$ for all
  $d \in D_{N,c}$.
\end{conjecture}

This conjecture implies FPS's conjecture.

\begin{conjecture}[FPS]
  For $N \ge 4$ and any $c \in \Q$, $1 \not\in D_{N,c}$.
\end{conjecture}

However, these conjectures seem to require radically new ideas not
available at this moment, as most current results in this field relies
on manipulating specific curves.

\subsubsection{Enumerating quadratic $x$ on $C_0(5)$}

Now we specialize to $N = 5$, which is the simplest case not yet
resolved. For $N = 5$, FPS already worked out a hyperelliptic model
for $C_0(5)$:
\[
y^2 = f(x) = x^6 + 8x^5 + 22x^4 + 22x^3 + 5x^2 + 6x + 1.
\]
$c$ is given by
\[
c = \frac{P_0(x) + P_(x) y}{h(x)} = \frac{g(x)}{2(P_0(x) - P_1(x) y)},
\]
where
\[
\begin{gathered}
  P_0(x) = - 9 - 24x - 95x^2 - 104x^3 - 46x^4 - 10x^5 - x^6,\\
  P_1(x) = - 9 + 3x + 6x^2 + x^3,\\
  g(x) = 64 + 110x + 325x^2 + 452x^3 + 271x^4 + 74x^5 + 8x^6,\\
  h(x) = 8x^2(3 + x)^2.
\end{gathered}
\]
Note that quadratic periodic points of exact period 5 (more precisely,
orbits of such quadratic points) map to quadratic points on $C_0(5)$,
so we may enumerate quadratic points on $y^2 = f(x)$ and verify if $c
\in \Q$ is satisfied and if the preimage of such a quadratic point on
$C_0(5)$ is indeed a quadratic orbit on $C_1(5)$ (rather than a
$\gal(\ol{\Q}/\Q)$-fixed orbit living in a degree-10 number field).

To enumerate quadratic points on $C_0(5)$, we fix a squarefree $d \in
\Z$, and enumerate $x \in K = \Q(\sqrt{d})$. We still enumerate in the
increasing order of height, where the height of $\frac{p_1}{q_1} +
\frac{p_2}{q_2} \sqrt{d} \in \Q(\sqrt{d})$, with $p_1$, $q_1$, $p_2$,
$q_2 \in \Z$ and $\gcd(p_1,\, q_1) = \gcd(p_2,\, q_2) = 1$, is defined
as $\max\set{|p_1|,\, |q_1|,\, |p_2|,\, |q_2|}$ as usual. For each
$x$, we then compute $y^2$ from the relation $y^2 = f(x)$. Since we
want to have $y \in K = \Q(\sqrt{d})$, we need to check if $f(x)$ is a
square in $K$. This turned out to be a rather expensive task
(equivalent to factorizing $T^2 - f(x)$ in $K[T]$), but $N(f(x))$ is a
rational square is a great criterion to exclude ineligible $f(x)$,
where $N$ is the conventional norm in $\Q(\sqrt{d})$. Therefore, the
main workflow boils down to enumerating $d$, $p_1$, $q_1$, $p_2$,
$q_2$, perform multiprecision rational arithmetic (calculating $f(x)$
and $N(f(x))$), and test if the numerator and denominator of $N(f(x))$
are integer squares.

We implemented this same procedure in several languages --- Sage,
Mathematica, and C++ (using FLINT 2). On darwin-x86\_64 with 2.9 GHz
Intel Core i7 processor, FLINT 2.4.4, Mathematica 9.0.1 Student
Edition, and Sage 6.3, it turned out that Mathematica was consistently
about 20 times faster than Sage, and C++ with FLINT (using fmpzxx and
fmpqxx) was consistently about 50 times faster than Mathematica (from
which we conclude that FLINT is the winner performance-wise when it
comes to multiprecision rational arithmetic).

With the C++ program we were able to enumerate squarefree $d \in
\set{-100, \dots, 100}$ and $x \in \Q(\sqrt{d})$ up to height 30, and
for specific $d$ of interest, i.e., $d = -87$ and $d = 33$,
% note that I recently discovered d = -87 and haven't get arount to
% really do it up to height 100 just yet
up to height 100. Only eight quadratic points (excluding all the
rational points already known in FPS) on $C_0(5)$ were found, four of
which give rational $c$:
\[
\begin{gathered}
  d = -87,
  x = -\frac{1}{6} \sqrt{-87} - \frac{1}{2},
  y = -\frac{5}{9} \sqrt{-87} + \frac{22}{3},
  c = \frac{13371}{86528} \sqrt{-87} + \frac{12767}{86528},
  \\
  d = -87,
  x = -\frac{1}{6} \sqrt{-87} - \frac{1}{2},
  y = \frac{5}{9} \sqrt{-87} - \frac{22}{3},
  c = -\frac{4}{3},
  \\
  d = -87,
  x = \frac{1}{6} \sqrt{-87} - \frac{1}{2},
  y = \frac{5}{9} \sqrt{-87} + \frac{22}{3},
  c = -\frac{13371}{86528} \sqrt{-87} + \frac{12767}{86528},
  \\
  d = -87,
  x = \frac{1}{6} \sqrt{-87} - \frac{1}{2},
  y = -\frac{5}{9} \sqrt{-87} - \frac{22}{3},
  c = -\frac{4}{3},
  \\
  d = 33,\,
  x = -\frac{1}{3} \sqrt{33} - 2,\,
  y = \frac{10}{9} \sqrt{33} + \frac{17}{3},\,
  c = -2,
  \\
  d = 33,\,
  x = -\frac{1}{3} \sqrt{33} - 2,\,
  y = -\frac{10}{9} \sqrt{33} - \frac{17}{3},\,
  c = \frac{269}{128} \sqrt{33} - \frac{6415}{384},
  \\
  d = 33,\,
  x = \frac{1}{3} \sqrt{33} - 2,\,
  y = \frac{10}{9} \sqrt{33} - \frac{17}{3},\,
  c = -\frac{269}{128} \sqrt{33} - \frac{6415}{384},
  \\
  d = 33,\,
  x = \frac{1}{3} \sqrt{33} - 2,\,
  y = -\frac{10}{9} \sqrt{33} + \frac{17}{3},
  c = -2.
\end{gathered}
\]
$c = -\frac{4}{3}$ and $-2$ can be shown to not yield any quadratic
periodic points of period 5.

%%% Local Variables:
%%% TeX-master: "report"
%%% End:
