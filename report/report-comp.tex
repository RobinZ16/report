\section{Computational efforts, confirmations, and observations}
\label{sec:comp}

\subsection{Desperately seeking quadratic periodic points}

Recall that the main question we propose in our paper,
Question~\ref{question}, concerns the quadratic periodic points of
$\phi_c(z) = z^2 + c$ of exact period $N$. It would be nice to
enumerate these points computionally, even if the problem cannot be
fully resolved theoretically. Sometimes we also pick us a few nice
observations and conjectures not directly related to the original
problem along the way.

\subsubsection{Basic enumeration strategy}

Enumerating integers is the easiest, but does not seem to help our
problem.

Enumerating rationals is relatively easy, which essentially boils down
to enumerating two coprime integers (with the denominator being
positive). Enumeration requires an ordering (or at least some kind of
bound so that the enumeration terminates at some point --- we cannot
enumerate all rationals in practice), and to obtain an ordering on the
rationals, we typically use a height function $h$, conventionally
defined as $h(p/q) = \max\set{|p|, |q|}$ for $(p, q) = 1$. Then we
enumerate in the increasing order of height, or use some other
ordering but terminate at a certain height (enuemrate every rational
up to some height).

Enumerating elements of $\U$ is by far the hardest. Elements of $\U$
can be represented as $\frac{p_1}{q_1} + \frac{p_2}{q_2}\sqrt{d}$,
where $p_1, q_1, p_2, q_2, d$ are integers with the further
restrictions that $q_1, q_2 > 0$, $(p_1, q_1) = (p_2, q_2) = 1$, and
$d$ is squarefree. To enumerate elements of $\U$, the most logical
ordering as we see it is to first enumerate $d$, then for each $d$,
define the height function $h$ as $h(\frac{p_1}{q_1} +
\frac{p_2}{q_2}\sqrt{d}) = \max\set{|p_1|, |q_1|, |p_2|, |q_2|}$, and
enumerate up to some height.

There are two approaches to finding the quadratic periodic points,
which are explained below.

\subsubsection{Enumerating $c \in \Q$}

The first possible approach is to enumerate $c \in \Q$. This is a
naive approach, and works for any $N$. For a specific $c$, to see if
there are quadratic periodic points of $\phi_c$ with exact period $N$,
we simply factorize $\Phi_N(z, c)$ in $\Q[z]$, and see if there are
quadratic factors.

%%% TRIMMED
%
% With regret, I realized that the observations and conjectures might
% not be as interesting as what I previously thought. When you are
% factorization a polynomial that describe N-cycles, it makes a lot of
% sense that every irreducible factor is related to N --- that's
% because if z is a periodic point, then phi_c(z), ..., phi_c^{N-1}(z)
% are all periodic points, with phi_c^N(z) back to z, so typically
% they are the solutions to the same irreducible factor; only in rare
% cases do they further reduce to smaller factors, but even then they
% still appear in clusters. (This is roughly what our "general result"
% is saying; and it might not be very interesting for similar
% reasons.
%
% In any case, I'm trimming this once-cherished conjecture out. It
% will be preserved in a branch called `trimmed`, in case it is
% somehow resurrected.
%
% --------------------------------------------------------------------
%
% For future reference, we make the following
% definition:
%
% \begin{definition}
%   For given $c \in \Q$ and $N$, we denote by $D_{N,c}$ the multiset
%   formed by the degrees of irreducible factors of $\Phi_N(z, c)$ in
%   $\Q[z]$.
% \end{definition}
%
% For instance, $\Phi_2(z, c) = z^2+z+(c+1)$ is irreducible for most
% $c \in \Q$, but factors as $(z + (1/2 + r))(z + (1/2 - r))$ when $c =
% -\frac{r^2 + 3}{4}$ for some rational $r \in \Q$. Therefore, $D_{2,c}
% = \set{2}$ for most $c$, but becomes $\set{1, 1}$ when $c$ is
% parametrized as above.
%
% With $D_{N,c}$, the question of whether there are quadratic periodic
% points reduces to whether 2 is in $D_{N,c}$ or not.

On a technical note, the complexity of factorizing $\Phi_N(z, c)$
blows up quickly since $\deg_z \Phi_N(z, c)$ grows roughly as $2^N$,
but it is still doable for small $N$.

We performed the search for $N =$ 4 and 5, enumerating $c$ up to
height 1000. (See the Mathematica notebook \texttt{mma/factor}.)

For $N = 4$, we found many $c$ with quadratic points, and it was later
confirmed by theory that the number is indeed infinite (see
Subsection~\ref{subsec:quadratic-4}).

%%% TRIMMED
% However, one interesting observation is that $D_{4,c}$ is always one
% of the following:
% \[
% \set{12},\, \set{4, 8},\, \set{2, 2, 8}.
% \]

For $N=5$, no quadratic factors were observed, and this provides
evidence for our conjecture that there are no quadratic periodic
points of exact period 5 for any $c \in \Q$ (see
Subsection~\ref{subsec:quadratic-5}).

%%% TRIMMED
% Even the collection of $c$'s such that $\Phi_{5,c}(z)$ is reducible is
% very limited:
% \[
% \begin{gathered}
%   c = -2,\, D_{5,c} = \set{5,\, 10,\, 15},\\
%   c = -4/3,\, D_{5,c} = \set{10,\, 20},\\
%   c = -16/9,\, D_{5,c} = \set{5,\, 25},\\
%   c = -64/9,\, D_{5,c} = \set{5,\, 25}.
% \end{gathered}
% \]
% Note again that the degrees of irreducible factors of $\Phi_{5,c}(z)$
% always have common factors with $N = 5$.

We also performed the search for $N = 6$ up to height 500. Quadratic
factor was found only for $c = -71/48$.

%%% TRIMMED
% \[
% c = -71/48,\, D_{6,c} = \set{2,\, 2,\, 2,\, 48},
% \]
% and two more $c$ were observed such that $\Phi_{6,c}(z)$ is reducible:
% \[
% \begin{gathered}
%   c = -2,\, D_{6,c} = \set{6,\, 6\, 18,\, 24},\\
%   c = -4,\, D_{4,c} = \set{6,\, 48}.
% \end{gathered}
% \]
% Note once again that (from the limited data we have) the degrees of
% irreducible factors of $\Phi_{6,c}(z)$ always have common factors with
% $N = 6$.
%
% Also recall from Morton1 that
% \begin{theorem}
%   For $N = 3$ and any $c \in \Q$, $D_{3, c}$ must be one of the
%   following:
%   \[
%   \set{6},\, \set{3,\, 3}, \set{3,\, 1,\, 1,\, 1}.
%   \]
% \end{theorem}
%
% Combining our knowledge of $D_{3,c}$, $D_{4,c}$, $D_{5,c}$, and
% $D_{6,c}$, we have the following conjecture.
%
% \begin{conjecture}[Weak factorization]
%   For $N \ge 3$ and any $c \in \Q$, one of the following holds:
%   \begin{enumerate}
%   \item $1 \in D_{N,c}$;
%   \item For all $d \in D_{N,c}$, $\gcd(d,\, N) > 1$.
%   \end{enumerate}
% \end{conjecture}
%
% $N = 3$ is still rather special, since there are inifinitely many $c$
% with rational periodic points of exact period 3, and we know the
% explicit parametrization of $c$ such that $\Phi_{3,c}(z)$ is
% reducible (Morton1). If we exclude $N = 3$, we have the following
% stronger conjecture.
%
% \begin{conjecture}[Strong factorization]
%   For $N \ge 4$ and any $c \in \Q$, we have $\gcd(d, N) > 1$ for all
%   $d \in D_{N,c}$.
% \end{conjecture}
%
% This conjecture implies FPS's conjecture.
%
% \begin{conjecture}[FPS]
%   For $N \ge 4$ and any $c \in \Q$, $1 \not\in D_{N,c}$.
% \end{conjecture}
%
% However, these conjectures seem to require radically new ideas not
% available at this moment, as most current results in this field relies
% on manipulating specific curves.

\subsubsection{Enumerating quadratic $x$ on $C_0(5)$}

Now we specialize to $N = 5$, which is the simplest case not yet
resolved. For $N = 5$, FPS already worked out a hyperelliptic model
for $C_0(5)$:
\[
y^2 = f(x) = x^6 + 8x^5 + 22x^4 + 22x^3 + 5x^2 + 6x + 1.
\]
$c$ is given by
\[
c = \frac{P_0(x) + P_(x) y}{h(x)} = \frac{g(x)}{2(P_0(x) - P_1(x) y)},
\]
where
\[
\begin{gathered}
  P_0(x) = - 9 - 24x - 95x^2 - 104x^3 - 46x^4 - 10x^5 - x^6,\\
  P_1(x) = - 9 + 3x + 6x^2 + x^3,\\
  g(x) = 64 + 110x + 325x^2 + 452x^3 + 271x^4 + 74x^5 + 8x^6,\\
  h(x) = 8x^2(3 + x)^2.
\end{gathered}
\]
Note that quadratic periodic points of exact period 5 (more precisely,
orbits of such quadratic points) map to quadratic points on $C_0(5)$,
so we may enumerate quadratic points on $y^2 = f(x)$ and verify if $c
\in \Q$ is satisfied and if the preimage of such a quadratic point on
$C_0(5)$ is indeed a quadratic orbit on $C_1(5)$ (rather than a
$\gal(\ol{\Q}/\Q)$-fixed orbit living in a degree-10 number field).

To enumerate quadratic points on $C_0(5)$, we fix a squarefree $d \in
\Z$, and enumerate $x \in K = \Q(\sqrt{d})$. We still enumerate in the
increasing order of height, where the height of $\frac{p_1}{q_1} +
\frac{p_2}{q_2} \sqrt{d} \in \Q(\sqrt{d})$, with $p_1$, $q_1$, $p_2$,
$q_2 \in \Z$ and $\gcd(p_1,\, q_1) = \gcd(p_2,\, q_2) = 1$, is defined
as $\max\set{|p_1|,\, |q_1|,\, |p_2|,\, |q_2|}$ as usual. For each
$x$, we then compute $y^2$ from the relation $y^2 = f(x)$. Since we
want to have $y \in K = \Q(\sqrt{d})$, we need to check if $f(x)$ is a
square in $K$. This turned out to be a rather expensive task
(equivalent to factorizing $T^2 - f(x)$ in $K[T]$), but $N(f(x))$ is a
rational square is a great criterion to exclude ineligible $f(x)$,
where $N$ is the conventional norm in $\Q(\sqrt{d})$. Therefore, the
main workflow boils down to enumerating $d$, $p_1$, $q_1$, $p_2$,
$q_2$, perform multiprecision rational arithmetic (calculating $f(x)$
and $N(f(x))$), and test if the numerator and denominator of $N(f(x))$
are integer squares.

We implemented this same procedure in several languages --- Sage,
Mathematica, and C++ (using FLINT 2). On darwin-x86\_64 with 2.9 GHz
Intel Core i7 processor, FLINT 2.4.4, Mathematica 9.0.1 Student
Edition, and Sage 6.3, it turned out that Mathematica was consistently
about 20 times faster than Sage, and C++ with FLINT (using fmpzxx and
fmpqxx) was consistently about 50 times faster than Mathematica (from
which we conclude that FLINT is the winner performance-wise when it
comes to multiprecision rational arithmetic).

With the C++ program we were able to enumerate squarefree $d \in
\set{-100, \dots, 100}$ and $x \in \Q(\sqrt{d})$ up to height 30, and
for specific $d$ of interest, i.e., $d = -87$ and $d = 33$,
% note that I recently discovered d = -87 and haven't get arount to
% really do it up to height 100 just yet
up to height 100. Only eight quadratic points (excluding all the
rational points already known in FPS) on $C_0(5)$ were found, four of
which give rational $c$:
\[
\begin{gathered}
  d = -87,
  x = -\frac{1}{6} \sqrt{-87} - \frac{1}{2},
  y = -\frac{5}{9} \sqrt{-87} + \frac{22}{3},
  c = \frac{13371}{86528} \sqrt{-87} + \frac{12767}{86528},
  \\
  d = -87,
  x = -\frac{1}{6} \sqrt{-87} - \frac{1}{2},
  y = \frac{5}{9} \sqrt{-87} - \frac{22}{3},
  c = -\frac{4}{3},
  \\
  d = -87,
  x = \frac{1}{6} \sqrt{-87} - \frac{1}{2},
  y = \frac{5}{9} \sqrt{-87} + \frac{22}{3},
  c = -\frac{13371}{86528} \sqrt{-87} + \frac{12767}{86528},
  \\
  d = -87,
  x = \frac{1}{6} \sqrt{-87} - \frac{1}{2},
  y = -\frac{5}{9} \sqrt{-87} - \frac{22}{3},
  c = -\frac{4}{3},
  \\
  d = 33,\,
  x = -\frac{1}{3} \sqrt{33} - 2,\,
  y = \frac{10}{9} \sqrt{33} + \frac{17}{3},\,
  c = -2,
  \\
  d = 33,\,
  x = -\frac{1}{3} \sqrt{33} - 2,\,
  y = -\frac{10}{9} \sqrt{33} - \frac{17}{3},\,
  c = \frac{269}{128} \sqrt{33} - \frac{6415}{384},
  \\
  d = 33,\,
  x = \frac{1}{3} \sqrt{33} - 2,\,
  y = \frac{10}{9} \sqrt{33} - \frac{17}{3},\,
  c = -\frac{269}{128} \sqrt{33} - \frac{6415}{384},
  \\
  d = 33,\,
  x = \frac{1}{3} \sqrt{33} - 2,\,
  y = -\frac{10}{9} \sqrt{33} + \frac{17}{3},
  c = -2.
\end{gathered}
\]
$c = -\frac{4}{3}$ and $-2$ can be shown to not yield any quadratic
periodic points of period 5.

%%% Local Variables:
%%% TeX-master: "report"
%%% End:
