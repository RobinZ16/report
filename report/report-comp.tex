\section{Computational efforts, confirmations, and observations}
\label{sec:comp}

\subsection{Desperately seeking quadratic periodic points}
\label{subsec:seek-quadratic}

Recall that the main question we propose in our paper,
Question~\ref{question}, concerns the quadratic periodic points of
$\phi_c(z) = z^2 + c$ of exact period $N$. It would be nice to
enumerate these points computationally, even if the problem cannot be
fully resolved theoretically. Sometimes we also pick us a few nice
observations and conjectures not directly related to the original
problem along the way.

\subsubsection{Basic enumeration strategy}
\label{sssec:enum}

Enumerating integers is the easiest, but does not seem to help our
problem.

Enumerating rationals is relatively easy, which essentially boils down
to enumerating two coprime integers (with the denominator being
positive). Enumeration requires an ordering (or at least some kind of
bound so that the enumeration terminates at some point --- we cannot
enumerate all rationals in practice), and to obtain an ordering on the
rationals, we typically use a height function $h$, conventionally
defined as $h(p/q) = \max\set{|p|, |q|}$ for $(p, q) = 1$. Then we
enumerate in the increasing order of height, or use some other
ordering but terminate at a certain height (i.e., enumerate every
rational number up to some height).

Enumerating elements of $\U$ is by far the hardest. Elements of $\U$
can be written as $\frac{p_1}{q_1} + \frac{p_2}{q_2}\sqrt{d}$, where
$p_1, q_1, p_2, q_2, d$ are integers with the further restrictions
that $q_1, q_2 > 0$, $(p_1, q_1) = (p_2, q_2) = 1$, and $d$ is
squarefree. To enumerate elements of $\U$, the most logical ordering
as we see it is to first enumerate $d$, then for each $d$, define the
height function $h$ as $h(\frac{p_1}{q_1} + \frac{p_2}{q_2}\sqrt{d}) =
\max\set{|p_1|, |q_1|, |p_2|, |q_2|}$, and enumerate up to some
height.

There are two approaches to finding the quadratic periodic points,
which are explained below.

\subsubsection{Enumerating $c \in \Q$}

The first possible approach is to enumerate $c \in \Q$. This is a
naive approach, and works for any $N$. For a specific $c$, to see if
there are quadratic periodic points of $\phi_c$ with exact period $N$,
we simply factorize $\Phi_N(z, c)$ in $\Q[z]$, and see if there are
quadratic factors.

On a technical note, the complexity of factorizing $\Phi_N(z, c)$
blows up quickly since $\deg_z \Phi_N(z, c)$ grows roughly as $2^N$,
but it is still doable for small $N$.

We performed the search for $N =$ 4 and 5, enumerating $c$ up to
height 1000. (See the Mathematica notebook \texttt{mma/enum-c.nb}.)

For $N = 4$, we found many $c$ with quadratic points, and it was later
confirmed by theory that the number is indeed infinite (see
Subsection~\ref{subsec:quadratic-4}).

For $N=5$, no quadratic factors were observed, and this provides
evidence for Conjecture~\ref{cj:n=5-zero} that there are no quadratic
periodic points of exact period 5 for any $c \in \Q$ (see
Subsection~\ref{subsec:quadratic-5}).

We also performed the search for $N = 6$ up to height 500. Quadratic
factor was found only for $c = -71/48$, the quadratic 6-cycle of which
is given in (\ref{eq:6-cycle}).

\subsubsection{Enumerating $x \in \U$ on $C_0(5)$}

Now we specialize to $N = 5$. In this case, recall from
(\ref{eq:c0(5)}) and (\ref{eq:c-in-xy}) that
\[
y^2 = f(x) = x^6 + 8x^5 + 22x^4 + 22x^3 + 5x^2 + 6x + 1,
\]
and $c$ is given in terms of $x$ and $y$ by
\[
c = \frac{g(x)}{2(P_0(x) - P_1(x) y)},
\]
where $P_0(x)$, $P_1(x)$, and $g(x)$ are defined in
(\ref{eq:poly-defs}).

Note that quadratic points on $C_1(5)$ always correspond to quadratic
points on $C_0(5)$, so we may enumerate quadratic points on $y^2 =
f(x)$, verify if the corresponding $c$ is rational, and for $c$
obtained this way, directly check if $\phi_c(z)$ has quadratic
periodic points of exact period 5 by factorizing $\Phi_5(z, c)$.

To enumerate quadratic points on $C_0(5)$, we enumerate $x \in \U$ as
described in Subsubsection~\ref{sssec:enum}, and for each specific $x
\in \Q(\sqrt{d})$, compute $f(x)$ and test if $f(x)$ is a square in
$\Q(\sqrt{d})$. The square test in $\Q(\sqrt{d})$, which is equivalent
to factorizing $T^2 - f(x)$ in $\Q(\sqrt{d})[T]$, is a rather
expensive task. However, observe that if we denote by $N$ the norm in
$\Q(\sqrt{d})$,\footnote{%
  The norm $N$ in $\Q(\sqrt{d})$, where $d$ is a squarefree integer,
  is defined as $N(a + b \sqrt{d}) = a^2 - b^2 d$ for $a, b \in \Q$.}
then a necessary condition for $f(x)$ to be a square in $\Q(\sqrt{d})$
is that $N(f(x))$ is a square in $\Q$. This turns out to be a very good
criterion that filters out most of the noise, and only leaves a very
limited number of false positives, which we can rule out by running
the expensive yet definitive test.

On a technical note, we implemented this same algorithm in three
languages --- C++ with the FLINT library, Mathematica, and Sage (see
\texttt{cpp/enum-x.cpp}, \texttt{mma/enum-x.nb}, and
\texttt{sage/enum-x.sage}). On our test machine (see
\texttt{misc/env.md} for details about the environment, including
software versions), it turned out that Mathematica was consistently
about 20 times faster than Sage, and C++ with FLINT (using the
\texttt{fmpzxx} and \texttt{fmpqxx} classes) was consistently about 50
times faster than Mathematica, from which we conclude that FLINT is
the performance-wise winner when it comes to multiprecision rational
arithmetic.

With the C++ program we were able to enumerate squarefree $d \in
\set{-100, \dots, 100}$ and $x \in \Q(\sqrt{d})$ up to height 30, and
for values of $d$ of special interest, i.e., $d = -87$ and $d = 33$,
up to height 100. Only eight quadratic points on $C_0(5)$ (excluding
all the rational points already listed in \cite{MR1480542}) were
found, which produces only two rational values of $c$ in total, $-2$
and $-4/3$, neither of which yields any quadratic periodic points of
exact period 5. This once again provides evidence that no such
periodic points exist.

Detailed results can be found in \texttt{misc/enum-x-results.txt}.

\subsection{Rational points on $P(a, b) = 0$}
\label{subsec:pab-ratpoint}

Recall that if we are able to find all the rational points on the
genus-11 curve $C_P$, defined by $P(a, b) = 0$ (see (\ref{eq:pab})),
then we can find all $c \in \Q$ such that $\phi_c(z) = z^2 + c$ has
periodic points in $\U$ of exact period 5.

Since we are looking for rational points, we can simply enumerate $b
\in \Q$, and solve the polynomial equation for rational $a$. We
enumerated $b$ up to height $10000$ (see \texttt{mma/pab-ratpoint.nb})
and only found five rational points, all with very small heights:
\[
(3, 0),\, (0, 0),\, (4, 1/3),\, (1, 8/3),\, (6, 9),
\]
and, as have been pointed out before, none of them produce any
quadratic periodic points of exact period 5. This is yet another
important piece of evidence to support our conjecture that no such
points exist.

\subsection{Quadratic extensions containing quadratic 4-cycles}
\label{subsec:d-distribution}

For the $N = 4$ case, we already have complete knowledge about all $c
\in \Q$ with quadratic periodic points of exact periodic 4, and have
an explicit parametrization for those periodic points, i.e.,
\[
\frac{t^4 - t^2 + \sqrt{(t^4 - 1)(t^2 + 2t - 1)}}{2t(t^2 - 1)}
\]
for $t \in \Q$, as seen in (\ref{eq:4-param}).\footnote{%
  Since $\phi_c(z) = z^2 + c$ permutes 4-cycles, every periodic point
  in some 4-cycle can be viewed as the $z_1$ in (\ref{eq:4-param}).}
In particular, we know that the quadratic extensions to $\Q$
containing quadratic 4-cycles are given by $\Q(\sqrt{d})$, where $d =
(t^4 - 1)(t^2 + 2t - 1)$ for some $t \in \Q$.

We naturally ask about the distribution of such quadratic extensions
within all possible quadratic extensions, or the distribution of such
$d$ in $\Q$. Can we obtain all rational numbers in this way?

Since $\Q(\sqrt{d})$ stays the same when we multiply $d$ by a rational
square, we actually only care about $d$ up to rational
squares. Therefore, the problem can be formulated purely inside $\Z$
instead: let $SF$ be the set of all squarefree integers, then what can
we say about the distribution of $d$ within $SF$, where $d$ is the
squarefree part of $(p^4 - q^4)(p^2 + 2pq - q^2)$, with $p$ and $q$
ranging over $\Z$?

With this new formulation in $\Z$, it becomes easier to do
computational experiments. We enumerate $p$ and $q$ up to some height
$H$ (in this case, the height for an integer is simply its absolute
value), and let $D_H$ be
\[
\begin{aligned}
  D_H = \{ d \in \Z \mid\ &
  \text{$d$ is the squarefree part of $(p^4 - q^4)(p^2 + 2pq - q^2)$},
  \\
  & \text{where $-H \le p, q \le H$}\}.
\end{aligned}
\]
We expect $D_H$ to cover most, if not all, of the squarefree integers
of reasonably small height (compared to $H$) that occur in $D_\infty$,
which we define as the set of all $d$ obtainable in this way (with no
bound on the height of $p$ and $q$). To make this intuition precise,
we take another positive integer $H'$, let $R_{H'}$ be the range
$[-H', H']$ and consider the size of the intersection $D_H \cap
R_{H'}$, and compared it to the size of the intersection $SF \cap
R_{H'}$; that is, the number of squarefree integers obtainable in the
range $R_{H'}$ (limited by the height of $p$ and $q$ that we enumerate
to), compared to the total number of squarefree integers in this
range.

We coded this up (see \texttt{mma/squarefree.nb}) and obtained the
results in Table~\ref{tab:squarefree}.

\begin{table}[hbtp]
  \centering
  \begin{tabular}{| c | c | c | c |}
    \hline
    $H$ & $H'$ & $\# D_H \cap R_{H'}$ & $\# SF \cap R_{H'}$ \\ \hline
    100 & 100 & 9 & 123 \\ \hline
    100 & 1000 & 20 & 1217 \\ \hline
    100 & 10000 & 53 & 12167 \\ \hline
    100 & 100000 & 131 & 121589 \\ \hline
    1000 & 100 & 9 & 123 \\ \hline
    1000 & 1000 & 20 & 1217 \\ \hline
    1000 & 10000 & 55 & 12167 \\ \hline
    1000 & 100000 & 140 & 121589 \\ \hline
  \end{tabular}
  \caption{Distribution of $d = (p^4 - q^4)(p^2 + 2pq - q^2)$ up to
    squares.}
  \label{tab:squarefree}
\end{table}

These results clearly suggests that the squarefree integers obtainable
this way is very sparse out of all squarefree integers; the density of
$D_\infty$ in $SF$ should be zero. However, It is not clear how to
approach this density theoretically.

In fact, it seems nontrivial even to prove that certain squarefree
integers are not obtainable. For instance, the nine numbers that we
found in the range $[-100, 100]$ are $-15$, $0$, $1$, $10$, $15$,
$41$, $51$, $70$, and $93$.\footnote{%
  Here we are counting 0 as the squarefree part of itself, and 1 as
  the squarefree part of perfect squares. Since we are mainly
  interested about the asymptotic behavior, these only special cases
  do not matter.}
It is almost certain empirically that, for instance, 2 should never
occur. To prove this, we essentially need to prove that the
Diophantine equation
\[
(p^4 - q^4)(p^2 + 2pq - q^2) = 2 k^2
\]
is not solvable in $\Z$. This seems very hard. It will be really
interesting if elements of $D_\infty$ (the squarefree integers that
are obtainable as squarefree parts of $(p^4 - q^4)(p^2 + 2pq - q^2)$)
have some special property that makes them stand out.

%%% Local Variables:
%%% TeX-master: "report"
%%% End:
