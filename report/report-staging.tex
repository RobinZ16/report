\section{6-Cycles in Quadratic Fields}

\begin{conjecture}[Galois Conjugacy]
  \label{conj:gc}
  Let $\{x_1, x_2, x_3, x_4, x_5, x_6\}$ be a $6$-cycle over a
  quadratic field $K$ for $\phi(x) = x^2 + c$ with $c \in \Q$. Then
  $x_4 = \overline{x_1}$.
\end{conjecture}

\begin{corollary}[Conditional Classification of 6-Cycles over Quadratic Fields]
  \label {cor:6-cycles}
  Conjecture \ref{conj:gc} and the weak BSD Conjecture, imply that the
  only 6-cycle over quadratic fields is in $K = \Q(\sqrt{33})$ and is
  given by:
  \[
  \phi(x) = x^2 - \frac{71}{48}, x_1 = -1 + \frac{\sqrt{33}}{12}
  \]
\end{corollary}

\begin{proof}
  Let $\{x_1, x_2, x_3, x_4, x_5, x_6\}$ be a $6$-cycle over a
  quadratic field $K$ and define $t = x_1 + x_2 + x_3 + x_4 + x_5 +
  x_6$ to be the trace of the orbit. By Conjecture \ref{conj:gc}, $x_4
  = \overline{x_1}, x_5 = \overline{x_2}, x_6 = \overline{x_3}$
  . Thus, we have
  \[
  t = x_1 + x_2 + x_3 + x_4 + x_5 + x_6 = x_1 + x_2 + x_3 +
  \overline{x_1} + \overline{x_2} + \overline{x_3} \in \Q
  \]
  Therefore, each 6-cycle can be represented as a $\Q$-point on $C_0
  (6)(\Q)$ via its trace. Using Stoll's characterization of the
  $\Q$-points of $C_1(6)$(cite), we know that the 6-cycle given by $K
  = \Q(\sqrt{33}), \phi(x) = x^2 - \frac{71}{48}, x_1 = -1 +
  \frac{\sqrt{33}}{12}$ is the only 6-cycle over a quadratic field and
  not over $\Q$.  Stoll finds, using a proof conditional on the weak
  BSD Conjecture, that there are no 6-cycles over $\Q$ either.
\end{proof}

\begin{lemma}
	Let $\{x_1, \ldots, x_{N}\}$ be an exact $N$-cycle defined over
	a Galois number field K, with N divisible by	$d = [K:\Q ]$. Then either: \\
	\begin{itemize}
	\item $x_{\frac{iN}{d}+1} = \sigma(x_1)$ for some nontrivial $\sigma \in
	\gal(K/\Q)$, $i \in \Z$
	\item $\{x_1, \ldots, x_{N}\} \cap \{\tau(x_1), \ldots, \tau(x_{N})\} =
	\emptyset$, $\forall \tau \in \gal(K/\Q)$ nontrivial.
	\end{itemize}
\end{lemma}

\begin{proof}
	Assume that neither of the above are true, so $x_{\frac{N}{d}+1}
	\neq \sigma(x_1)$ $\forall \sigma \in \gal(K/\Q)$ and $\{x_1, \ldots
	, x_{N}\} \cap \{\tau(x_1), \ldots, \tau(x_{N})\} \neq \emptyset$ $
	\forall \tau \in \gal(K/\Q)$. Then $\exists x_j \in \{x_1, \ldots,
	x_{N}\} \cap \{\tau(x_1), \ldots, \tau(x_{N})\}$ for some $\tau
	\in \gal(K/\Q)$. Applying $\phi$ an appropriate number of times
	allows us to renumerate the cycle	such that $x_1$ is in the
	intersection. Thus, $x_k = \tau(x_1)$ for some $1 \leq k \leq N$
	and therefore $\tau \equiv \phi^{k-1}$ on the N-cycle (because Galois
	conjugation commutes with $\phi$). Since $K$ is Galois,
	$\tau^d = Id$ and so:
	\[
		x_d = \tau^d(x_d) = \phi^{dk-d}(x_d) = x_{dk}
	\]
	Notice that the application of $\phi^{N-(d-1)}$ yields $x_1 =
	x_{dk-(d-1)} = x_{d(k-1) + 1}$. Notice that this produces a
	contradiction if $d(k-1) + 1 \not\equiv 1$ (modulo $N$): if $l$
	is the representative of $d(k-1) + 1$ modulo N, then $\{x_1, \ldots
	, x_l\}$ gives an $l$-cycle with $l < N$. Thus, it must be that
	$d(k-1) + 1 \equiv 1$ (modulo $N$). The only such $k$ are given
	by $k = \frac{lN}{d} + 1$ for some $l \in \Z$, so $x_{\frac{lN}{d}
	+ 1} = x_k = \tau(x_1)$.	However, we already have that
	$x_{\frac{iN}{d}+1} \neq \sigma(x_1)$ $\forall \sigma \in \gal(K/\Q)$,
	$i \in \Z$. This gives a contradiction.
\end{proof}

\begin{remark}[N = 6 case]
	Stoll conditionally proved that there are no rational points with
	exact period 6 assuming the weak form of the BSD Conjecture(cite).
	This allows us to use this lemma to conditionally prove Conjecture
	\ref{conj:gc} and thus Corollary \ref{cor:6-cycles}. All that
	remains to be shown is that$\{x_1, \ldots, x_{N}\} \cap \{\overline{
	x_1}, \ldots, \overline{x_{N}}\} = \emptyset$ is impossible.
\end{remark}

In general, we conjecture that:
\begin{conjecture}
	Let $\{x_1, \ldots, x_{N}\}$ be an exact $N$-cycle defined over
	number field K, with N divisible by	$d = [K:\Q ]$. Then it is
	never the case that $\{x_1, \ldots, x_{N}\} \cap \{\tau(x_1), \ldots
	, \tau(x_{N})\} = \emptyset$, $\forall \tau \in \gal(K/\Q)$. In
	particular, this implies that $x_{\frac{N}{d}+1} =\sigma(x_1)$ for
	some $\sigma \in \gal(K/\Q)$. Therefore the rational points on
	$C_0(N)$ are sufficient to find points in $K$ of exact period $N$.
\end{conjecture}

\section{Key Results}

%Entire section to be expanded upon and written more nicely%
For N = 5, we took $P_{c,q}$ and did long division with general
quadratic $q^2 + aq + b$ and then took the resultant of the linear and
constant remainders to get a polynomial equation in $a$ and $b$. SAGE
computed that the genus of the resulting curve is 11. Thus there are
finitely many rational triples $(a,b,c)$ that satisfy both
equations. Thus there are finitely many quadratic points in $C_0(5)$
and thus finitely many quadratic points of exact period 5.
\begin{theorem}
  There are finitely many points in quadratic fields with exact period
  5 for $\phi(x) = x^2 + c$ with $c \in \Q$.
\end{theorem}

A result that follows directly from the parametrization of 4-cycles
given by Morton (cite):
\begin{theorem}
  There are infinitely many points in quadratic fields with exact
  period 4 for $\phi(x) = x^2 + c$ with $c \in \Q$.
\end{theorem}

Plugging in the $c$ from Morton's parametrization and factoring
$\Phi_4(x,c)$ with respect to $x$ yields:
\begin{lemma}
  For $c$ such that $\phi$ has 4-cycles over quadratic fields, the
  degree 12 polynomial $\Phi_4(x,c)$ must factor as $\{2,2,8\}$ (list
  of degrees of factors, with the 8th degree factor not necessarily
  irreducible).
\end{lemma}

Using Panraksa's theorem that $\Phi_4(x,c)$ cannot factor as
$\{2,2,2,2,4\}$, we obtain the following result:
\begin{theorem}
  For each $c \in \Q$, there is at most one 4-cycle over quadratic
  fields.
\end{theorem}
