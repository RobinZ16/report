\section{Preliminary notions and tools}

\subsection{Genus of curves}

First we introduce the normalization and genus of a projective curve.

\begin{definition}[normalization of a singular curve]
  %%% TODO
\end{definition}

\begin{remark}
  \label{rem:ratpoint-normalization}
  The number of rational points on the normalization of $C$ is greater
  than or equal to the number of rational points on $C$.
\end{remark}

There are two different notions of genus, the \emph{arithmetic genus}
and the \emph{geometric genus}.

\begin{definition}
  Let $X$ be a projective curve over a field $K$. The \emph{arithmetic
    genus} of $X$ is defined to be the integer
  \[
  p_a(X) = 1 - \chi_K(\O_X),
  \]
  where $\chi$ denotes the Euler-Poincar\'e characteristic.
\end{definition}

\begin{definition}
  Let $Y$ be a smooth projective curve over a field $K$. The
  \emph{geometric genus} of $Y$ is defined to be
  \[
  p_g(Y) = \dim_K H^0(Y, \omega_{Y/K}),
  \]
  where $H^0$ denotes the zeroth cohomology group.
\end{definition}

It turns out that for $X$ smooth and geometrically connected, we have
$p_a(X) = p_g(X)$ (see Subsection 7.3.2 of \cite{MR1917232}). In
particular, for the normalization of a curve $C$, the arithmetic genus
and the geometric genus coincide. From now on we will simply use
$g(C)$ to denote the genus of the normalization of the curve $C$.

\begin{remark}
  It is worth noting that $g(C)$ can be computed reasonably
  efficiently, provided that we have an explicit equation for $C$. For
  instance, if the polynomial equation $p(x, y) = 0$ defines an affine
  model of $C$, then we may feed the polynomial $p(x, y)$ into some
  computer algebra system to obtain $g(C)$. Computer algebra systems
  capable of computing the genus of an algebraic curve include, but
  are not limited to, Magma and Sage (using its interface to
  Singular).
\end{remark}

The reason that we are interested in the genus of a curve is due to
the following theorem.

\begin{theorem}[Faltings, 1983 \cite{MR718935}]
  Let $C$ be a non-singular algebraic curve of genus $g$ over
  $\Q$. Then the set of rational points $C(\Q)$ on $C$ satisfy:
  \begin{enumerate}
  \item If $g = 0$, then either $C(\Q) = \es$, or $|C(\Q)| = \infty$,
    in which case $C$ is isomorphic over $\Q$ to the projective line
    $\P^1$;

  \item If $g = 1$, then either $C(\Q) = \es$, or $C$ is an
    \emph{elliptic curve}, in which case $C(\Q)$ is a finitely
    generated abelian group;

  \item If $g \ge 2$, then $|C(\Q)| < \infty$.
  \end{enumerate}
\end{theorem}

Faltings' theorem shows that the genus of a curve gives us information
about the number of rational points on it. In particular, there are
only finitely many rational points on a high genus curve (from now on
we refer to a curve with genus $\ge 2$ as a \emph{high genus
  curve}). Note that this works for any high genus curve (singular or
not) by Remark~\ref{rem:ratpoint-normalization}.

\begin{remark}
  Faltings' theorem is ineffective, since it does not provide an
  explicit bound on the number of rational points; nor does it present
  a method to find such a bound.

  Under certain circumstances, there are methods to find an explicit
  bound (usually not sharp), most notably Chabauty and Coleman's
  method \cite{MR808103}. Chabauty and Coleman underlies most triumphs
  in this field.
\end{remark}

\subsection{Jacobian of curves}

%%% TODO

\subsection{Resultant of polynomials}

Another tool that we need is the resultant of two polynomials.

\begin{proposition}
  Let
  \[
  \begin{gathered}
    A(X) = a_0 X^n + a_1 X^{n-1} + \cdots + a_n,\\
    B(X) = b_0 X^m + b_1 X^{m-1} + \cdots + b_m
  \end{gathered}
  \]
  be polynomials of degrees $n$ and $m$ with coefficients
  in a field $K$. There exists a polynomial
  \[
  \res(a_0, \dots, a_n, b_0, \dots, b_m) \in \Z[a_0, \dots, a_n, b_0,
  \dots, b_m],
  \]
  in the coefficients of $A$ and $B$, called the \emph{resultant of
    $A$ and $B$}, with the following properties:
  \begin{enumerate}
  \item $\res(A, B) = 0$ if and only if $A$ and $B$ have a comman root
    in $\bar{K}$;

  \item If $a_0 b_0 \ne 0$ and if we factor $A$ and $B$ as
    \[
    A = a_0 \prod_{i=1}^n (X - \alpha_i),\,
    B = b_0 \prod_{i=1}^m (X - \beta_j),
    \]
  \end{enumerate}
  then
  \[
  \res(A, B) = a_0^m b_0^n \prod_{i=1}^n \prod_{j=1}^m (\alpha_i -
  \beta_j).
  \]
\end{proposition}

For a proof of this proposition, see Section~2.4 of \cite{MR2316407}.

\begin{remark}
  Since $\res(A, B)$ is a polynomial in the coefficients of $A$ and
  $B$, when the dimensions $n$ and $m$ are reasonably small, $\res(A,
  B)$ can be computed efficiently by computer algebra
  systems. Computer algebra systems capable of computing the resultant
  of two polynomials include, but are not limited to, Magma,
  Mathematica, and Sage.

  As the resultant can be computed efficiently, it is a good way to
  reduce the number of variables in a set of simultaneous polynomial
  equations. For instance, if we have two polynomial equations $A(X,
  Y) = 0$ and $B(X, Y) = 0$, then we may take the resultant with
  respect to $X$ (viewing $X$ as the variable in the above
  definition). Denote it by $\res_X(A, B)$ (we will continue to use
  the $\res_X$ notation when there are more than one variables). Then
  by Property~(1) of the resultant, there is an $X$ solving both $A$
  and $B$ if and only if $Y$ solves $\res_X(A, B)$, which is a
  polynomial in $Y$, since $A$ and $B$ both have coefficients in
  $K[Y]$ when viewed as polynomials in $X$. Therefore, we reduced two
  equations in two unknowns to one equation in one unknown. Similar
  argument applies when there are more variables.
\end{remark}

%%% Local Variables:
%%% TeX-master: "report"
%%% End:
