\section{Directions from here}
\label{sec:map}

It has been mentioned at the end of Subsection~\ref{subsec:general}
that it might be worthwhile to check
Conjecture~\ref{cj:galois-conjugate} for finite fields, and in case it
is true for infinitely many characteristics, some Chebotarev
density-type argument might yield the full conjecture over number
fields.

On the other hand, when we were trying to generalize the $N = 5$ proof
to accommodate the case where $c$ is also allowed to be quadratic, we
encountered several intrinsic difficulties in our method that disabled
the generalization. Nevertheless, we found a remotely related problem
that is interesting on its own right.

\begin{question}
  \label{q:z-c-same-extension}
  Recall the polynomial $\Phi_N(z, c)$ from
  Subsection~\ref{subsec:model-general}. Is it true that for $z, c \in
  \U$ solving $\Phi_N(z, c) = 0$, $z$ and $c$ must be in the same
  quadratic extension, i.e., $\Q(z, c)$ has degree at most two over
  $\Q$?
\end{question}

For a fixed $N$ this may be reduced to a finite computation as
follows. Suppose not, then we can write $z = a_1 + \sqrt{b_1}$ and $c
= a_2 + \sqrt{b_2}$, where $a_1, b_1, a_2, b_2$ are rational,
$[\Q(\sqrt{b_1}) : \Q] = [\Q(\sqrt{b_2}) : \Q] = 2$, and
$\Q(\sqrt{b_1}) \ne \Q(\sqrt{b_2})$. Substituting these into
$\Phi_N(z, c)$ we have $\Phi_N(z, c) \in \Q(\sqrt{b_1}, \sqrt{b_2})$,
where $\Q(\sqrt{b_1}, \sqrt{b_2})$ has a basis $1, \sqrt{b_1},
\sqrt{b_2}, \sqrt{b_1 b_2}$ over $\Q$. Since $\Phi_N(z, c) = 0$, the
four coefficients with respect to the above basis must all be zero, so
we obtain four polynomial equations in four unknowns $a_1, b_1, a_2,
b_2$. Taking the resultant three times, we end up with a polynomial
equation in one variable (this polynomial could be gigantic even for
small $N$), and we only need to check for its finitely many rational
roots. If it turns out that we have no rational solutions satisfying
our requirements, then we can conclude that the original assumption is
false, and hence $z$ and $c$ must lie in the same quadratic extension.

This method might be computational infeasible as $N$ grows large, but
it is useful to at least test the conjecture for small $N$.

A more general question to ask is which polynomials $P$ in $\Q[X, Y]$
have the property that if $X, Y \in \U$ solve $P(X, Y) = 0$, then $X,
Y$ lie in the same extension. (This is obviously not always true; for
instance, $X^2 + Y^2 - 5$ is irreducible, and admits a solution $(X,
Y) = (\sqrt{2}, \sqrt{3})$.) In particular, it is interesting to see
if this holds true for general dynatomic polynomials (see Section~4.1
of \cite{MR2316407} for the definition of dynatomic polynomials).

%%% Local Variables:
%%% TeX-master: "report"
%%% End:
