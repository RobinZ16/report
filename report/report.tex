\documentclass{amsart}


%%%%%%%%%%%%%%%%%%%%%%%%%%%%%% packages %%%%%%%%%%%%%%%%%%%%%%%%%%%%%%

% Here are some common packages used in almost all documents.
\usepackage{amssymb}
\usepackage{graphicx}
\usepackage{color}
\usepackage{hyperref}

\usepackage{enumerate} % customizable styles for counter printing


%%%%%%%%%%%%%%%%%%%%%%% amsthm theorem styles %%%%%%%%%%%%%%%%%%%%%%%%

% Current styles are based on suggestions by the amsthm package
% authors.
\theoremstyle{plain}
\newtheorem{theorem}{Theorem}[section]
\newtheorem{lemma}[theorem]{Lemma}
\newtheorem{proposition}[theorem]{Proposition}
\newtheorem{corollary}[theorem]{Corollary}
\newtheorem{claim}[theorem]{Claim}

\theoremstyle{definition}
\newtheorem{definition}[theorem]{Definition}
\newtheorem{example}[theorem]{Example}
\newtheorem{conjecture}[theorem]{Conjecture}

\theoremstyle{remark}
\newtheorem{remark}[theorem]{Remark}
\newtheorem{note}[theorem]{Note}
\newtheorem{case}{Case}


%%%%%%%%%%%%%%%%%%%%%%%%% custom definitions %%%%%%%%%%%%%%%%%%%%%%%%%

% Since we are collaborating, we need to be careful with personal
% shortcuts to avoid clash.
\newcommand{\C}{\mathbb{C}}
\newcommand{\F}{\mathbb{F}}
\newcommand{\N}{\mathbb{N}}
\renewcommand{\O}{\mathcal{O}}
\renewcommand{\P}{\mathbb{P}}
\newcommand{\Q}{\mathbb{Q}}
\newcommand{\U}{\mathcal{U}}
\newcommand{\Z}{\mathbb{Z}}

\renewcommand{\l}{\lambda}

\newcommand{\es}{\emptyset}
\newcommand{\ol}{\overline}

\newcommand{\gal}{\mathrm{Gal}}
\newcommand{\preper}{\mathrm{PrePer}}
\newcommand{\res}{\mathrm{Res}}

\newcommand{\set}[1]{\left\{#1\right\}}
\newcommand{\tup}[1]{\left<#1\right>}


\title{On quadratic periodic points of quadratic polynomials}
\author{Zhiming Wang}
\author{Robin Zhang}
\date{\today}

\begin{document}

\begin{abstract}
  Never write an abstract before you finish the whole paper.
\end{abstract}

\maketitle

\tableofcontents

\section{Introduction}
\label{sec:bg}

Our entire research project has its roots in the study of the dynamics
of morphisms over projective spaces, specifically the periodic points
(or more generally, preperiodic points) of such morphisms. One
profound conjecture in this field is the Uniform Boundedness
Conjecture, put forth by Morton and Silverman in
1994~\cite{MR1264933}, which bounds the number of preperiodic points
of a morphism in terms of the the degree of the base field (over the
rational field $\Q$), the dimension of the projective space, and the
degree of the morphism. We will define the necessary objects leading
up to the Uniform Boundedness Conjecture, briefly describe the status
of the conjecture, and discuss the specific cases that we will
concentrate on. At the end of this section, our major results will be
presented, along with a few open conjectures.

\begin{definition}
  The \emph{projective space} $\P^N(K)$ of dimension $N$ over a number
  field $K$ is the space of equivalence classes $K^{N+1}/\sim$ in
  $K^{N+1}$, where the equivalence relation $\sim$ is given by
  \[
  (x_0, \dots, x_N) \sim (y_0, \dots, y_N)
  \]
  if and only if $x_i y_j = x_j y_i$ for all $0 \le i, j \le N$. We
  denote such an equivalence class by $[x_0, \dots, x_N]$, where
  $(x_0, \dots, x_N)$ is any element of the equivalence class.
\end{definition}

\begin{definition}
  Given a number field $K$, a \emph{rational map} of degree $d$
  between projective spaces $\P^N(\ol{K})$ and $\P^M(\ol{K})$ is a
  map
  \[
  \begin{gathered}
    \phi: \P^N(\ol{K}) \to \P^M(\ol{K})\\
    P \mapsto [f_0(p), \dots, f_M(p)],
  \end{gathered}
  \]
  where $p \in P$ is any element of the equivalence class $P$, and
  $f_0, \dots, f_M \in \ol{K}[X_0, \dots, X_N]$ are homogeneous
  polynomials of degree $d$ with no common factors. (Well-definedness
  is guaranteed by homogeneity.)

  If all the coefficients of $f_0, \dots, f_M$ lie in $K$, we say that $\phi$ is \emph{defined over} $K$.

  The rational map $\phi$ is \emph{defined at} $P$ if at least one of
  the values $f_0(p), \dots, f_M(p)$ is nonzero.

  The rational map $\phi$ is called a \emph{morphism} if it is defined
  everywhere in $\P^N(\ol{K})$, or equivalently, if the only solution
  in $\ol{K}$ to the simultaneous equations
  \[
  f_0(X_0, \dots, X_N) = \cdots = f_M(X_0, \dots, X_N) = 0
  \]
  is the trivial solution $X_0 = \dots, X_N = 0$.
\end{definition}

\begin{definition}
  Given any morphism $\phi: \P^N \to \P^N$, the \emph{preperiodic
    points of $\phi$} in some space $\P^N(K)$ is the collection of
  points in $\P^N(K)$ with finite forward orbits, or equivalently,
  \[
  \preper(\phi, \P^N(K)) = \set{P \in \P^N(K) : \text{exist $n_1, n_2
      \in \N$ such that $\phi^{n_1}(P) = \phi^{n_2}(P)$}}.
  \]
\end{definition}

Now we are ready to state the Uniform Boundedness Conjecture.

\begin{conjecture}[Morton-Silverman, 1994 ~\cite{MR1264933}]
  Fix integers $d \ge 2$, $N \ge 1$, and $D \ge 1$. There is a
  constant $C(d, N, D)$ such that for all number fields $K/\Q$ of
  degree at most $D$ and all morphisms $\P^N \to \P^N$ of degree $d$
  defined over $K$,
  \[
  \#\preper(\phi, \P^N(K)) \ge C(d, N, D).
  \]
\end{conjecture}

Very little is known about this conjecture. In fact, even the simplest
case $(d, N, D) = (2, 1, 1)$ is not known, that is, the problem of
bounding the number of rational preperiodic points of degree-2
rational morphisms. If we specialize to degree-2 polynomials (i.e.,
maps of the type $[X_0, X_1] \mapsto [a X_0^2 + b X_0 X_1 + c X_1^2,
X_1^2]$, conveniently written as $a z^2 + b z + c$), note that $a z^2
+ b z + c$ can be reduced to the form $z^2 + c'$ by linear
conjugation, which preserves the sizes of orbits, so it suffices to
consider degree-2 polynomials of the type $\phi_c(z) = z^2 + c$. In
this case of $\phi_c(z) = z^2 + c$, the uniform bound is again not
known, but there are partial results when we further stipulate the
period.

One can easily show that there are one-parameter families of
$c$-values for which $\phi_c(z)$ has rational periodic point(s) of
exact period 1, 2, or 3. It was shown by Morton~\cite{MR1665198} that
$\phi_c$ cannot have rational periodic points of exact period 4. The
non-existence result was extended to exact period 5 by Flynn, Poonen,
and Schaefer~\cite{MR1480542}, and to exact period 6 (conditional on
the Birch and Swinnerton-Dyer conjecture for a specific abelian
variety) by Stoll~\cite{MR2465796}.

We are interested in generalizing the aforementioned results for
quadratic polynomials and specific small periods to quadratic number
fields, i.e., the $D = 2$ case. For a more detailed discussion of
other takes on the Uniform Boundedness Conjecture, along with a list
of references, see Section 3.3 of \cite{MR2316407}.

From now on, unless otherwise stated, the dimension of the projective
space will always be 1, and we reserve the letter $N$ for the exact
period of periodic points.

Below are our major results.

For $N = 4$, our main result is that 

\begin{theorem}
  For each $c \in \Q$, there is at most one 4-cycle with points
	defined over quadratic fields.
\end{theorem}

For $N = 5$, we have

\begin{theorem}
  \label{th:n=5-finite}
  There are finitely many rational values $c$ such that $\phi_c(z) =
  z^2 + c$ has a periodic point of exact period $N$ in some quadratic
  extension to $\Q$.
\end{theorem}

In fact, all empirical evidence points to the following stronger
conjecture:

\begin{conjecture}
  \label{cj:n=5-zero}
  There are no rational values $c$ such that $\phi_c(z) = z^2 + c$ has
  a periodic point of exact period $N$ in some quadratic extension to
  $\Q$.
\end{conjecture}

A general result we produced is that

\begin{theorem}
	Let $\phi_c(z) = z^2 + c$ with $c \in \Q$. Let $\{x_1, \ldots, x_{N}\}$ be
	an exact $N$-cycle defined over a Galois number field K, with N 
	divisible by	$d = [K:\Q ]$. Then either: \\
	\begin{itemize}
	\item $x_{\frac{iN}{d}+1} = \sigma(x_1)$ for some nontrivial $\sigma \in
	\gal(K/\Q)$, $i \in \Z$
	\item $\{x_1, \ldots, x_{N}\} \cap \{\tau(x_1), \ldots, \tau(x_{N})\} =
	\emptyset$, $\forall \tau \in \gal(K/\Q)$ nontrivial.
	\end{itemize}
\end{theorem}

The above theorem has some interesting consequences if the our conjecture
that the second case is never true (although we do not have many examples,
all of our numerical evidence falls under the first category). If the first
case always holds, then the $x_i$'s produce rational points on $C_0(N)$
because the trace $t = \sum\limits_{i=1}^N x_i$ is always rational. Thus,
the search for these periodic points is reduced to finding rational points
on $C_0(N)$.

Other conjectures can be found in Section~\ref{sec:comp},
``Computational efforts, confirmations, and observations'', and
Section~\ref{sec:map}, ``Directions from here''.

%%% Local Variables:
%%% TeX-master: "report"
%%% End:


\section{Background}

In this section, we introduce some preliminary notions and tools
necessary for further development of the subject. Readers who are
already familiar with the concepts introduced could safely skip this
section.

\subsection{Genus of curves}

First we introduce the normalization and genus of a projective curve.

Normalization is a widely used technique in algebraic geometry to
desingularize algebraic curves (and more generally, schemes). For each
curve $C$ with finitely many singularities, there is a canonical way
to construct a nonsingular curve $\tilde{C}$, called the
\emph{normalization of $C$}, along with a surjection $\varphi:
\tilde{C} \to C$, called the \emph{normalization morphism}, such that
$\varphi$ is an isomorphism outside the finitely many singularities of
$C$. For a rigorous treatment of normalization, see Subsection 4.1.2
of \cite{MR1917232}.

\begin{remark}
  \label{rem:ratpoint-normalization}
  The number of rational points on the $\tilde{C}$ is greater
  than or equal to the number of rational points on $C$, since for
  each nonsingular rational point on $C$ there is exactly one
  corresponding rational point on $\tilde{C}$, and for each singular
  rational point on $C$ there is one or more corresponding rational
  points ``upstairs''.
\end{remark}

There are two different notions of genus, the \emph{arithmetic genus}
and the \emph{geometric genus}.

\begin{definition}
  Let $X$ be a projective curve over a field $K$. The \emph{arithmetic
    genus} of $X$ is defined to be the integer
  \[
  p_a(X) = 1 - \chi_K(\O_X),
  \]
  where $\chi$ denotes the Euler-Poincar\'e characteristic.
\end{definition}

\begin{definition}
  Let $Y$ be a smooth projective curve over a field $K$. The
  \emph{geometric genus} of $Y$ is defined to be
  \[
  p_g(Y) = \dim_K H^0(Y, \omega_{Y/K}),
  \]
  where $H^0$ denotes the zeroth cohomology group.
\end{definition}

It turns out that for $X$ smooth and geometrically connected, we have
$p_a(X) = p_g(X)$ (see Subsection 7.3.2 of \cite{MR1917232}). In
particular, for the normalization of a curve $C$, the arithmetic genus
and the geometric genus coincide. From now on we will simply use
$g(C)$ to denote the genus of the normalization of the curve $C$.

\begin{remark}
  It is worth noting that $g(C)$ can be computed reasonably
  efficiently, provided that we have an explicit equation for $C$. For
  instance, if the polynomial equation $p(x, y) = 0$ defines an affine
  model of $C$, then we may feed the polynomial $p(x, y)$ into some
  computer algebra system to obtain $g(C)$. Computer algebra systems
  capable of computing the genus of an algebraic curve include, but
  are not limited to, Magma and Sage (using its interface to
  Singular).
\end{remark}

The reason that we are interested in the genus of a curve is due to
the following theorem.

\begin{theorem}[Faltings, 1983 \cite{MR718935}]
  Let $C$ be a non-singular algebraic curve of genus $g$ over
  $\Q$. Then the set of rational points $C(\Q)$ on $C$ satisfy:
  \begin{enumerate}
  \item If $g = 0$, then either $C(\Q) = \es$, or $|C(\Q)| = \infty$,
    in which case $C$ is isomorphic over $\Q$ to the projective line
    $\P^1$;

  \item If $g = 1$, then either $C(\Q) = \es$, or $C$ is an
    \emph{elliptic curve}, in which case $C(\Q)$ is a finitely
    generated abelian group;

  \item If $g \ge 2$, then $|C(\Q)| < \infty$.
  \end{enumerate}
\end{theorem}

Faltings' theorem shows that the genus of a curve gives us information
about the number of rational points on it. In particular, there are
only finitely many rational points on a high genus curve (from now on
we refer to a curve with genus $\ge 2$ as a \emph{high genus
  curve}). Note that this works for any high genus curve (singular or
not) by Remark~\ref{rem:ratpoint-normalization}.

\begin{remark}
  Faltings' theorem is ineffective, since it does not provide an
  explicit bound on the number of rational points; nor does it present
  a method to find such a bound.

  Under certain circumstances, there are methods to find an explicit
  bound (usually not sharp), most notably Chabauty and Coleman's
  method \cite{MR808103}. Chabauty and Coleman underlies most triumphs
  in this field.
\end{remark}

\subsection{Resultant of polynomials}

Another tool that we need is the resultant of two polynomials.

\begin{proposition}
  \label{res}
  Let
  \[
  \begin{gathered}
    A(X) = a_0 X^n + a_1 X^{n-1} + \cdots + a_n,\\
    B(X) = b_0 X^m + b_1 X^{m-1} + \cdots + b_m
  \end{gathered}
  \]
  be polynomials of degrees $n$ and $m$ with coefficients
  in a field $K$. There exists a polynomial
  \[
  \res(a_0, \dots, a_n, b_0, \dots, b_m) \in \Z[a_0, \dots, a_n, b_0,
  \dots, b_m],
  \]
  in the coefficients of $A$ and $B$, called the \emph{resultant of
    $A$ and $B$}, with the following properties:
  \begin{enumerate}
  \item $\res(A, B) = 0$ if and only if $A$ and $B$ have a comman root
    in $\ol{K}$;

  \item If $a_0 b_0 \ne 0$ and if we factor $A$ and $B$ as
    \[
    A = a_0 \prod_{i=1}^n (X - \alpha_i),\,
    B = b_0 \prod_{i=1}^m (X - \beta_j),
    \]
  \end{enumerate}
  then
  \[
  \res(A, B) = a_0^m b_0^n \prod_{i=1}^n \prod_{j=1}^m (\alpha_i -
  \beta_j).
  \]
\end{proposition}

For a proof of this proposition, see Section~2.4 of \cite{MR2316407}.

\begin{remark}
  Since $\res(A, B)$ is a polynomial in the coefficients of $A$ and
  $B$, when the dimensions $n$ and $m$ are reasonably small, $\res(A,
  B)$ can be computed efficiently by computer algebra
  systems. Computer algebra systems capable of computing the resultant
  of two polynomials include, but are not limited to, Magma,
  Mathematica, and Sage.

  As the resultant can be computed efficiently, it is a good way to
  reduce the number of variables in a set of simultaneous polynomial
  equations. For instance, if we have two polynomial equations $A(X,
  Y) = 0$ and $B(X, Y) = 0$, then we may take the resultant with
  respect to $X$ (viewing $X$ as the variable in the above
  definition). Denote it by $\res_X(A, B)$ (we will continue to use
  the $\res_X$ notation when there are more than one variables). Then
  by Property~(1) of the resultant, there is an $X$ solving both $A$
  and $B$ if and only if $Y$ solves $\res_X(A, B)$, which is a
  polynomial in $Y$, since $A$ and $B$ both have coefficients in
  $K[Y]$ when viewed as polynomials in $X$. Therefore, we reduced two
  equations in two unknowns to one equation in one unknown. Similar
  argument applies when there are more variables.
\end{remark}

%%% Local Variables:
%%% TeX-master: "report"
%%% End:


\section{Computational results for quadratic periodic points\\ with
  rational $c$}

We want to search, for rational $c$, quadratic periodic points with
exact period $N$, where by quadratic points we mean elements of
some quadratic number field $K/\Q$. This boils down to finding
quadratic points on $C_1(N)$, which is the smooth projective model of
$\Phi_N(z, c)$. There are several approaches.

\subsection{Enumerating $c \in \Q$}

The first possible approach is to enumerate $c \in \Q$. A natural
order of enumeration is the increasing order of height, where the
height of a rational number $p/q$, with $p,\, q \in \Z$ and $\gcd(p,
q) = 1$, is defined as $\max\set{|p|,\, |q|}$.

For each fixed $c$, we factorize the polynomial $\Phi_{N,c}(z)$ over
$\Q$ and look at its irreducible factors. Let $D_{N,c}$ be the
multiset of the degrees of $\Phi_{N,c}(z)$ irreducible
factors. Clearly the elements of $D_{N,c}$ sum to $\deg \Phi_{N,c}(z)
= \sum_{d|N}2^{d \mu(N/d)}$. Then $z^2 + c$ has quadratic periodic
points of exact period $N$ if and only if $2 \in D_{N,c}$. Factorizing
$\Phi_{N,c}(z)$ over $\Q$ to obtain $D_{N,c}$ is doable (polynomial
time algorithm exists) for all $N$ with software, e.g., Mathematica,
although the complexity grows exponentially with $N$ (note that $\deg
\Phi_{N,c}$ grows roughly as $2^N$).

We performed the search with $N = 4$ and 5, enumerating $c$ up to
height 1000. (See the Mathematica notebook \texttt{misc/mma/factor}.)

For $N = 4$, we found various $c$ with quadratic rational points:
$-4/5$, $-31/48$, $-95/48$, $-155/72$, $-209/72$. From this result it
is conjectured that there are infinitely many such $c$'s, and this is
later confirmed by theory. It is also worth noting that for $c$'s that
we observed, $D_{4,c}$ is always one of the following:
\[
\set{12},\, \set{4, 8},\, \set{2, 2, 8}.
\]
Note that the degrees of irreducible factors of $\Phi_{4,c}(z)$
(elements of $D_{4,c}$) always have common factors with $N = 4$. It is
further observed that $c$'s such that $\Phi_{4,c}(z)$ is reducible
always has ``fairly divisible'' denominators, i.e., their prime
factors are ``small'' compared to themselves. The implication of this
is not clear.

For $N=5$, no quadratic factors were observed, and it is conjectured
that there are no quadratic periodic points of exact period 5 for any
$c \in \Q$. Even the collection of $c$'s such that $\Phi_{5,c}(z)$ is
reducible is very limited:
\[
\begin{gathered}
  c = -2,\, D_{5,c} = \set{5,\, 10,\, 15},\\
  c = -4/3,\, D_{5,c} = \set{10,\, 20},\\
  c = -16/9,\, D_{5,c} = \set{5,\, 25},\\
  c = -64/9,\, D_{5,c} = \set{5,\, 25}.
\end{gathered}
\]
Note again that the degrees of irreducible factors of $\Phi_{5,c}(z)$ always have common
factors with $N = 5$.

We also performed the search for $N = 6$ up to height 500. Quadratic
factor was found for one $c$:
\[
c = -71/48,\, D_{6,c} = \set{2,\, 2,\, 2,\, 48},
\]
and two more $c$ were observed such that $\Phi_{6,c}(z)$ is reducible:
\[
\begin{gathered}
  c = -2,\, D_{6,c} = \set{6,\, 6\, 18,\, 24},\\
  c = -4,\, D_{4,c} = \set{6,\, 48}.
\end{gathered}
\]
Note once again that (from the limited data we have) the degrees of
irreducible factors of $\Phi_{6,c}(z)$ always have common factors with
$N = 6$.

% I will add bibliography and citations later
Also recall from Morton1 that
\begin{theorem}
  For $N = 3$ and any $c \in \Q$, $D_{3, c}$ must be one of the
  following:
  \[
  \set{6},\, \set{3,\, 3}, \set{3,\, 1,\, 1,\, 1}.
  \]
\end{theorem}

Combining our knowledge of $D_{3,c}$, $D_{4,c}$, $D_{5,c}$, and
$D_{6,c}$, we have the following conjecture.

\begin{conjecture}[Weak factorization]
  For $N \ge 3$ and any $c \in \Q$, one of the following holds:
  \begin{enumerate}
  \item $1 \in D_{N,c}$;
  \item For all $d \in D_{N,c}$, $\gcd(d,\, N) > 1$.
  \end{enumerate}
\end{conjecture}

$N = 3$ is still rather special, since there are inifinitely many $c$
with rational periodic points of exact period 3, and we know the
explicit parametrization of $c$ such that $\Phi_{3,c}(z)$ is
reducible (Morton1). If we exclude $N = 3$, we have the following
stronger conjecture.

\begin{conjecture}[Strong factorization]
  For $N \ge 4$ and any $c \in \Q$, we have $\gcd(d, N) > 1$ for all
  $d \in D_{N,c}$.
\end{conjecture}

This conjecture implies FPS's conjecture.

\begin{conjecture}[FPS]
  For $N \ge 4$ and any $c \in \Q$, $1 \not\in D_{N,c}$.
\end{conjecture}

However, these conjectures seem to require radically new ideas not
available at this moment, as most current results in this field relies
on manipulating specific curves.

\subsection{Enumerate quadratic $x$ on $C_0(5)$}

Now we specialize to $N = 5$, which is the simplest case not yet
resolved. For $N = 5$, FPS already worked out a hyperelliptic model
for $C_0(5)$:
\[
y^2 = f(x) = x^6 + 8x^5 + 22x^4 + 22x^3 + 5x^2 + 6x + 1.
\]
$c$ is given by
\[
c = \frac{P_0(x) + P_(x) y}{h(x)} = \frac{g(x)}{2(P_0(x) - P_1(x) y)},
\]
where
\[
\begin{gathered}
  P_0(x) = - 9 - 24x - 95x^2 - 104x^3 - 46x^4 - 10x^5 - x^6,\\
  P_1(x) = - 9 + 3x + 6x^2 + x^3,\\
  g(x) = 64 + 110x + 325x^2 + 452x^3 + 271x^4 + 74x^5 + 8x^6,\\
  h(x) = 8x^2(3 + x)^2.
\end{gathered}
\]
Note that quadratic periodic points of exact period 5 (more precisely,
orbits of such quadratic points) map to quadratic points on $C_0(5)$,
so we may enumerate quadratic points on $y^2 = f(x)$ and verify if $c
\in \Q$ is satisfied and if the preimage of such a quadratic point on
$C_0(5)$ is indeed a quadratic orbit on $C_1(5)$ (rather than a
$\gal(\bar{\Q}/\Q)$-fixed orbit living in a degree-10 number field).

To enumerate quadratic points on $C_0(5)$, we fix a squarefree $d \in
\Z$, and enumerate $x \in K = \Q(\sqrt{d})$. We still enumerate in the
increasing order of height, where the height of $\frac{p_1}{q_1} +
\frac{p_2}{q_2} \sqrt{d} \in \Q(\sqrt{d})$, with $p_1$, $q_1$, $p_2$,
$q_2 \in \Z$ and $\gcd(p_1,\, q_1) = \gcd(p_2,\, q_2) = 1$, is defined
as $\max\set{|p_1|,\, |q_1|,\, |p_2|,\, |q_2|}$ as usual. For each
$x$, we then compute $y^2$ from the relation $y^2 = f(x)$. Since we
want to have $y \in K = \Q(\sqrt{d})$, we need to check if $f(x)$ is a
square in $K$. This turned out to be a rather expensive task
(equivalent to factorizing $T^2 - f(x)$ in $K[T]$), but $N(f(x))$ is a
rational square is a great criterion to exclude ineligible $f(x)$,
where $N$ is the conventional norm in $\Q(\sqrt{d})$. Therefore, the
main workflow boils down to enumerating $d$, $p_1$, $q_1$, $p_2$,
$q_2$, perform multiprecision rational arithmetic (calculating $f(x)$
and $N(f(x))$), and test if the numerator and denominator of $N(f(x))$
are integer squares.

We implemented this same procedure in several languages --- Sage,
Mathematica, and C++ (using FLINT 2). On darwin-x86\_64 with 2.9 GHz
Intel Core i7 processor, FLINT 2.4.4, Mathematica 9.0.1 Student
Edition, and Sage 6.3, it turned out that Mathematica was consistently
about 20 times faster than Sage, and C++ with FLINT (using fmpzxx and
fmpqxx) was consistently about 50 times faster than Mathematica (from
which we conclude that FLINT is the winner performance-wise when it
comes to multiprecision rational arithmetic).

With the C++ program we were able to enumerate squarefree $d \in
\set{-100, \dots, 100}$ and $x \in \Q(\sqrt{d})$ up to height 30, and
for specific $d$ of interest, i.e., $d = -87$ and $d = 33$,
% note that I recently discovered d = -87 and haven't get arount to
% really do it up to height 100 just yet
up to height 100. Only eight quadratic points (excluding all the
rational points already known in FPS) on $C_0(5)$ were found, four of
which give rational $c$:
\[
\begin{gathered}
  d = -87,
  x = -\frac{1}{6} \sqrt{-87} - \frac{1}{2},
  y = -\frac{5}{9} \sqrt{-87} + \frac{22}{3},
  c = \frac{13371}{86528} \sqrt{-87} + \frac{12767}{86528},
  \\
  d = -87,
  x = -\frac{1}{6} \sqrt{-87} - \frac{1}{2},
  y = \frac{5}{9} \sqrt{-87} - \frac{22}{3},
  c = -\frac{4}{3},
  \\
  d = -87,
  x = \frac{1}{6} \sqrt{-87} - \frac{1}{2},
  y = \frac{5}{9} \sqrt{-87} + \frac{22}{3},
  c = -\frac{13371}{86528} \sqrt{-87} + \frac{12767}{86528},
  \\
  d = -87,
  x = \frac{1}{6} \sqrt{-87} - \frac{1}{2},
  y = -\frac{5}{9} \sqrt{-87} - \frac{22}{3},
  c = -\frac{4}{3},
  \\
  d = 33,\,
  x = -\frac{1}{3} \sqrt{33} - 2,\,
  y = \frac{10}{9} \sqrt{33} + \frac{17}{3},\,
  c = -2,
  \\
  d = 33,\,
  x = -\frac{1}{3} \sqrt{33} - 2,\,
  y = -\frac{10}{9} \sqrt{33} - \frac{17}{3},\,
  c = \frac{269}{128} \sqrt{33} - \frac{6415}{384},
  \\
  d = 33,\,
  x = \frac{1}{3} \sqrt{33} - 2,\,
  y = \frac{10}{9} \sqrt{33} - \frac{17}{3},\,
  c = -\frac{269}{128} \sqrt{33} - \frac{6415}{384},
  \\
  d = 33,\,
  x = \frac{1}{3} \sqrt{33} - 2,\,
  y = -\frac{10}{9} \sqrt{33} + \frac{17}{3},
  c = -2.
\end{gathered}
\]
$c = -\frac{4}{3}$ and $-2$ can be shown to not yield any quadratic
periodic points of period 5.


\section{6-Cycles in Quadratic Fields}

\begin{conjecture}[Galois Conjugacy]
  \label{conj:gc}
  Let $\{x_1, x_2, x_3, x_4, x_5, x_6\}$ be a $6$-cycle over a
  quadratic field $K$ for $\phi(x) = x^2 + c$ with $c \in \Q$. Then
  $x_4 = \overline{x_1}$.
\end{conjecture}

\begin{corollary}[Conditional Classification of 6-Cycles over Quadratic Fields]
  \label {cor:6-cycles}
  Conjecture \ref{conj:gc} and the weak BSD Conjecture, imply that the
  only 6-cycle over quadratic fields is in $K = \Q(\sqrt{33})$ and is
  given by:
  \[
  \phi(x) = x^2 - \frac{71}{48}, x_1 = -1 + \frac{\sqrt{33}}{12}
  \]
\end{corollary}

\begin{proof}
  Let $\{x_1, x_2, x_3, x_4, x_5, x_6\}$ be a $6$-cycle over a
  quadratic field $K$ and define $t = x_1 + x_2 + x_3 + x_4 + x_5 +
  x_6$ to be the trace of the orbit. By Conjecture \ref{conj:gc}, $x_4
  = \overline{x_1}, x_5 = \overline{x_2}, x_6 = \overline{x_3}$
  . Thus, we have
  \[
  t = x_1 + x_2 + x_3 + x_4 + x_5 + x_6 = x_1 + x_2 + x_3 +
  \overline{x_1} + \overline{x_2} + \overline{x_3} \in \Q
  \]
  Therefore, each 6-cycle can be represented as a $\Q$-point on $C_0
  (6)(\Q)$ via its trace. Using Stoll's characterization of the
  $\Q$-points of $C_1(6)$(cite), we know that the 6-cycle given by $K
  = \Q(\sqrt{33}), \phi(x) = x^2 - \frac{71}{48}, x_1 = -1 +
  \frac{\sqrt{33}}{12}$ is the only 6-cycle over a quadratic field and
  not over $\Q$.  Stoll finds, using a proof conditional on the weak
  BSD Conjecture, that there are no 6-cycles over $\Q$ either.
\end{proof}

\begin{lemma}
	Let $\{x_1, \ldots, x_{N}\}$ be an exact $N$-cycle defined over
	a Galois number field K, with N divisible by	$d = [K:\Q ]$. Then either: \\
	\begin{itemize}
	\item $x_{\frac{iN}{d}+1} = \sigma(x_1)$ for some nontrivial $\sigma \in
	\gal(K/\Q)$, $i \in \Z$
	\item $\{x_1, \ldots, x_{N}\} \cap \{\tau(x_1), \ldots, \tau(x_{N})\} =
	\emptyset$, $\forall \tau \in \gal(K/\Q)$ nontrivial.
	\end{itemize}
\end{lemma}

\begin{proof}
	Assume that neither of the above are true, so $x_{\frac{N}{d}+1}
	\neq \sigma(x_1)$ $\forall \sigma \in \gal(K/\Q)$ and $\{x_1, \ldots
	, x_{N}\} \cap \{\tau(x_1), \ldots, \tau(x_{N})\} \neq \emptyset$ $
	\forall \tau \in \gal(K/\Q)$. Then $\exists x_j \in \{x_1, \ldots,
	x_{N}\} \cap \{\tau(x_1), \ldots, \tau(x_{N})\}$ for some $\tau
	\in \gal(K/\Q)$. Applying $\phi$ an appropriate number of times
	allows us to renumerate the cycle	such that $x_1$ is in the
	intersection. Thus, $x_k = \tau(x_1)$ for some $1 \leq k \leq N$
	and therefore $\tau \equiv \phi^{k-1}$ on the N-cycle (because Galois
	conjugation commutes with $\phi$). Since $K$ is Galois,
	$\tau^d = Id$ and so:
	\[
		x_d = \tau^d(x_d) = \phi^{dk-d}(x_d) = x_{dk}
	\]
	Notice that the application of $\phi^{N-(d-1)}$ yields $x_1 =
	x_{dk-(d-1)} = x_{d(k-1) + 1}$. Notice that this produces a
	contradiction if $d(k-1) + 1 \not\equiv 1$ (modulo $N$): if $l$
	is the representative of $d(k-1) + 1$ modulo N, then $\{x_1, \ldots
	, x_l\}$ gives an $l$-cycle with $l < N$. Thus, it must be that
	$d(k-1) + 1 \equiv 1$ (modulo $N$). The only such $k$ are given
	by $k = \frac{lN}{d} + 1$ for some $l \in \Z$, so $x_{\frac{lN}{d}
	+ 1} = x_k = \tau(x_1)$.	However, we already have that
	$x_{\frac{iN}{d}+1} \neq \sigma(x_1)$ $\forall \sigma \in \gal(K/\Q)$,
	$i \in \Z$. This gives a contradiction.
\end{proof}

\begin{remark}[N = 6 case]
	Stoll conditionally proved that there are no rational points with
	exact period 6 assuming the weak form of the BSD Conjecture(cite).
	This allows us to use this lemma to conditionally prove Conjecture
	\ref{conj:gc} and thus Corollary \ref{cor:6-cycles}. All that
	remains to be shown is that$\{x_1, \ldots, x_{N}\} \cap \{\overline{
	x_1}, \ldots, \overline{x_{N}}\} = \emptyset$ is impossible.
\end{remark}

In general, we conjecture that:
\begin{conjecture}
	Let $\{x_1, \ldots, x_{N}\}$ be an exact $N$-cycle defined over
	number field K, with N divisible by	$d = [K:\Q ]$. Then it is
	never the case that $\{x_1, \ldots, x_{N}\} \cap \{\tau(x_1), \ldots
	, \tau(x_{N})\} = \emptyset$, $\forall \tau \in \gal(K/\Q)$. In
	particular, this implies that $x_{\frac{N}{d}+1} =\sigma(x_1)$ for
	some $\sigma \in \gal(K/\Q)$. Therefore the rational points on
	$C_0(N)$ are sufficient to find points in $K$ of exact period $N$.
\end{conjecture}

\section{Key Results}

%Entire section to be expanded upon and written more nicely%
For N = 5, we took $P_{c,q}$ and did long division with general
quadratic $q^2 + aq + b$ and then took the resultant of the linear and
constant remainders to get a polynomial equation in $a$ and $b$. SAGE
computed that the genus of the resulting curve is 11. Thus there are
finitely many rational triples $(a,b,c)$ that satisfy both
equations. Thus there are finitely many quadratic points in $C_0(5)$
and thus finitely many quadratic points of exact period 5.
\begin{theorem}
  There are finitely many points in quadratic fields with exact period
  5 for $\phi(x) = x^2 + c$ with $c \in \Q$.
\end{theorem}

A result that follows directly from the parametrization of 4-cycles
given by Morton (cite):
\begin{theorem}
  There are infinitely many points in quadratic fields with exact
  period 4 for $\phi(x) = x^2 + c$ with $c \in \Q$.
\end{theorem}

Plugging in the $c$ from Morton's parametrization and factoring
$\Phi_4(x,c)$ with respect to $x$ yields:
\begin{lemma}
  For $c$ such that $\phi$ has 4-cycles over quadratic fields, the
  degree 12 polynomial $\Phi_4(x,c)$ must factor as $\{2,2,8\}$ (list
  of degrees of factors, with the 8th degree factor not necessarily
  irreducible).
\end{lemma}

Using Panraksa's theorem that $\Phi_4(x,c)$ cannot factor as
$\{2,2,2,2,4\}$, we obtain the following result:
\begin{theorem}
  For each $c \in \Q$, there is at most one 4-cycle over quadratic
  fields.
\end{theorem}

\bibliography{bibliography}{}
\bibliographystyle{plain}

\end{document}
