\documentclass{amsart}


%%%%%%%%%%%%%%%%%%%%%%%%%%%%%% packages %%%%%%%%%%%%%%%%%%%%%%%%%%%%%%

% Here are some common packages used in almost all documents.
\usepackage{amssymb}
\usepackage{graphicx}
\usepackage{color}
\usepackage[colorlinks]{hyperref}


%%%%%%%%%%%%%%%%%%%%%%% amsthm theorem styles %%%%%%%%%%%%%%%%%%%%%%%%

% Current styles are based on suggestions by the amsthm package
% authors.
\theoremstyle{plain}
\newtheorem{theorem}{Theorem}[section]
\newtheorem{lemma}[theorem]{Lemma}
\newtheorem{proposition}[theorem]{Proposition}
\newtheorem{corollary}[theorem]{Corollary}
\newtheorem{claim}[theorem]{Claim}

\theoremstyle{definition}
\newtheorem{definition}[theorem]{Definition}
\newtheorem{example}[theorem]{Example}
\newtheorem{conjecture}[theorem]{Conjecture}

\theoremstyle{remark}
\newtheorem{remark}[theorem]{Remark}
\newtheorem{note}[theorem]{Note}
\newtheorem{case}{Case}


%%%%%%%%%%%%%%%%%%%%%%%%% custom definitions %%%%%%%%%%%%%%%%%%%%%%%%%

% Since we are collaborating, we need to be careful with personal
% shortcuts to avoid clash.
\newcommand{\C}{\mathbb{C}}
\newcommand{\F}{\mathbb{F}}
\renewcommand{\P}{\mathbb{P}}
\newcommand{\Q}{\mathbb{Q}}

\newcommand{\preper}{\mathrm{PrePer}}
\newcommand{\gal}{\mathrm{Gal}}

\newcommand{\tup}[1]{\left<#1\right>}


%%%%%%%%%%%%%%%%%%%%%%%%%%% document body %%%%%%%%%%%%%%%%%%%%%%%%%%%%

\begin{document}

\section{6-Cycles in Quadratic Fields}

\begin{conjecture}[Galois Conjugacy]
	\label{conj:gc}
	Let $\{x_1, x_2, x_3, x_4, x_5, x_6\}$ be a $6$-cycle over a 
	quadratic field $K$. Then 
	$x_4 = \overline{x_1}, x_5 = \overline{x_2}, x_6 = \overline{x_3}$
\end{conjecture}

\begin{corollary}[Conditional Classification of 6-Cycles over Quadratic Fields]
	The above conjecture implies that the only 6-cycle over quadratic fields is in $K = \Q(\sqrt{33})$ and is given by:
	\[
		\phi(x) = x^2 - \frac{71}{48}, x_1 = -1 + \frac{\sqrt{33}}{12}
	\]
\end{corollary}

\begin{proof}
	Let $\{x_1, x_2, x_3, x_4, x_5, x_6\}$ be a $6$-cycle over a 
	quadratic field $K$ and define $t = x_1 + x_2 + x_3 + x_4 + x_5 + x_6$
	to be the trace of the orbit. By \ref{conj:gc}, 
	$x_4 = \overline{x_1}, x_5 = \overline{x_2}, x_6 = \overline{x_3}$. 
\end{proof}

\end{document}
