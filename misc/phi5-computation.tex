\documentclass{amsart}


%%%%%%%%%%%%%%%%%%%%%%%%%%%%%% packages %%%%%%%%%%%%%%%%%%%%%%%%%%%%%%

% Here are some common packages used in almost all documents.
\usepackage{amssymb}
\usepackage{graphicx}
\usepackage{color}
\usepackage{hyperref}

\usepackage{enumerate} % customizable styles for counter printing


%%%%%%%%%%%%%%%%%%%%%%% amsthm theorem styles %%%%%%%%%%%%%%%%%%%%%%%%

% Current styles are based on suggestions by the amsthm package
% authors.
\theoremstyle{plain}
\newtheorem{theorem}{Theorem}[section]
\newtheorem{lemma}[theorem]{Lemma}
\newtheorem{proposition}[theorem]{Proposition}
\newtheorem{corollary}[theorem]{Corollary}
\newtheorem{claim}[theorem]{Claim}

\theoremstyle{definition}
\newtheorem{definition}[theorem]{Definition}
\newtheorem{example}[theorem]{Example}
\newtheorem{conjecture}[theorem]{Conjecture}

\theoremstyle{remark}
\newtheorem{remark}[theorem]{Remark}
\newtheorem{note}[theorem]{Note}
\newtheorem{case}{Case}


%%%%%%%%%%%%%%%%%%%%%%%%% custom definitions %%%%%%%%%%%%%%%%%%%%%%%%%

% Since we are collaborating, we need to be careful with personal
% shortcuts to avoid clash.
\newcommand{\C}{\mathbb{C}}
\newcommand{\F}{\mathbb{F}}
\newcommand{\N}{\mathbb{N}}
\renewcommand{\O}{\mathcal{O}}
\renewcommand{\P}{\mathbb{P}}
\newcommand{\Q}{\mathbb{Q}}
\newcommand{\U}{\mathcal{U}}
\newcommand{\Z}{\mathbb{Z}}

\renewcommand{\l}{\lambda}

\newcommand{\es}{\emptyset}
\newcommand{\ol}{\overline}

\newcommand{\gal}{\mathrm{Gal}}
\newcommand{\preper}{\mathrm{PrePer}}
\newcommand{\res}{\mathrm{Res}}

\newcommand{\set}[1]{\left\{#1\right\}}
\newcommand{\tup}[1]{\left<#1\right>}


\title{Computational methods for degree 2 $K/\Q$ points on
  $\Phi_{5,c}(z)$}
\author{Zhiming Wang}
\date{\today}

\begin{document}

\maketitle

Niccol\`o suggested that we make use of the hyperelliptic curve
\[
\mathcal{C}: y^2 = x^6 + 8x^5 + 22x^4 + 22x^3 + 5x^2 + 6x + 1,
\]
which is birationally equivalent to $C_0(5)$ (FPS p.5). $c$ is given
in terms of $(x, y)$ on $\mathcal{C}$ by
\[
c = \frac{P_0(x) + P_1(x) y}{8x^2(3+x)^2},
\]
where
\[
\begin{gathered}
  P_0(x) = -9 - 24x - 95x^2 - 104x^3 - 46x^4 - 10x^5 - x^6,\\
  P_1(x) = -9 + 3x + 6x^2 + x^3.
\end{gathered}
\]

Niccol\`o suggested that maybe it is possible to use software to find
all $K$ points on $\mathcal{C}$, given a specific $K =
\Q(\sqrt{d})$. Then we can compute $c$ from these points, and then by
a miracle, we might find some rational $c$.

Using this method, we will be enumerating over $d$, which are
squarefree integers; and looking for $K$ points on a
not-too-horrible curve $\mathcal{C}$. It might be faster than brute
forcing all rational $c$.

\end{document}
